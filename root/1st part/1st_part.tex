\chapter{Multivalent braids and wild character varieties of Sibuya type}
\section{Basic terminologies} 
\begin{definition}
Let $\omega$ be a braid word, then we have an embedding of the braid diagram in $[0,1]_x \times \R_z$, co-oriented below, as an \emph{open front diagram} of $\omega$.
\end{definition}
\begin{definition}
	Let $\omega$ be a braid word representing a braid $\beta\in Br_n^+$ i.e. $[\omega] = \beta$. Then we define $Q^o_{\omega}$ to be the quiver 
\begin{itemize}
		\item whose vertices are labeled by the regions of the open front projection
		\item whose arrows are labeled by pairs of vertices whose corresponding regions are adjacent(bordered by the front projection of the braid) subject to the condition that the arrows always go against the co-orientation(hairs).
	\end{itemize}
There are two distinguished vertices of $Q^o_\omega$. We denote the vertex corresponding to the region $z\rightarrow \infty$(resp. $z\rightarrow -\infty$) as $U$(resp. $D$).
\end{definition}

\begin{definition}
	Locally for each crossing $c$, there is a region all the hairs are pointing outward, we call this $N_c$(read north of $c$). Starting from $N_c$, as we move counter-clockwise about the crossing, we call the corresponding regions $N_c,E_c,S_c,W_c$ respectively(read north of, east of, south of, west of $c$).
\end{definition}
\begin{theorem}
	Suppose we have a fixed braid word $\omega$ and $v$ is a vertex corresponding to a region given by the front projection of $\omega$. Any path from $d$ to a vertex $v$ have same length.
\end{theorem}
\begin{proof}
we prove the statement by the induction on the length of the braid words. The statement is trivial for trivial braid word because there is only one path from d to any vertex. Suppose the statement is true for all quivers associated to braid words whose length is less than $n$. Suppose $\omega$ is a braid word of length $n$. Suppose there are two distinct paths $p_1$ and $p_2$ from $d$ to $v$. Let $\omega = s_{i_1}\cdots s_{i_n}$. There are two cases to consider :\\
(case1) $v$ is the region east of the new crossing generated by $s_{i_n}$. Then the paths to $v$ must have passed the region right below the region of $v$ because that's the only possible way to get to $v$ under the constraint that the arrow always go against the co-orientation. If we remove the last edge and vertex from the paths $p_1,p_2$, we have paths $p'_1,p'_2$ ending at the same vertex(i.e. south of the crossing generated by $s_{i_n}$) and are entirely contained in the subquiver $Q^o_{\omega '}$ where $\omega ' = s_{i_1}\cdots s_{i_{n-1}}$. Therefore, by the induction hypothesis the lengths of $p'_1$ and $p'_2$ are the same which immediately implies $\operatorname{length}(p_1) = \operatorname{length}(p'_1) + 1 =\operatorname{length}(p'_2) + 1 = \operatorname{length}(p_2)$.\\
(case2) Suppose $v$ is not the region that is the east of the crossing generated by $s_{i_n}$. Without loss of generality, we can assume that two paths do not pass through the region east of the crossing, because we can always replace the part of the path $s_c\rightarrow e_c \rightarrow n_c$ with $S_c\rightarrow W_c \rightarrow N_c$ having the same length. Then once we know that two paths $p_1$ and $p_2$ do not pass through the region east of the crossing, we know that they are entirely contained in the subquiver $\omega ' = s_{i_1}\cdots s_{i_{n-1}}$. Then the $\operatorname{length}(p_1) = \operatorname{length}(p_2)$ by the induction hypothesis.
\end{proof}


\begin{definition}
 The valency of the vertex of a quiver is the number of incoming arrows. The height of the vertex of a quiver is the length of a path starting from $d$ ending at that vertex which is well-defined by the previous theorem. Note that for every crossing $c$, $E_c$ and $W_c$ have the same height. 
\end{definition}

\begin{definition}
 We say two vertices are adjacent if there is a crossing $c$ such that two vertices are $E_c$ and $W_c$ of this crossing. Let $k$ be a positive integer, then we define a natural ordering on the set of all height $k$ vertices generated by the relations, for each crossing $W_c \leq E_c $.
 
\end{definition}

\begin{theorem}\label{quoted1}
	The above ordering is well-defined and is a total ordering on the set of all height $k$ vertices.
\end{theorem}

\begin{proof} 
To prove the claim, we have to prove the following facts :
\begin{enumerate}[label=(\roman*)]
	\item For any two distinct points of height $k$, there is a chain of crossings connecting two points.
	\item there is no chain of crossings starting at a point and ending at the same point.
\end{enumerate}
We prove the claim by induction on the length of braid words. For the trivial braid, (\Rn{1}) holds because there is only one vertex for each height and (\Rn{2}) holds because there is no arrow starting from that unique and ending at the point.\\
 Now suppose the claim holds for all $\operatorname{length} < n$ braids and suppose $\Rn{1}$ does not hold for the braid word $\omega = s_{i_1}\cdots s_{i_n}$. Then one of the two vertices should be the vertex corresponding to $E_c$ of the crossing generated by $s_{i_n}$. Let's call this vertex $v$ and the other as $v'$. Then we know that by the induction hypothesis there is a chain of crossing connecting $v'$ and $W_c$. Since $W_c$ is connected by the crossing $c$ to $E_c = v$ and $v$, $v'$ are connected by a chain of crossings which is a contradiction. \\
 Now suppose the claim holds for all $\operatorname{length} < n$ braids and suppose $\Rn{2}$. Suppose there is a point where there is a chain of crossing starting and ending at the same point. By the induction hypothesis, the chain of crossing generated by $s_{i_n}$ along the way. Without loss of generality, using cyclic shift we can assume the chain of crossings starts from $e_c$ which is a contradiction because $E_c$ is not the $W_{c'}$ of any crossing $c'$.
\end{proof}


\begin{definition}
	 There are two distinguished $A_n$ sub-quivers of $Q^o_{\omega}$. We denote $R_{\omega}$(resp. $L_{\omega}$) to be the full sub-quiver containing all the vertices corresponding to the rightmost(resp. leftmost) regions. Alternatively, the path following the largest(resp. smallest) arrows at each step. We denote the vertex of $R_{\omega}$(resp. $L_{\omega}$) of height $k$ by $R^k_{\omega}$(resp. $L^k_{\omega}$).
\end{definition}

\begin{definition}
	Let $\mathcal{M}^{fr}_{\omega}$ be the framed moduli space classifying pairs $(F,g)$ where $F$ is the representation of $Q^o_{\omega}$ and $g\in \operatorname{GL}_n(\mathbb{C})$ subject to the following conditions:
	\begin{itemize}
		\item For each vertex the vectorspace associated to it is a subspace of $\mathbb{C}^n$ of dimension equal to its height.
		\item  All maps are inclusion maps.
		\item  $gF(R^k_{\omega})=F(L^k_{\omega})$ for all $k=0,1,\cdots, n$.
		\item For each crossing $c$, then the sequence $0\rightarrow F(S_c)\rightarrow F(E_c)\oplus F(W_c)\rightarrow F(N_c)\rightarrow 0$ is a short exact sequence where maps are induced by the inclusion maps.
	\end{itemize}
\end{definition}

\begin{remark}
	There is a natural left action of $x\in \operatorname{GL}_n(\mathbb{C})$ on $(F,g)\in\mathcal{M}^{fr}_{\omega}$, that is, $x\cdot(F,g)=(xF,xGx^{-1})$ where $xF$ is left translation on quiver representation and $xgx^{-1}$ is conjugation.
\end{remark}

\begin{theorem}
\cite[Prop.~1.5]{shende2017legendrian} $\mathcal{M}_1(S^1_x \times \R_z,\Lambda_\omega,\{\sigma_{z\ll 0}\};\C)\cong \operatorname{GL}_n(\mathbb{C})\backslash \mathcal{M}^{fr}_{\omega}$.
\end{theorem}

Now consider the following map
\begin{definition}
Suppose we have the braid word $\omega$ of a braid $\beta$. Let $\{v_1,v_2,\cdots,v_m\}$ be the complete list of all height 1 vertices of $Q^o_{\omega}$ such that $v_i<v_j$ if and only if $i<j$. We define $\iota_\omega$ to be the  forgetful map 
\begin{align*}
\iota_{\omega}~:~ &\mathcal{M}^{fr}_{\omega}\rightarrow(\mathbb{P}^{n-1})^m\times \operatorname{GL}_n(\mathbb{C})\\
& (F,g) \mapsto ([F(v_1)],\cdots,[F(v_m)], g)
\end{align*}
\end{definition}

\begin{definition}
	Let $\omega$ be a braid word. We denote the infinite cyclic copies of the original braid word $\omega$ by $\omega^\infty$. More precisely, suppose $\omega$ is a braid word given by a collection of sections $\{\sigma_i : [0,1]_x\rightarrow [0,1]_x\times \mathbb{R}^2_{y,z}\}_{i =1,\dots,n}$, then $\omega^\infty$ is given by the collection $\{\overline{\sigma_i} : \mathbb{R}_x\rightarrow \mathbb{R}^3_{x,y,z} ~|~ \overline{\sigma}(x,y,z) = \sigma(x-[x],y,z)\}_{i =1,\dots,n}$. Again, I will abuse the notation $\omega^\infty$ to denote its front projection onto $\mathbb{R}^2_{x,z}$. 
\end{definition}	
	
	
	
\begin{definition}	
	Let $\omega$ be a braid word, then we have the quiver $Q^o_\omega$ associated to it. We define the quiver $Q_\omega^\infty$ to be the quotient of  $\prod_{i \in \mathbb{Z}} Q^o_\omega$ by the relations $\{R_{\omega,i} = L_{\omega,i+1}\}_{i\in \mathbb{Z}}$ where $R_{\omega,i}$(resp. $L_{\omega,i+1}$) is the subquiver $R_{\omega}$(resp. $L_{\omega}$) in the $i^{th}$(resp. $i+1^{th}$) copy of $Q^o_\omega$ in $\prod_{i \in \mathbb{Z}} Q^o_\omega$. Therefore, we have the quotient map of quivers $q : \prod_{i \in \mathbb{Z}} Q^o_\omega \rightarrow Q_\omega^\infty$. Let the signature function $\sigma' : \operatorname{Vert}(\prod_{i \in \mathbb{Z}} Q_\omega) \rightarrow \mathbb{Z}$ be $\sigma'(v) = i$ if $v$ is in the $i^{th}$ copy of $Q^o_\omega$ in $\prod_{i \in \mathbb{Z}} Q_\omega$. Define $\sigma : Q_\omega^\infty \rightarrow \mathbb{Z}$ to be $\sigma(v) = \operatorname{min}_{w\in q^{-1}(v)} \sigma'(w)$ if $|q^{-1}(v)| < \infty$ and $0$ otherwise. Note that for $v\in \operatorname{Vert}(Q_\omega^\infty)$, $q^{-1}(v)$ is infinite if and only if $R_\omega^{\operatorname{height}(v)} = L_\omega^{\operatorname{height}(v)}$ i.e. there is a unique vertex of $\operatorname{height}(v)$ in $Q^o_\omega$. We can think of $Q^o_\omega$ as the full subquiver of $Q_\omega^\infty$ spanned by the signature $0$ vertices. 
\end{definition}

\begin{definition}	
Suppose we have a quiver representation $F_\omega$ of $Q^o_\omega$, then we define the induces quiver representation $F^{\infty}_{\omega}$ of the quiver $Q^{\infty}_{\omega}$ to be $F_\omega^\infty(v) := g^{\sigma(v)}\cdot F_\omega(v - \sigma(v))$. 
\end{definition}	
Under the above correspondence, we get an isomorphism $\mathcal{M}^{fr}_\omega \cong \mathcal{M}^{fr,\infty}_\omega$.
\begin{definition}
$\Upsilon^k_0$ is the set of all height $k$, signature zero vertices in $Q^{\infty}_{\omega}$. For each vertex $v$ of $Q^{\infty}_{\omega}$, we define $I_k(v)$ to be the set of all the height $k$ vertices that have paths to $v$.
\end{definition}

\section{Multivalent braid words}                            
\begin{definition}
 A braid word $\omega$ is multivalent if and only if all the vertices of height greater than $1$ in $Q^\infty_\omega$ has valencies greater than $1$.
\end{definition}

\begin{theorem}
	If the braid word $\omega$ is multivalent, then $\iota_{\omega}$ is an embedding.
\end{theorem}
\begin{proof}
It is enough to prove that once we specify vectorspaces to height $1$ vertices of $Q^o_\omega$ and $g\in \operatorname{GL}_n(\mathbb{C})$, the quiver representation of $Q^o_\omega$ extending the above datum is unique. Since $\mathcal{M}_\omega^{fr,\infty}\cong \mathcal{M}_\omega^{fr}$, it is enough to prove once we specify vectorspaces to signature 0 height 1 vertices of $Q_\omega^\infty$ and $g\in \operatorname{GL}_n(\mathbb{C})$, the quiver representation $F_\omega^\infty$ of $Q_\omega^\infty$ extending the above datum is unique. Since $(F_\omega^\infty,g)\in \mathcal{M}_\omega^{fr,\infty}$, for $v\in \operatorname{Vert}(Q_\omega^\infty)$, $F_\omega^\infty(v) = g^{\sigma(v)}\cdot F_\omega^\infty(v-\sigma(v))$. If $\operatorname{height}(v)=1$, then $v-\sigma(v)$ is height $1$ signature $0$ vertex. Therefore, $F_\omega^\infty(v)$ is uniquely determined. \\
Now we prove the statement by induction on the heights of vertices. Assume $\forall v\in \operatorname{Vert}(Q_\omega^\infty)$ with $\operatorname{height}(v)<h$, $F_\omega^\infty(v)$ are determined. Suppose $v\in \operatorname{Vert}(Q_\omega^\infty)$ such that $\operatorname{height}(v)=h>1$, then there are at least two vertices of height $h-1$ that have arrows to $v$, say $v'$ and $v''$. Without loss of generality, $v'\leq v''$. Then by $\omega^\infty,Q_\omega^\infty$-analogue of \ref{quoted1}, there is a chain of crossings $c_1,\dots,c_k$ and $v'=v_0,\dots,v_k=v''$ where $v_{i-1},v_i$ are west and east of the crossing $c_i$. Therefore, if we choose any $c_i$ to be $c$, then $v=N_c$. By the induction hypothesis, $F_\omega^\infty(W_c)$, $F_\omega^\infty(E_c)$.and $F_\omega^\infty(S_c)$ have already been specified because heights of $W_c$,$E_c$,and $S_c$ are $h-1$, $h-1$,and $h-2$ respectively. By the crossing condition,
\[
0\rightarrow F_\omega^\infty(S_c)\rightarrow F_\omega^\infty(E_c)\oplus F_\omega^\infty(W_c) \rightarrow F_\omega^\infty(N_c)\rightarrow 0
\]
$F_\omega^\infty(N_c)$ is uniquely determined.
\end{proof}

\begin{theorem}\label{quoted3}
	If the braid word $\omega$ is multivalent, then the image of $\iota_{\omega}$ is 
\begin{align*}
X_\omega = \{&((z_v)_{v\in \Upsilon_0^1},g) \in \prod_{v\in \Upsilon_0^1} \mathbb{P}_v^{n-1}\times \operatorname{GL}_n(\mathbb{C})~| \\
& \forall u\in \operatorname{Vert}(Q_\omega^\infty),~\operatorname{dim}_\mathbb{C}(\sum_{v\in I_1(u)}g^{\sigma(v)}\cdot z_{v-\sigma(v)}) = \operatorname{height}(u),\\
& \forall c\in \operatorname{Cross}(\omega^\infty),~ \operatorname{dim}_\mathbb{C}(\sum_{v\in I_1(E_c)\cup I_1(W_c)}g^{\sigma(v)}\cdot z_{v-\sigma(v)}) = \operatorname{height}(N_c)\}
\end{align*}
where $\mathbb{P}^{n-1}_v$ is a copy of $\mathbb{P}^{n-1}$ labeled by $v\in \Upsilon_0^1$.
\end{theorem}
\begin{proof}
Instead of proving $\operatorname{Im}(\iota_\omega) = X_\omega$, I will prove that for $\iota'$, the map obtained by pre-composing the canonical isomorphism $\mathcal{M}_\omega^{fr,\infty}\cong \mathcal{M}_\omega^{fr}$ to $\iota_\omega$, $\operatorname{Im}(\iota') = X_\omega$.\\
First, let's prove $\operatorname{Im}(\iota')\subset X_\omega$. Recall that
\begin{align*}
\mathcal{M}_\omega^{fr,\infty} = \{
& (F_\omega^\infty,g)\in \operatorname{Rep}(Q_\omega^\infty)\times \operatorname{GL}_n(\mathbb{C})~|\text{All the maps of }F_\omega^\infty\text{ are inclusion maps},\\
&\forall v\in \operatorname{Vert}(Q_\omega^\infty),~\operatorname{dim}_\mathbb{C}(F_\omega^\infty(v)) = \operatorname{height}(v),\\
&\forall v\in \operatorname{Vert}(Q_\omega^\infty),~F_\omega^\infty(v+n) = g^n \cdot F_\omega^\infty(v),\\
&\forall c\in \operatorname{Cross}(\omega^\infty),~0\rightarrow F_\omega^\infty(S_c)\rightarrow F_\omega^\infty(E_c)\oplus F_\omega^\infty(W_c)\rightarrow F_\omega^\infty(N_c)\rightarrow 0 \\
&\text{are short exact sequences}
\}
\end{align*}
We need to show that for $(F_\omega^\infty,g)\in \mathcal{M}_\omega^{fr,\infty}$, $\iota'_\omega ((F_\omega^\infty,g)) = ((F_\omega^\infty(v))_{v\in \Upsilon_0^1},g)\in X_\omega$ i.e. 
\begin{itemize}
\item $\forall u \in \operatorname{Vert}(Q_\omega^\infty)$, $\operatorname{dim}_\mathbb{C}(\sum_{v\in I_1(u)} g^{\sigma(v)}\cdot F_\omega^\infty(v-\sigma(v))) = \operatorname{height}(u)$
\item $\forall c \in \operatorname{Cross}(\omega ^\infty)$, $\operatorname{dim}_\mathbb{C}(\sum_{v\in I_1(E_c)\cup I_1(W_c)} g^{\sigma(v)}\cdot F_\omega^\infty(v-\sigma(v))) = \operatorname{height}(N_c)$
\end{itemize}
It is enough to prove that
\begin{itemize}
\item $\forall u \in \operatorname{Vert}(Q_\omega^\infty)$, $\sum_{v\in I_1(u)} g^{\sigma(v)}\cdot F_\omega^\infty(v-\sigma(v)) = F_\omega^\infty(u)$
\item $\forall c \in \operatorname{Cross}(\omega ^\infty)$, $\sum_{v\in I_1(E_c)\cup I_1(W_c)} g^{\sigma(v)}\cdot F_\omega^\infty(v-\sigma(v)) = F_\omega^\infty(N_c)$
\end{itemize} 
because of the condition $\forall u\in \operatorname{Vert}(Q_\omega^\infty),~\operatorname{dim}_\mathbb{C}(F_\omega^\infty(u)) = \operatorname{height}(u)$ defining $\mathcal{M}_\omega^{fr,\infty}$. Moreover, $g^{\sigma(v)}\cdot F_\omega^\infty(v-\sigma(v)) = F_\omega^\infty(v)$ by the definition of $\mathcal{M}_\omega^{fr,\infty}$. Therefore, we need to prove that, 
\begin{enumerate}[label=(\roman*)]
\item $\sum_{v\in I_1(u)} F_\omega^\infty(v)=F_\omega^\infty(u)$
\item $\sum_{v\in I_1(E_c)\cup I_1(W_c)} F_\omega^\infty(v)=F_\omega^\infty(N_c)$
\end{enumerate}
Proof of (\Rn{1}) : For each $v\in I_1(u)$, there is a path from $v$ to $u$, say $v=v_0 \rightarrow \cdots \rightarrow v_k = u$. Since $F_\omega^\infty(v_i)\subset F_\omega^\infty(v_{i+1})$, $F_\omega^\infty(v)\subset F_\omega^\infty(u)$. Therefore, $\sum_{v\in I_1(u)} F_\omega^\infty(v) \subset F_\omega^\infty(u)$. \\
Conversely, we prove $F_\omega^\infty(u)\subset \sum_{v\in I_1(u)}F_\omega^\infty(v)$ by induction on the height of $u$. If $\operatorname{height}(u)=1$, the statement is trivial. Now suppose the statement holds for all $u\in \operatorname{Vert}(Q_\omega^\infty)$ of $\operatorname{height}(u) = h$. Since $\omega$ is multivalent, $u$ has at least two vertices of height $h-1$ that have arrows to $u$, say $v'$ and $v''$. Without loss of generality $v'\leq v''$. Then by $\omega^\infty, Q_\omega^\infty$-analogue of \ref{quoted1}, there is a chain of crossings $c_1,\dots,c_k$ and $v'=v_0,\dots,v_k=v''$ where $v_{i-1},v_i$ are west and east of the crossing $c_i$. Therefore, if we choose any $c_i$ to be $c$, then $u=N_c$ i.e. $v$ is the north of the crossing $c$. Then by the crossing condition of $M_\omega^{fr,\infty}$
\[
0\rightarrow F_\omega^\infty(S_c)\rightarrow F_\omega^\infty(E_c)\oplus F_\omega^\infty(W_c)\rightarrow F_\omega^\infty(N_c)\rightarrow 0
\]
we have $F_\omega^\infty(u) = F_\omega^\infty(N_c)=F_\omega^\infty(E_c)+F_\omega^\infty(W_c)$ by the induction hypothesis,
\begin{align*}
&F_\omega^\infty(E_c) = \sum_{v\in I_1(E_c)} F_\omega^\infty(v),~F_\omega^\infty(W_c) = \sum_{v\in I_1(W_c)} F_\omega^\infty(v)\\
\Rightarrow &~F_\omega^\infty(u) = \sum_{v\in I_1(E_c)\cup I_1(W_c)} F_\omega^\infty(v)
\end{align*}
because $E_c$ and $W_c$ have arrows to $u=N_c~\Rightarrow~I_1(E_C)\cup I_1(W_c)\subset I_1(u)$. Therefore,
\[
F_\omega^\infty(u) = \sum_{v\in I_1(E_c)\cup I_1(W_c)} F_\omega^\infty(v)\subset \sum_{v\in I_1(u)} F_\omega^\infty(v)
\]\\
Proof of (\Rn{2}) : By (\Rn{1}), we have
\begin{align*}
&~\sum_{v\in I_1(E_c)\cup I_1(W_c)} F_\omega^\infty(v) \\
&=~\sum_{v\in I_1(E_c)} F_\omega^\infty(v)~+~\sum_{v\in I_1(W_c)} F_\omega^\infty(v)\\
&=~F_\omega^\infty(E_c)~+~F_\omega^\infty(W_c)
\end{align*}
This is equal to $F_\omega^\infty(N_c)$ by the crossing condition of $\mathcal{M}_\omega^{fr,\infty}$. Therefore, the proof of (\Rn{2}) is complete.\\
Now let's prove $X_\omega \subset \operatorname{Im}(\iota'_\omega)$. Let $((z_v)_{v\in \Upsilon_0^1},g)$ be an arbitrary point of $X_\omega$. I will define a point $(F_\omega^\infty,g)$ such that $\iota'_\omega((F_\omega^\infty,g))=((z_v)_{v\in \Upsilon_0^1},g)$. We define a quiver representation $F_\omega^\infty$ to be 
\[
F_\omega^\infty(u):=\sum_{v\in I_1(u)}g^{\sigma(v)}\cdot z_{v-\sigma(v)}
\]
and all the arrows of $F_\omega^\infty$ are inclusion maps. The inclusion maps are well-defined because if there is an arrow from $u$ to $u'$, then $I_1(u)\subset I_1(u')$.\\
Note that if $u\in\Upsilon_0^1$ i.e. $\operatorname{height}(u)=1$ and $\sigma(u) = 0$, then 
\[
F_\omega^\infty(u):=\sum_{v\in \{u\}}g^{\sigma(v)}\cdot z_{v-\sigma(v)} = g^{\sigma(u)}\cdot z_{u-\sigma(u)} 
\]
If $(F_\omega^\infty ,g)$ is indeed a point of $\mathcal{M}_\omega^{fr,\infty}$, then $\iota'_\omega ((F_\omega^\infty,g)) = ((z_v)_{v\in\Upsilon_0^1},g)$. Thus, it is enough to prove that $(F_\omega^\infty,g)\in\mathcal{M}_\omega^{fr,\infty}$ i.e.
\begin{enumerate}[label = (\roman*)]
\item All maps of $F_\omega^\infty$ are inclusion maps
\item $\forall v\in \operatorname{Vert}(Q_\omega^\infty)$, $\operatorname{dim}_\mathbb{C}(F_\omega^\infty(v)) = \operatorname{height}(v)$
\item $\forall v\in \operatorname{Vert}(Q_\omega^\infty)$ and $n\in \mathbb{Z}$, $F_\omega^\infty(v+n) = F_\omega^\infty(v)$
\item $\forall c\in \operatorname{Cross}(\omega^\infty)$, $0\rightarrow F_\omega^\infty(S_c)\rightarrow F_\omega^\infty(E_c)\oplus F_\omega^\infty(W_c)\rightarrow F_\omega^\infty(N_c)\rightarrow 0$ are short exact sequences
\end{enumerate}
We get (\Rn{1}) immediately from the definition of $F_\omega^\infty$.\\
To prove (\Rn{2}), note that $\operatorname{dim}_\mathbb{C}(\sum_{v\in I_1(u)} g^{\sigma(v)}\cdot z_{v-\sigma(v)})=\operatorname{height}(u)$ because $((z_v)_{v\in \Upsilon_0^1},g)\in X_\omega$. Therefore, $\operatorname{dim}_\mathbb{C}(F_\omega^\infty(u)) = \operatorname{dim}_\mathbb{C}(\sum_{v\in I_1(u)} g^{\sigma(v)}\cdot z_{v-\sigma(v)})=\operatorname{height}(u)$.\\
To prove (\Rn{3}), note that $\forall u\in \operatorname{Vert}(Q_\omega^\infty)$ and $n\in \mathbb{Z}$, $I_1(u+n) = I_1(u) + n$. Therefore, 
\begin{align*}
	F_\omega^\infty(u+n) &= \sum_{v \in I_1(u+n)} (g^{\sigma(v)}\cdot z_{v - \sigma(v)})\\
	&= \sum_{v \in I_1(u)+n} (g^{\sigma(v)}\cdot z_{v - \sigma(v)})\\
	&= \sum_{v-n \in I_1(u)} (g^{\sigma(v)}\cdot z_{v - \sigma(v)})\\
	&= \sum_{v' \in I_1(u)} (g^{\sigma(v'+n)}\cdot z_{(v'+n) - \sigma(v'+n)})\\
	&= g^n \cdot (\sum_{v' \in I_1(u)} (g^{\sigma(v')}\cdot z_{v' - \sigma(v')}))\\
	&= g^n \cdot F_\omega^\infty(u)
\end{align*}
Finally, let's prove (\Rn{4}). Let $c\in \operatorname{Cross}(\omega^\infty)$, then
\begin{align*}
&\operatorname{height}(S_c) + 2 = \operatorname{height}(E_c)+1 = \operatorname{height}(W_c)+1 = \operatorname{height}(N_c)\\
\Rightarrow~& \operatorname{dim}_\mathbb{C}(F_\omega^\infty(S_c)) + 2 = \operatorname{dim}_\mathbb{C}(F_\omega^\infty(E_c))+1 = \operatorname{dim}_\mathbb{C}(F_\omega^\infty(W_c))+1 = \operatorname{dim}_\mathbb{C}(F_\omega^\infty(N_c))
\end{align*}
Since there are arrows from $S_c$ to $E_c$, $W_c$ and from $E_c$, $W_c$ to $N_c$, $I_1(S_c)\subset I_1(E_c),I_1(W_c)\subset I_1(N_c)$. Therefore, $F_\omega^\infty(S_c)\subset F_\omega^\infty(E_c)\cap F_\omega^\infty(W_c)$ and $F_\omega^\infty(E_c)\cup F_\omega^\infty(W_c)\subset F_\omega^\infty(N_c)$. By the condition that 
\begin{align*}
&\operatorname{dim}_\mathbb{C}(\sum_{v\in I_1(E_c)\cup I_1(W_c)}g^{\sigma(v)}\cdot z_{v-\sigma(v)}) = \operatorname{height}(N_c)\\
\Rightarrow~& \operatorname{dim}_\mathbb{C}(F_\omega^\infty(E_c) + F_\omega^\infty(W_c))=\operatorname{height}(N_c)=\operatorname{dim}_\mathbb{C}(F_\omega^\infty(N_c))\\
\Rightarrow~& F_\omega^\infty(E_c) + F_\omega^\infty(W_c) = F_\omega^\infty(N_c)
\end{align*}
we have surjection part of the short exact sequence i.e. $F_\omega^\infty(E_c)\oplus F_\omega^\infty(W_c)\rightarrow F_\omega^\infty(N_c)\rightarrow 0$ is exact. Thus we have a short exact sequence
\[
0\rightarrow F_\omega^\infty(E_c)\cap F_\omega^\infty(W_c) \rightarrow F_\omega^\infty(E_c)\oplus F_\omega^\infty(W_c)\rightarrow F_\omega^\infty(N_c)\rightarrow 0
\] 
I claim that $F_\omega^\infty(E_c)\cap F_\omega^\infty(W_c) = F_\omega^\infty (S_c)$. Since $F_\omega^\infty(E_c)$, $F_\omega^\infty(W_c)$,and $F_\omega^\infty(S_c)$ are codimension $1$, $1$,and $2$ inside of $F_\omega^\infty(N_c)$ respectively, it is enough to show that $F_\omega^\infty(E_c)$ and $F_\omega^\infty(W_c)$ intersect transversely in $F_\omega^\infty(N_c)$ i.e. $F_\omega^\infty(E_c) \neq F_\omega^\infty(W_c)$. This is true because otherwise
\begin{align*}
\operatorname{height}(N_c)~&=~\operatorname{dim}_\mathbb{C}(F_\omega^\infty(N_c))\\
&=~\operatorname{dim}_\mathbb{C}(F_\omega^\infty(E_c)+F_\omega^\infty(W_c))\\
&=~\operatorname{dim}_\mathbb{C}(F_\omega^\infty(E_c))\\
&=~\operatorname{height}(E_c) 
\end{align*}
which is a contradiction.\\
Therefore, $(F_\omega^\infty, g)$ satisfy all the condition (\Rn{1})-(\Rn{4}) i.e. $(F_\omega^\infty, g)\in \mathcal{M}_\omega^{fr,\infty}$.
\end{proof}


\section{Bivalent braid words}                          
\begin{definition}
	 A braid word $\omega$ is bivalent if it is multivalent and the valencies of all the vertices of height $1,\dots,n-1$ in $Q_\omega^\infty$ are equal to $2$ where $n$ is the number of strands of $\omega$. 
\end{definition}
\begin{lemma}\label{quoted6}
Let $\omega$ be a braid word and $v\in \operatorname{Vert}(Q_\omega$($v\in \operatorname{Vert}(Q_\omega^\infty)$ resp.), then there is a path from $v$ to the unique height $n$ vertex $U$. The path passes through a vertex of height $n-1$.
\end{lemma}
\begin{proof}
If we prove the statement for $Q^o_\omega$, then the proof for the $Q_\omega^\infty$ case follows immediately. Let's prove the claim by induction on the length of the braid word $\omega$. If $\operatorname{length}(\omega) = 0$ i.e. trivial braid word, there is a path from $D$ to $U$ and along the way it passes through all the points. Therefore, restricting this path to start from the point that we are interested in gives the desired path. Now let's assume that the statement holds for all braid words of length less than $k$. Let $\omega = s_{i_1}\cdots s_{i_k}$, then define $\omega' = s_{i_1}\cdots s_{i_{k-1}}$. We get that $Q^o_{\omega'}$ is the subquiver of $Q^o_\omega$ where $\operatorname{Vert}(Q^o_\omega) - \operatorname{Vert}(Q^o_{\omega'}) = \{v\}$ is a singleton where $v$ is the east of the crossing added by $s_{i_k}$, say $c$. Note that $v=E_c$ has an arrow to $N_c$ and $N_c$ has a path to $U$ by the induction hypothesis because $N_c \in \operatorname{Vert}(Q^o_{\omega'})$. Therefore, we can extend the path from $N_c$ to $U$ to start from $E_c$ i.e. $E_c\rightarrow (N_c\rightarrow \cdots \rightarrow U)$.
\end{proof}

\begin{lemma}\label{quoted4}
Let $\omega$ be a bivalent braid word and $u\in \operatorname{Vert}(Q_\omega^\infty)$ of $\operatorname{height}(u)<n$, then $|I_1(u)|=\operatorname{height}(u)$. In particular, if $c\in \operatorname{Cross}(\omega^\infty)$, then $I_1(S_c)=\operatorname{height}(S_c)$, $I_1(E_c)=\operatorname{height}(E_c)$, and $I_1(W_c)=\operatorname{height}(W_c)$.
\end{lemma}
\begin{proof}
We prove the statement by induction on the height of u. If $\operatorname{height}(u) = 1$, then $I_1(u) = \{u \} \Rightarrow |I_1(u)| = 1 = \operatorname{height}(u)$ holds. Now suppose the statement holds for vertices of heights less than $h$ and $\operatorname{height}(u) =h$ where $h<n$. Since $\omega$ is bivalent, we have exactly two vertices of height $h-1$ and a crossing $c$, where those two vertices are the east and west of the crossing $c$. By the induction hypothesis, $|I_1(E_c)| = |I_1(W_c)| = h-1$, $|I_1(S_c)| = h-2$ and $I_1(S_c)\subset I_1(E_c),I_1(W_c)$. Let
\begin{align*}
	I_1(E_c) - I_1(S_c) &= \{v_1\} \\
	I_1(W_c) - I_1(S_c) &= \{v_2\}
\end{align*}
Since $\omega$ is bivalent, 
\begin{align*}
	I_1(N_c) &= I_1(E_c) \cup I_1(W_c) \\
			 &= \{v_1, v_2\} \cup I_1(S_c)
\end{align*}
If $v_1 \neq v_2\Leftrightarrow I_1(E_c) \neq I_1(W_c)$, then $|I_1(N_c)| = 2 + (h-2) = h$. Therefore, it is enough to prove that $\forall c\in \operatorname{Cross}(\omega^\infty)$, $I_1(E_c) \neq I_1(W_c)$. This follows from \ref{quoted2} below.
\end{proof}

\begin{lemma}\label{quoted2}
Let $\omega$ be a bivalent braid word and $u,v$ be distinct vertices of $Q_\omega^\infty$ of the same height, then $I_1(u) \neq I_1(v)$. Note that the height of $u,v$ cannot be $n$ because there is only one vertex of height $n$, say $U$.
\end{lemma}
\begin{proof}
We prove the claim by the induction on the height of $u$ and $v$. If $\operatorname{height}(u) = \operatorname{height}(v) = 1$, then the claim holds because $I_1(u) = \{u\}$ and $I_1(v) = \{v\}$. \\
Now suppose the claim holds for vertices of heights less than $h$. Then there are exactly two height $h-1$ vertices for each, say $\{u_1,u_2\}$, $\{v_1,v_2\}$, that have arrows to $u$ and $v$. Note that there are crossings $c,c'$ such that $\{u_1,u_2\}=\{E_c,W_c\}$, $\{v_1,v_2\}=\{E_{c'},W_{c'}\}$. $\{u_1,u_2\}\neq\{v_1,v_2\}$, otherwise $c=c'\Rightarrow u=N_c=N_{c'}=v$ which is a contradiction. Therefore, by the induction hypothesis, $I_1(u) = I_1(u_1)\cup I_1(u_2)\neq I_1(v_1)\cup I_1(v_2) = I_1(v)$.
\end{proof}

\begin{lemma}\label{quoted5}
 Let $\omega$ be a bivalent braid word and $c\in \operatorname{Cross}(\omega^\infty)$, then $I_1(E_c)\cap I_1(W_c) = I_1(S_c)$ and $|I_1(E_c)\cup I_1(W_c)| = \operatorname{height}(N_c)$.
\end{lemma}
\begin{proof}
$I_1(S_c)\subset I_1(E_c)\cap I_1(W_c)$ because there are arrows from $S_c$ to $E_c,W_c$. Since $E_c\neq W_c$, by Lemma 35, $I_1(E_c)\neq I_1(W_c)$. Therefore, 
\begin{align*}
&~|I_1(E_c)\cap I_1(W_c)|\leq |I_1(E_c)|-1 = |I_1(S_c)| \\
\Rightarrow &~I_1(E_c)\cap I_1(W_c) = I_1(S_c)
\end{align*}
As a consequence, 
\begin{align*}
|I_1(E_c)\cup I_1(W_c)|&= |I_1(E_c)|+|I_1(W_c)|-|I_1(E_c)\cap I_1(W_c)|\\
&= |I_1(E_c)|+|I_1(W_c)|-|I_1(S_c)|\\
&= (h-1)+(h-1)-(h-2)=h\\
&=|I_1(N_c)|\\
&=\operatorname{height}(N_c)
\end{align*}
\end{proof}

\begin{definition}
Suppose $L_1,L_2,\cdots , L_k \subset \mathbb{C}^n$ are lines($1$ dimensional subspaces). We say $\{L_1,L_2,\cdots , L_k\}$ are linearly independent if and only if for nonzero $v_i \in L_i$, $\{v_1,v_2,\cdots , v_k\}$ are linearly independent. Note that the definition does not depend on the choice of $v_i$'s.
\end{definition}

\begin{theorem}
	Let $\omega$ be a bivalent braid word, then $\iota_\omega$ is an open embedding whose image is 
\begin{align*}
X'_\omega =\{ &((z_v)_{v\in\Upsilon_0^1},g)\in \prod_{v\in \Upsilon_0^1} \mathbb{P}^{n-1} \times \operatorname{GL}_n(\mathbb{C})~|\\
&\forall c\in \operatorname{Cross}(\omega)\subset \operatorname{Cross}(\omega^\infty)\text{ with } \operatorname{height}(N_c)=n,\\
& \operatorname{dim}_\mathbb{C}(\sum_{v\in I_1(E_c)\cup I_1(W_c)}g^{\sigma(v)}\cdot z_{v-\sigma(v)}) = n\}
\end{align*}
\end{theorem}
\begin{proof}
Since bivalent braids are multivalent, by \ref{quoted3}, the image of $\iota_\omega$ is given by 
\begin{align*}
X_\omega~=~\{ &((z_v)_{v\in \Upsilon_0^1},g)\in \prod_{v\in\Upsilon_0^1}\mathbb{P}_v^{n-1}\times \operatorname{GL}_n(\mathbb{C})~|\\
& \forall u\in \operatorname{Vert}(Q_\omega^\infty),~\operatorname{dim}_\mathbb{C}(\sum_{v\in I_1(u)} g^{\sigma(v)}\cdot z_{v-\sigma(v)}) = \operatorname{height}(u),\\
& \forall c\in \operatorname{Cross}(\omega^\infty),~\operatorname{dim}_\mathbb{C}(\sum_{v\in I_1(E_c)\cup I_1(W_c)} g^{\sigma(v)}\cdot z_{v-\sigma(v)}) = \operatorname{height}(N_c)\}
\end{align*}
I claim that $X_\omega = X'_\omega$. Since the condition
\[
\forall c\in \operatorname{Cross}(\omega)\subset \operatorname{Cross}(\omega^\infty)\text{ with } \operatorname{height}(N_c)=n, ~ \operatorname{dim}_\mathbb{C}(\sum_{v\in I_1(E_c)\cup I_1(W_c)}g^{\sigma(v)}\cdot z_{v-\sigma(v)}) = n
\]
defining $X'_\omega$ is subsumed in the condition
\[
\forall c\in \operatorname{Cross}(\omega^\infty),~\operatorname{dim}_\mathbb{C}(\sum_{v\in I_1(E_c)\cup I_1(W_c)} g^{\sigma(v)}\cdot z_{v-\sigma(v)}) = \operatorname{height}(N_c)
\]
defining $X_\omega$, $X_\omega \subset X'_\omega$.\\
Now let's prove $X'_\omega \subset X_\omega$ i.e. suppose $((z_v)_{v\in \Upsilon_0^1},g)$ satisfy
\begin{enumerate}[label = (\roman*)]
\item $\forall c\in \operatorname{Cross}(\omega)\subset \operatorname{Cross}(\omega^\infty)\text{ with } \operatorname{height}(N_c)=n, ~ \operatorname{dim}_\mathbb{C}(\sum_{v\in I_1(E_c)\cup I_1(W_c)}g^{\sigma(v)}\cdot z_{v-\sigma(v)}) = n$\\
\\
then it also satisfy\\
\item $\forall u\in \operatorname{Vert}(Q_\omega^\infty),~\operatorname{dim}_\mathbb{C}(\sum_{v\in I_1(u)} g^{\sigma(v)}\cdot z_{v-\sigma(v)}) = \operatorname{height}(u)$
\item $\forall c\in \operatorname{Cross}(\omega^\infty),~\operatorname{dim}_\mathbb{C}(\sum_{v\in I_1(E_c)\cup I_1(W_c)} g^{\sigma(v)}\cdot z_{v-\sigma(v)}) = \operatorname{height}(N_c)$
\end{enumerate}
Let $u\in \operatorname{Vert}(Q_\omega^\infty)$ and $c\in \operatorname{Cross}(\omega^\infty)$. First let's assume $\operatorname{height}(u)<n$. Since $\omega$ is bivalent, by \ref{quoted4}, $|I_1(u)|=\operatorname{height}(u)$ and by \ref{quoted4} and \ref{quoted5}, $|I_1(E_c)\cup I_1(W_c)| = |I_1(N_c)| = \operatorname{height}(N_c)$. Therefore, the condition (\Rn{1}) is equivalent to saying that 
\[
\forall c\in \operatorname{Cross}(\omega) \text{ with } \operatorname{height}(N_c) = n, ~\{g^{\sigma(v)}\cdot z_{v-\sigma(v)}\}_{v\in I_1(E_c)\cup I_1(W_c)}
\text{ are linearly independent}\] 
Likewise, the conditions (\Rn{2}),(\Rn{3}) are equivalent to saying that 
\begin{align*}
&\forall u\in \operatorname{Vert}(Q_\omega^\infty), ~\{g^{\sigma(v)}\cdot z_{v-\sigma(v)}\}_{v\in I_1(u)} \text{ are linearly independent} \\
&\forall c\in \operatorname{Cross}(\omega^\infty), ~\{g^{\sigma(v)}\cdot z_{v-\sigma(v)}\}_{v\in I_1(E_c)\cup I_1(W_c)}\text{ are linearly independent}
\end{align*}
Let's prove (\Rn{1}) implies (\Rn{2}) and (\Rn{3}) using these paraphrased statements. Assume (\Rn{1}) holds for $((z_v)_{v\in \Upsilon_0^1},g)$. Suppose for some $u\in \operatorname{Vert}(Q_\omega^\infty)$, $\{g^{\sigma(v)}\cdot z_{v-\sigma(v)}\}_{v\in I_1(u)}$ are linearly dependent. Let $u' = u-\sigma(u)$, then $I_1(u')=I_1(u)-\sigma(u)$. Therefore, 
\begin{align*}
\{g^{\sigma(v)}\cdot z_{v-\sigma(v)}\}_{v\in I_1(u')}~&=~\{g^{\sigma(v)}\cdot z_{v-\sigma(v)}\}_{v+\sigma(u)\in I_1(u)}\\
&=~\{g^{\sigma(v')-\sigma(u)}\cdot z_{v'-\sigma(v')}\}_{v'\in I_1(u)} \\
&=~g^{-\sigma(u)}\cdot\{g^{\sigma(v)}\cdot z_{v-\sigma(v)}\}_{v\in I_1(u)}
\end{align*}
are also linearly dependent. Therefore, without loss of generality, we can assume $\sigma(u) = 0$ i.e. $u\in \operatorname{Vert}(Q_\omega)\subset \operatorname{Vert}(Q_\omega^\infty)$. By \ref{quoted6}, there is a path from $u$ to a height $n-1$ vertex $p$. Thus, $I_1(u)\subset I_1(p)$. Since $\omega$ is bivalent, for some $c\in \operatorname{Cross}(\omega)$, $p=E_c\text{ or }W_c$. By the condition (\Rn{1}), $\{g^{\sigma(v)}\cdot z_{v-\sigma(v)}\}_{v\in I_1(E_c)\cup I_1(W_c)}$ are linearly independent, thus its subset $\{g^{\sigma(v)}\cdot z_{v-\sigma(v)}\}_{v\in I_1(u)}$ are linearly independent as well. Therefore, $((z_v)_{v\in\Upsilon_0^1},g)$ satisfy (\Rn{2}).\\
Now let's show that $((z_v)_{v\in\Upsilon_0^1},g)$ satisfy (\Rn{3}). Suppose $c\in \operatorname{Cross}(\omega^\infty)$. If $\operatorname{height}(N_c)=n$, then the condition (\Rn{1}) is equal to the condition (\Rn{3}), there is nothing to prove. Suppose $\operatorname{height}(N_c)<n$, then $I_1(E_c)\cup I_1(W_c)=I_1(N_c)$. Thus, the condition (\Rn{3}) translates to $\{g^{\sigma(v)}\cdot z_{v-\sigma(v)}\}_{v\in I_1(N_c)}$ are linearly independent, which follows from (\Rn{2}) that we already proved.\\
Now let's show that (\Rn{3}) holds when $\operatorname{height}(u)=n$ i.e. $u$ is the unique height $n$ point $U$. Since $\omega$ is bivalent, there is a crossing $c$ such that $u=N_c$. Therefore,
\[
\operatorname{dim}_\mathbb{C}(\sum_{v\in I_1(E_c)\cup I_1(W_c)} g^{\sigma(v)}\cdot z_{v-\sigma(v)}) = \operatorname{height}(N_c) = n
\]
Since $I_1(E_c)\cup I_1(W_c)\subset I_1(N_c)=I_1(u)$, 
\[
\operatorname{dim}_\mathbb{C}(\sum_{v\in I_1(u)} g^{\sigma(v)}\cdot z_{v-\sigma(v)}) = n=\operatorname{height}(u)
\]
\end{proof}

\begin{theorem}
(Combinatorial characterization) A braid word is multivalent(bivalent resp.) if and only if in between $s_i$'s there is at least one(exactly one resp.) $s_{i-1}$ upto cyclic shifts for $2\leq i<n$.
\end{theorem}
\begin{proof}
The statement follows from the fact that 
\begin{itemize}
\item there is a one to one correspondence between the set of region of height $k$ and $s_k$'s in the braid word for $1\leq k<n$.

\item the valency of a region corresponding to a certain $s_k$ is equal to the number of $s_{k-1}$ in between that certain $s_k$ and the next $s_k$.
\end{itemize}
\end{proof}

\begin{definition}
$\mathcal{M}_1(S^1_x \times \R_z,\Lambda_\omega,\{\sigma_{z\ll 0}\};\C)$ is called \emph{Generalized Sibuya space} if $\omega$ is a bivalent braid.
\end{definition}

   
\section{Examples : generalized Sibuya spaces}
By nequation, I mean a formula that expresses the non-equality of two expressions.\\
Whenever I denote capital $X$ with subscript i.e. $X_j$, I mean an element of some projective space.\\
Lower case $x$'s with subscripts are used to denote the homogeneous coordinates of $X_j$'s.\\
For example $X_j=[x_{1,j} : \cdots : x_{n,j}]$.\\
I will also denote
\[
	(X_1,\cdots,X_{n-1})=([x_{1,1}:\cdots:x_{n,1}],[x_{1,2}:				\cdots:x_{n,2}],
	\cdots,
	[x_{1,n-1}:\cdots:x_{n,n-1}])
\] as 
\[
	X
	=
	\begin{bmatrix}
		x_{1,1}&\cdots&x_{1,n-1}\\
		\vdots&\ddots&\vdots\\
		x_{n,1}&\cdots&x_{n,n-1}
	\end{bmatrix}
\]
We will use
\[
		\left( \begin{array}{c|c|c|c}
			& & & \\
			X_1&X_2&\cdots& X_{n}\\
			& & &
		\end{array}\right)
		=
		\left( \begin{array}{ccc}
			x_{1,1}&\cdots & x_{1,n} \\
			\vdots&\ddots & \vdots \\
			x_{k+1,1}&\cdots& x_{k+1,n}
		\end{array}\right)
\]
to denote the matrix whose entries are $x_{i,j}$ which are sections of the line bundle $O_{\mathbb{P}^k}(1)$.\\
Thereby
\[
det
		\left(\begin{array}{c|c|c|c}
			& & & \\
			X_1&X_2&\cdots& X_{n}\\
			& & &
		\end{array}\right)
		=
		det
		\left( \begin{array}{ccc}
			x_{1,1}&\cdots & x_{1,n} \\
			\vdots&\ddots & \vdots \\
			x_{k+1,1}&\cdots& x_{k+1,n}
		\end{array}\right)
\]
is a section of the line bundle $\underbrace{O_{\mathbb{P}^k}(1)\boxtimes\cdots \boxtimes O_{\mathbb{P}^k}(1)}_{n-copies}$ on $\underbrace{\mathbb{P}^k\times\cdots \times \mathbb{P}^k}_{n-copies}$.
\begin{example}
	($\bf{Empty~Set}$) In this section we compute 2 kinds of moduli spaces that are empty.\\
	In the first case, we consider the moduli space of braids represented by a bivalent braid word that the number of  $s_1$ in the expression is less than the~number~of~strands with unipotent monodromy $u=I$\\
The framed moduli space is 
		\begin{align*}
		\mathcal{M}^{fr}=
		\{(X_1,\cdots,X_{k}) \in 										\underbrace{\mathbb{P}^{n-1}\times \cdots \times \mathbb{P}			^{n-1}}_{k-copies}~|~		
		& \operatorname{det}
		\left( \begin{array}{c|c|c|c|c|c|c}
			& & & & & &\\
			X_1&X_2&\cdots&X_{k}&X_1&\cdots& X_{n-k}\\
			& & & & & &
		\end{array}\right)
		\neq 0, \\
		& \operatorname{det}
		\left( \begin{array}{c|c|c|c|c|c}
			& & & & &\\
			X_2&\cdots&X_{k}&X_1&\cdots& X_{n-k+1}\\
			& & & & &
		\end{array}\right)
		\neq 0,
		\cdots,\\
		& \operatorname{det}
		\left( \begin{array}{c|c|c|c|c|c|c}
			& & & & & &\\
			X_k&X_1&\cdots&X_{k}&X_1&\cdots&X_{n-k-1}\\
			& & & & & &
		\end{array}\right)
		\neq 0
		\}
	\end{align*}
Take the first nequation and subtract $1^{st}$column from the $(k+1)^{th}$column we get
\[
	\operatorname{det}
		\left( \begin{array}{c|c|c|c|c|c|c|c}
		&   &   &      &      & &   &                \\
		X_1 &X_2&\cdots&X_{k} &0 &X_2&\cdots&X_{n-k}\\
		&   &   &      &      & &   &
		\end{array}\right)
		\neq 0
\]

which is never true. Therefore, the framed moduli  space $\mathcal{M}^{fr}=\emptyset$ and so is the moduli space.\\

In the second case, we consider the moduli space of braids represented by a bivalent braid word that the number of  $s_1$ in the expression is less than (the number of strands)$-1$ with unipotent monodromy 
\[u= 
\begin{pmatrix}
	1&0&\cdots&0&0\\
	0&1&\cdots&0&0\\
	\vdots&\vdots&\ddots&\vdots&\vdots\\
	0&0&\cdots&1&1\\
	0&0&\cdots&0&1
\end{pmatrix}
\] \\
The framed moduli space is 
		\begin{align*}
		\mathcal{M}^{fr}=
		\{(X_1,\cdots,X_{k}) \in 										\underbrace{\mathbb{P}^{n-1}\times \cdots \times \mathbb{P}			^{n-1}}_{k-copies}~|~		
		 \operatorname{det}
		\left( \begin{array}{c|c|c|c|c|c|c}
			& & & & & &\\
			X_1&X_2&\cdots&X_{k}&uX_1&\cdots& uX_{n-k}\\
			& & & & & &
		\end{array}\right)
		\neq 0&, \\
		\operatorname{det}
		\left( \begin{array}{c|c|c|c|c|c}
			& & & & &\\
			X_2&\cdots&X_{k}&uX_1&\cdots& uX_{n-k+1}\\
			& & & & &
		\end{array}\right)
		\neq 0,
		\cdots,&\\
		 \operatorname{det}
		\left( \begin{array}{c|c|c|c|c|c|c}
			& & & & & &\\
			X_k&uX_1&\cdots&uX_{k}&u^2 X_1&\cdots& u^2 X_{n-k-1}\\
			& & & & & &
		\end{array}\right)
		\neq 0&
		\}
	\end{align*}
Take the first nequation and subtract $1^{st}$(resp. $2^{nd}$) column from the $(k+1)^{th}$(resp. $(k+2)^{th}$) column we get
\[
	\operatorname{det}
		\left( \begin{array}{c|c|c|c|c|c|c|c|c}
		&   &   &      &0      & 0     & &    &                \\
		&   &   &      &\vdots &\vdots & &    &            \\
		X_1 &X_2&\cdots&X_{k}  &0      &0&uX_3&\cdots&uX_{n-k}\\
		&   &   &      &x_{n,1}&x_{n,2}& &    &            \\
		&   &   &      &0      &0      & &    &
		\end{array}\right)
		\neq 0
\]

which is never true. Therefore, the framed moduli  space $\mathcal{M}^{fr}=\emptyset$ and so is the moduli space.
\end{example}

\begin{example}
($\bf{Points}$) In this section, we compute the moduli space of braids represented by the braid word \[(s_1s_2\cdots s_{n-1})^{n-1}\] and the unipotent monodromy 
\[u= 
\begin{pmatrix}
	1&0&\cdots&0&0\\
	0&1&\cdots&0&0\\
	\vdots&\vdots&\ddots&\vdots&\vdots\\
	0&0&\cdots&1&1\\
	0&0&\cdots&0&1
\end{pmatrix}
\] \\
i.e. a unipotent matrix given by the partition $(\underbrace{1,1,\cdots,1}_{n-1},2)$.
The framed moduli space associated to the above data is given as follows

%Experiment%
	\begin{align*}
		\mathcal{M}^{fr}=\{(X_1,X_2,\cdots,X_{n-1}) \in 						\underbrace{\mathbb{P}^{n-1}\times \cdots \times \mathbb{P}^{n-1}}_{(n-1)-copies}~|~		
		\operatorname{det}
		\left( \begin{array}{c|c|c|c|c}
			& & & &\\
			X_1&X_2&\cdots&X_{n-1}&uX_1\\
			& & & &
		\end{array}\right)
		\neq 0, \\
		\operatorname{det}
		\left( \begin{array}{c|c|c|c|c}
			& & & &\\
			X_2&\cdots & X_{n} & uX_1 & uX_2\\
			& & & &
		\end{array}\right)
		\neq 0,\cdots , \operatorname{det}
		\left( \begin{array}{c|c|c|c|c}
			& & & &\\
			X_{n-1}&uX_1&\cdots&uX_{n-2}&uX_{n-1}\\
			& & & &
		\end{array}\right)
		\neq 0\}
	\end{align*}
%%%%%%%%%%%%%%%%%
Using the elementary column operation of subtracting the first column from the last column we get
	\begin{align*}
%start set notation%
		\mathcal{M}^{fr}=\{(X_1,X_2,\cdots,X_{n-1}) \in 						\underbrace{\mathbb{P}^{n-1}\times \cdots \times					\mathbb{P}^{n-1}}_{(n-1)-copies}~|~		
%1st determinant%	
		\operatorname{det}
		\left( \begin{array}{c|c|c|c|c}
			& & & &0\\
			& & & &\vdots\\
			X_1&X_2&\cdots&X_{n-1}&0\\
			& & & &x_{n,1}\\
			& & & &0\\
		\end{array}\right)
		\neq 0,\\
%2nd determinant%		
		\operatorname{det}
		\left(\begin{array}{c|c|c|c|c}
			& & & &0\\
			& & & &\vdots\\
			X_2&\cdots&X_{n-1}&uX_{1}&0\\
			& & & & x_{n,2}\\
			& & & &0\\
		\end{array}\right)
		\neq 0,
%dots
		\cdots,
%last determinant%		
		\operatorname{det}
		\left(\begin{array}{c|c|c|c|c}
			& & & &0\\
			& & & &\vdots\\
			X_{n-1}&uX_1&\cdots&uX_{n-2}&0\\
			& & & &x_{n,n-1} \\
			& & & &0\\
		\end{array}\right)
		\neq 0\} 
		\end{align*}\\
Applying the cofactor expansion formula with respect to the last column we get\\
	\begin{align*}
		\mathcal{M}^{fr}=\{(X_1,X_2,\cdots,X_{n-1}) \in 						\underbrace{\mathbb{P}^{n-1}\times \cdots \times \mathbb{P}^{n-1}}_{n-copies}~|~
		x_{n,1}\cdot \operatorname{det}
		\left( \begin{array}{ccc}
			x_{1,1}&\cdots &x_{1,n-1}\\
			\vdots &\ddots & \vdots \\
			x_{n-2,1}&\cdots &x_{n-2,n-1}\\
			x_{n,1}&\cdots & x_{n,n}
		\end{array}\right)
		\neq 0,\\
		x_{n,2}\cdot \operatorname{det}
		\left( \begin{array}{ccc}
			x_{1,1}&\cdots &x_{1,n-1}\\
			\vdots &\ddots & \vdots \\
			x_{n-2,1}&\cdots &x_{n-2,n-1}\\
			x_{n,1}&\cdots & x_{n,n}
		\end{array}\right)
		\neq 0,
		\cdots,
		x_{n,n-1}\cdot \operatorname{det}
		\left( \begin{array}{ccc}
			x_{1,1}&\cdots &x_{1,n-1}\\
			\vdots &\ddots & \vdots \\
			x_{n-2,1}&\cdots &x_{n-2,n-1}\\
			x_{n,1}&\cdots & x_{n,n}
		\end{array}\right)
		\neq 0\}
		\end{align*}
		\begin{align*}
=\{(X_1,X_2,\cdots,X_{n-1}) \in\underbrace{\mathbb{P}^{n-1}\times \cdots \times \mathbb{P}^{n-1}}_{n-copies}~|~
		x_{n,i}\neq 0\text{ for } i=1,\cdots, n-1\\
		,
		\operatorname{det}
		\left( \begin{array}{ccc}
			x_{1,1}&\cdots &x_{1,n-1}\\
			\vdots &\ddots & \vdots \\
			x_{n-2,1}&\cdots &x_{n-2,n-1}\\
			x_{n,1}&\cdots & x_{n,n}
		\end{array}\right)
		\neq 0\}
	\end{align*}
To get the moduli space out of the framed moduli space above, we have to quotient it out by the centralizer subgroup of $u$  
, that is,
	\begin{align*}
		\operatorname{C}_{\operatorname{GL}_n(\mathbb{C})}(u) = \{ 
		&\left(\begin{array}{ccc|cc}
			c_{1,1}&\cdots&c_{1,n-2}&0&c_{1,n}\\
			\vdots&\ddots&\vdots&0&c_{1,n}\\
			c_{n-2,1}&\cdots&c_{n-2,n-2}&0&c_{n-2,n}\\
			\hline
			0&\cdots&0&0&c_{n-1,n}\\
			c_{n,1}&\cdots&c_{n,n-2}&c_{n,n-1}&c_{n,n}
		\end{array}\right)\in \operatorname{GL}_{n}(\mathbb{C})|\\
		&\operatorname{det}
		\left(\begin{array}{ccc}
			c_{1,1}&\cdots&c_{1,n-2}\\
			\vdots&\ddots&\vdots\\
			c_{n-2,1}&\cdots&c_{n-2,n-2}
		\end{array}\right),c_{n,n-1},c_{n-1,n}\neq 0\}
	\end{align*}\\
%%
%%
It acts diagonally on $\underbrace{\mathbb{P}^{n-1}\times\cdots\mathbb{P}^{n-1}}_{(n-1)-copies}$ where the action on each coordinate is given by left multiplication. \\
To simplify the notation, I will denote 
\[
	(X_1,\cdots,X_{n-1})=([x_{1,1}:\cdots:x_{n,1}],[x_{1,2}:				\cdots:x_{n,2}],
	\cdots,
	[x_{1,n-1}:\cdots:x_{n,n-1}])\] as 
\[
	X
	=
	\begin{bmatrix}
		x_{1,1}&\cdots&x_{1,n-1}\\
		\vdots&\ddots&\vdots\\
		x_{n,1}&\cdots&x_{n,n-1}
	\end{bmatrix}
\]
%%
%%
I claim that for any $(X_1,\cdots,X_{n-1}) \in \underbrace{\mathbb{P}^{n-1}\times\cdots\mathbb{P}^{n-1}}_{(n-1)-copies}$, there exists $A\in \operatorname{C}_{\operatorname{GL}_n(\mathbb{C})}(u)$ such that 
\[
	A\cdot X=
	\left[
	\begin{array}{c|c}
		I
		&
		\begin{array}{c}
			0  \\
			\vdots \\
			0
		\end{array}\\
		\hline\\
		\begin{array}{ccc}
			1&\cdots&1
		\end{array}
		&
		1				 
	\end{array}	
	\right]
\]
%%
Let
%% 
\[
	A_1=
	\left(\begin{array}{c|c}
		\left(\begin{array}{ccc}
			x_{1,1}&\cdots&x_{1,n-2}\\
			\vdots&\ddots&\vdots\\
			x_{n-2,1}&\cdots&x_{n-2,n-2}
		\end{array}\right)^{-1}
		&
		\begin{array}{cc}
			0&0\\
			\vdots&\vdots\\
			0&0
		\end{array}		
		\\
		\hline\\
		\begin{array}{ccc}
			0&\cdots&0\\
			a_1&\cdots&a_{n-2}
		\end{array}
		&
		\begin{array}{cc}
			0&1\\
			a_{n-1}&a_{n}
		\end{array}
	\end{array}\right)
\]
with $a_{n-1}\neq 0$ such that 
\[
\sum_{i=1}^{n} a_{i}\cdot (x_{i,1},\cdots,x_{i,n})=(1,\cdots,1)
\]
We can always find such because we know that
\[
		\operatorname{det}
		\left( \begin{array}{ccc}
			x_{1,1}&\cdots &x_{1,n-1}\\
			\vdots &\ddots & \vdots \\
			x_{n-2,1}&\cdots &x_{n-2,n-1}\\
			x_{n,1}&\cdots & x_{n,n}
		\end{array}\right)
		\neq 0
\]

Then we get 
\[
	A_1\cdot X=
	\left[
	\begin{array}{c|c}
		I
		&
		\begin{array}{c}
			b_1  \\
			\vdots \\
			b_{n-1}
		\end{array}\\
		\hline\\
		\begin{array}{ccc}
			x_{n,1} & \cdots & x_{n,n-2} \\
			1      & \cdots  & 1
		\end{array}
		&
		\begin{array}{c}		
			x_{n,n-1} \\
			1		
		\end{array}			 
		\end{array}
		\right]
\]
Since we know that $x_{n,i}\neq 0$ for $i=1,\cdots,n-1$, without loss of generality we put $x_{n,i}=1$ for $i=1,\cdots,n-1$
. Then we get
\[
	A_1\cdot X =
	\left[
	\begin{array}{c|c}
		I
		&
		\begin{array}{c}
			b_1  \\
			\vdots \\
			b_{n-1}
		\end{array}\\
		\hline\\
		\begin{array}{ccc}
			1 & \cdots & 1 \\
			1 & \cdots & 1
		\end{array}
		&
		\begin{array}{c}		
			1 \\
			1		
		\end{array}			 
		\end{array}
		\right]
\]
such that $b_1+\cdots +b_{n-1}+1\neq 0$. Now let 
\[
	A_{2}=
	\left(
	\begin{array}{c|c}
		\begin{array}{ccc}
			a_{1,1}   & \cdots & a_{1,n-2} \\
			\vdots    & \ddots & \vdots \\
			a_{n-2,1} & \cdots & a_{n-2,n-2}
		\end{array}
		&
		\begin{array}{cc}
			0      & a_{1,n}  \\
			\vdots & \vdots \\
			0      & a_{n-2,n}
		\end{array}\\
		\hline\\
		\begin{array}{ccc}
			0 & \cdots & 0 \\
			0 & \cdots & 0
		\end{array}
		&
		I				 
	\end{array}\right)	
\]
such that 
\begin{align*}
&a_{i,1}\cdot (1,0,\cdots,0,b_1)+
	a_{i,2}\cdot (0,1,0,\cdots,0,b_2) + 
	\cdots + 
	a_{i,n-2}\cdot (0,\cdots,0,1,b_{n-2})+
	a_{i,n}\cdot (1,\cdots,1)\\
&= (0,\cdot,0,\overset{\substack{\text{i}^{th}\\\downarrow}}{1},0,\cdots,0)
\end{align*}
Again, we can find such because
\[
	\operatorname{det}
	\left[
	\begin{array}{c|c}
		I
		&
		\begin{array}{c}
			b_1  \\
			\vdots \\
			b_{n-1}
		\end{array}\\
		\hline\\
		\begin{array}{ccc}
			1 & \cdots & 1 
		\end{array}
		&
		\begin{array}{c}		
			1 		
		\end{array}			 
	\end{array}\right]
	\neq 0
\]
Then $A_{2}\cdot A_{1}$ is the desired $A$.
\end{example}

\begin{example}
($\bf{Regular~Unipotent~Fibers}$) Finite type integral schemes. Also, it has a stratification into rational varieties at most 1 stratum in each dimension. $(s_1)^3$, $(s_1 s_2)^2$\\
\\
$\bf{Example1}$\\
In this section, we compute the moduli space associated to the braid word $:(s_1s_2)^2$ with regular unipotent monodromy. The framed moduli space is given by
\[
	\mathcal{M}^{fr}=\{
	([x:y:z],[a:b:c])\in \mathbb{P}^2 \times \mathbb{P}^2~|~
	\operatorname{det}
	\begin{pmatrix}
	x&a&y\\
	y&b&z\\
	z&c&0\\
	\end{pmatrix}
	\neq 0,~
	\operatorname{det}
	\begin{pmatrix}
	x&a&b\\
	y&b&c\\
	z&c&0\\
	\end{pmatrix}
	\neq 0	 	
	\}
\]
The centralizer subgroup of the unipotent matrix
\[u=
	\begin{pmatrix}
		1&1&0\\
		0&1&1\\
		0&0&1
	\end{pmatrix}
\] 
is
\[
	C:=\operatorname{C}_{\operatorname{GL}_3(\mathbb{C})}(u)=
	\{
	\begin{pmatrix}
	\alpha&\beta&\gamma\\
	0&\alpha&\beta\\
	0&0&\alpha\\
	\end{pmatrix}
	\in \operatorname{GL}_3(\mathbb{C})~|~
	\alpha \neq 0
	\}
\]

We can cover $\mathcal{M}^{fr}$ with open subsets $U_1,U_2$ where 
\begin{align*}
	&U_{1}=\{
	([x:y:z],[a:b:c])\in \mathbb{P}^2 \times \mathbb{P}^2~|~
	\operatorname{det}
	\begin{pmatrix}
	x&a&y\\
	y&b&z\\
	z&c&0\\
	\end{pmatrix}
	\neq 0,~
	\operatorname{det}
	\begin{pmatrix}
	x&a&b\\
	y&b&c\\
	z&c&0\\
	\end{pmatrix}
	\neq 0,
	z\neq 0	 	
	\}\\
	&U_{2}=
	\{
	([x:y:z],[a:b:c])\in \mathbb{P}^2 \times \mathbb{P}^2~|~
	\operatorname{det}
	\begin{pmatrix}
	x&a&y\\
	y&b&z\\
	z&c&0\\
	\end{pmatrix}
	\neq 0,~
	\operatorname{det}
	\begin{pmatrix}
	x&a&b\\
	y&b&c\\
	z&c&0\\
	\end{pmatrix}
	\neq 0,
	c\neq 0	 	
	\}	
\end{align*}
Therefore, we have a pushout square
\begin{displaymath}
\begin{tikzcd}
  U_1\arrow[hookrightarrow]{r} & \mathcal{M}^{fr} \\
  U_1 \cap U_2 \arrow[hookrightarrow]{u} \arrow[r,hookrightarrow] & U_2 \arrow[hookrightarrow]{u} 
\end{tikzcd} 
\end{displaymath}\\
Quotienting out by the centralizer subgroup $C$, we get
\begin{displaymath}
\begin{tikzcd}
  \overline{U}_1:=C\backslash U_1\arrow[r,hookrightarrow] & \mathcal{M}=C\backslash\mathcal{M}^{fr} \\
  \overline{U_1 \cap U}_2:=C\backslash U_1 \cap U_2 \arrow[hookrightarrow]{u} \arrow[r,hookrightarrow] & \overline{U}_2:=C\backslash U_2 \arrow[hookrightarrow]{u} 
\end{tikzcd} 
\end{displaymath}
Also note that, for the action of $C$, the centralizer subgroup of any element of $\mathcal{M}^{fr}$ is the set of scalar multiplication matrices.\\
First let's simplify, $\overline{U}_1$. Suppose
\[
	\begin{bmatrix}
	x&a\\
	y&b\\
	z&c
	\end{bmatrix}
	\in U_1
\]
Then there exists a 
\[
	\begin{pmatrix}
		\alpha & \beta & \gamma \\
		0 & \alpha & \beta \\
		0 & 0 & \alpha	
	\end{pmatrix}
	\in
	C
\]
such that
\[
	\begin{pmatrix}
		\alpha & \beta & \gamma \\
		0 & \alpha & \beta \\
		0 & 0 & \alpha	
	\end{pmatrix}
	\cdot
	\begin{bmatrix}
		x\\
		y\\
		z
	\end{bmatrix}
	=
	\begin{bmatrix}
		0\\
		0\\
		1
	\end{bmatrix}	
\] 
That is 

	\begin{align*}
		&\alpha=1\\
		&\beta=-\frac{y}{z}\\
		&\gamma=\frac{y^2}{z^2}-\frac{x}{z}
	\end{align*}
This expression makes sense because $z\neq 0$ in $U_1$.
If we take an element of $U_1$ with 
\[
	\begin{bmatrix}
		x\\
		y\\
		z
	\end{bmatrix}
	=
	\begin{bmatrix}
		0\\
		0\\
		1
	\end{bmatrix}	
\]
as a representative from each $C$-orbit, we see that 
\begin{align*}
	\overline{U}_1
	&\cong
	\{
	([0:0:1],[a:b:c])\in \mathbb{P}^2 \times \mathbb{P}^2~|~
	\operatorname{det}
	\begin{pmatrix}
	0&a&0\\
	0&b&1\\
	1&c&0\\
	\end{pmatrix}
	\neq 0,~
	\operatorname{det}
	\begin{pmatrix}
	0&a&b\\
	0&b&c\\
	1&c&0\\
	\end{pmatrix}
	\neq 0,
	1\neq 0	 	
	\}\\
	&\cong
	\{
	[a:b:c]\in \mathbb{P}^2~|~
	a\neq 0,~
	\operatorname{det}
	\begin{pmatrix}
	a&b\\
	b&c
	\end{pmatrix}
	\neq 0	 	
	\}\\
	&\cong
	\{
	(b,c)\in \mathbb{A}^2~|~
	c-b^2\neq 0 	
	\}	
\end{align*}
Under this identification,
\[
	\overline{U_1\cap U}_2
	\cong
	\{
	(b,c)\in \mathbb{A}^2~|~
	c-b^2\neq 0,c\neq 0 	
	\}
\]
Now let's simplify, $ \overline{U}_2 $. Suppose

\[
	\begin{bmatrix}
	x&a\\
	y&b\\
	z&c
	\end{bmatrix}
	\in U_2
\]
Then there exists a 
\[
	\begin{pmatrix}
		\alpha & \beta & \gamma \\
		0 & \alpha & \beta \\
		0 & 0 & \alpha	
	\end{pmatrix}
	\in
	C
\]
such that
\[
	\begin{pmatrix}
		\alpha & \beta & \gamma \\
		0 & \alpha & \beta \\
		0 & 0 & \alpha	
	\end{pmatrix}
	\cdot
	\begin{bmatrix}
		a\\
		b\\
		c
	\end{bmatrix}
	=
	\begin{bmatrix}
		0\\
		0\\
		1
	\end{bmatrix}	
\] 
That is 

	\begin{align*}
		&\alpha=1\\
		&\beta=-\frac{b}{c}\\
		&\gamma=\frac{b^2}{c^2}-\frac{a}{c}
	\end{align*}
This expression makes sense because $c\neq 0$ in $U_1$.
If we take an element of $U_2$ with 
\[
	\begin{bmatrix}
		a\\
		b\\
		c
	\end{bmatrix}
	=
	\begin{bmatrix}
		0\\
		0\\
		1
	\end{bmatrix}	
\]
as a representative from each $C$-orbit, we see that 
\begin{align*}
	\overline{U}_2
	&\cong
	\{
	([x:y:z],[0:0:1])\in \mathbb{P}^2 \times \mathbb{P}^2~|~
	\operatorname{det}
	\begin{pmatrix}
	x&a&y\\
	y&b&z\\
	z&c&0\\
	\end{pmatrix}
	\neq 0,~
	\operatorname{det}
	\begin{pmatrix}
	x&a&b\\
	y&b&c\\
	z&c&0\\
	\end{pmatrix}
	\neq 0,
	1\neq 0	 	
	\}\\
	&\cong
	\{
	[x:y:z]\in \mathbb{P}^2~|~
	x\neq 0,~
	det	
	\begin{pmatrix}
	x&y\\
	y&z
	\end{pmatrix}
	\neq 0	 	
	\}\\
	&\cong
	\{
	(y,z)\in \mathbb{A}^2~|~
	z-y^2\neq 0 	
	\}	
\end{align*}

Under these identifications, the pushout square above becomes
\begin{displaymath}
\begin{tikzcd}
	\{
	(b,c)\in \mathbb{A}^2~|~
	c-b^2\neq 0 	
	\}
  \arrow[hookrightarrow]{r} & \mathcal{M} \\
	\{
	(b,c)\in \mathbb{A}^2~|~
	c-b^2\neq 0,c\neq 0 	
	\} 
  \arrow[hookrightarrow]{u} \arrow[hookrightarrow]{r}{f} 
  & \{
	(y,z)\in \mathbb{A}^2~|~
	z-y^2\neq 0 	
	\} 
  \arrow[hookrightarrow]{u} 
\end{tikzcd} 
\end{displaymath}\\
where
\begin{align*}
	f:\{(b,c)\in \mathbb{A}^2~|~c-b^2\neq 0,& c \neq 0\} 				\rightarrow 
	\{(y,z)\in \mathbb{A}^2~|~z-y^2\neq 0 \} \\
	(b,c)&\mapsto (-\frac{bc}{b^2-c},\frac{c^2}{b^2-c})
\end{align*}
Now define a variety $V$ to be 
\begin{align*}
	&V:=\{(Y,Z,W)\in \mathbb{A}^3 ~|~ Y^2W-ZW+YZ=0\}-\{(0,0,0)\}
\end{align*}

I claim that $V$ is isomorphic to $\mathcal{M}$. More precisely, I claim that 

\begin{displaymath}
\begin{tikzcd}
	\{
	(b,c)\in \mathbb{A}^2~|~
	c-b^2\neq 0 	
	\}
  \arrow[hookrightarrow]{r}{i} & V \\
	\{
	(b,c)\in \mathbb{A}^2~|~
	c-b^2\neq 0,c\neq 0 	
	\} 
  \arrow[hookrightarrow]{u}{g} \arrow[hookrightarrow]{r}{f} 
  & \{
	(y,z)\in \mathbb{A}^2~|~
	z-y^2\neq 0 	
	\} 
  \arrow[hookrightarrow]{u}{\iota}
\end{tikzcd} 
\end{displaymath}\\
is a pushout square where
\begin{align*}
	i:\{(b,c)\in \mathbb{A}^2~&|~c-b^2\neq 0\} 							\rightarrow 
	V \\
	(b,c)&\mapsto (-\frac{bc}{b^2-c},\frac{c^2}{b^2-c},b)
\end{align*}
\begin{align*}
	\iota:\{(y,z)\in \mathbb{A}^2~&|~z-y^2\neq 0 \} 						\rightarrow 
	V \\
	(y,z)&\mapsto (y,z,\frac{yz}{z-y^2})
\end{align*}

It is easy to check the square commutes. $f,g$ are inclusion maps by construction.
$\iota$ is also an inclusion map because we can recover $(y,z)$ from $\iota (y,z)$ by projecting onto the $1^{st}\& 2^{nd}$ coordinates. \\
For $i$, we can recover $b$ from $i(b,c)$ by projecting onto the $3^{rd}$ coordinate. We can recover $c$ from $i(b,c)$ by multiplying $2^{nd}\&3^{rd}$ coordinates and dividing with -(the $1^{st}$ coordinate). \\
Let's check that the images of $i$ and $\iota$ form an open cover of $V$. The images of $i$ and $\iota$ are
\[
	i(\{(b,c)\in \mathbb{A}^2~|~c-b^2\neq 0\})=\{(Y,Z,W)\in \mathbb{A}^3| Y^2W-ZW+YZ=0, YW+Z+W^2\neq 0\}
\]
\begin{align*}
	\iota (\{(y,z)\in \mathbb{A}^2~|~z-y^2\neq 0 \})
	&=
	\{(Y,Z,W)\in \mathbb{A}^3 ~|~ Y^2W-ZW+YZ=0,Y^2-Z\neq 0\}\\
	&=
	\{(Y,Z,W)\in \mathbb{A}^3 ~|~ W=\frac{YZ}{Z-Y^2},Y^2-Z\neq 0\}
\end{align*}
Clearly, they are open subsets of $V$.\\
Let's check that if $(Y,Z,W)\in V$ and $Y^2=Z$, then $YW+Z+W^2\neq0$. If $Y^2=Z$, then the equation $Y^2W-ZW+YZ=0$ becomes $YZ=0$. Therefore, we get $Y=Z=0$. Since $(0,0,0)$ is not contained in $V$, $W$ can only take non-zero values. Therefore, $YW+Z+W^2=W^2\neq0$. We conclude that the images of $i$ and $\iota$ cover $V$.\\
\\
Now let's check that 
\[
	i^{-1}(
	\iota (\{(y,z)\in \mathbb{A}^2~|~z-y^2\neq 0 \})
	)
		=
	\{
	(b,c)\in \mathbb{A}^2~|~
	c-b^2\neq 0,c\neq 0 	
	\}
\]
The image of $\iota$ is
\[
	\iota (\{(y,z)\in \mathbb{A}^2~|~z-y^2\neq 0 \})
	=
	\{(Y,Z,W)\in \mathbb{A}^3 ~|~ W=\frac{YZ}{Z-Y^2},Y^2-Z\neq 0\}
\]
we have
\begin{align*}
i(b,c)\in \iota &(\{(y,z)\in \mathbb{A}^2~|~z-y^2\neq 0 \})\\ \Longleftrightarrow &(-\frac{bc}{b^2-c})^2\neq \frac{c^2}{b^2-c}\\
\Longleftrightarrow & ~b^2c^2\neq c^2(b^2-c)\\
\Longleftrightarrow &~c^3\neq 0 \\
\Longleftrightarrow &~c\neq 0
\end{align*}
Therefore,
\[
	i^{-1}(
	\iota (\{(y,z)\in \mathbb{A}^2~|~z-y^2\neq 0 \})
	)
		=
	\{
	(b,c)\in \mathbb{A}^2~|~
	c-b^2\neq 0,c\neq 0 	
	\}
\]
as desired.\\
Therefore, $V$ is isomorphic to $\mathcal{M}$.
\end{example}
\begin{example}
In this section, we compute the moduli space associated to the braid word $:s_1^3$ with regular unipotent monodromy. The framed moduli space is given by
\begin{align*}
	\mathcal{M}^{fr}=\{
	([x:y],[z:w],[a:b])\in \mathbb{P}^1 \times \mathbb{P}^1\times \mathbb{P}^1~|~
	\operatorname{det}
	\begin{pmatrix}
	x&z\\
	y&w
	\end{pmatrix}
	\neq 0&,\\
	\operatorname{det}
	\begin{pmatrix}
	z&a\\
	w&b
	\end{pmatrix}
	\neq 0,~
	\operatorname{det}
	\begin{pmatrix}
	a&x+y\\
	b&y
	\end{pmatrix}
	\neq 0&	 	
	\}
\end{align*}
The centralizer subgroup of the unipotent matrix
\[u=
	\begin{pmatrix}
		1&1\\
		0&1
	\end{pmatrix}
\] 
is
\[
	C:=\operatorname{C}_{\operatorname{GL}_2(\mathbb{C})}(u)=
	\{
	\begin{pmatrix}
	\alpha&\beta\\
	0&\alpha\\
	\end{pmatrix}
	\in \operatorname{GL}_2(\mathbb{C})~|~
	\alpha \neq 0
	\}
\]
From this point on, we will use the following notation 
\begin{align*}
 &\infty:=[1:0]\\
 &X:=[x:y]\\
 &Z:=[z:w]\\
 &A:=[a:b] 
\end{align*}
We can cover $\mathcal{M}^{fr}$ with open subsets $U_1,U_2$ where 
\begin{align*}
	&U_{1}=\{
	(X,Z,A)\in \mathcal{M}^{fr}~|~X\neq \infty\}\\
	&U_{2}=\{
	(X,Z,A)\in \mathcal{M}^{fr}~|~Z\neq \infty\}\\	
\end{align*}
Therefore, we have a pushout square
\begin{displaymath}
\begin{tikzcd}
  U_1\arrow[hookrightarrow]{r} & \mathcal{M}^{fr} \\
  U_1 \cap U_2 \arrow[hookrightarrow]{u} \arrow[r,hookrightarrow] & U_2 \arrow[hookrightarrow]{u} 
\end{tikzcd} 
\end{displaymath}\\
Quotienting out by the centralizer subgroup $C$, we get
\begin{displaymath}
\begin{tikzcd}
  \overline{U}_1:=C\backslash U_1\arrow[r,hookrightarrow] & \mathcal{M}=C\backslash\mathcal{M}^{fr} \\
  \overline{U_1 \cap U}_2:=C\backslash U_1 \cap U_2 \arrow[hookrightarrow]{u} \arrow[r,hookrightarrow] & \overline{U}_2:=C\backslash U_2 \arrow[hookrightarrow]{u} 
\end{tikzcd} 
\end{displaymath}
Also note that, for the action of $C$, the centralizer subgroup of any element of $\mathcal{M}^{fr}$ is the set of scalar multiplication matrices.\\
First let's simplify, $\overline{U}_1$. Suppose
\[
	\begin{bmatrix}
	x&z&a\\
	y&w&b
	\end{bmatrix}
	\in U_1
\]
Then there exists a 
\[
	\begin{pmatrix}
		\alpha & \beta\\
		0 & \alpha 
	\end{pmatrix}
	\in
	C
\]
such that
\[
	\begin{pmatrix}
		\alpha & \beta\\
		0 & \alpha 	
	\end{pmatrix}
	\cdot
	\begin{bmatrix}
		x\\
		y
	\end{bmatrix}
	=
	\begin{bmatrix}
		0\\
		1
	\end{bmatrix}	
\] 
That is 

	\begin{align*}
		&\alpha=1\\
		&\beta=-\frac{x}{y}
	\end{align*}
This expression makes sense because $y\neq 0$ in $U_1$.
If we take an element of $U_1$ with 
\[
	\begin{bmatrix}
		x\\
		y
	\end{bmatrix}
	=
	\begin{bmatrix}
		0\\
		1
	\end{bmatrix}	
\]
as a representative from each $C$-orbit, we see that 
\begin{align*}
	\overline{U}_1
	&\cong
	\{
	([0:1],[z:w],[a:b])\in \mathbb{P}^1 \times \mathbb{P}^1\times \mathbb{P}^1~|~
	\operatorname{det}
	\begin{pmatrix}
	0&z\\
	1&w
	\end{pmatrix}
	\neq 0,~
	\operatorname{det}
	\begin{pmatrix}
	z&a\\
	w&b
	\end{pmatrix}
	\neq 0,~
	\operatorname{det}
	\begin{pmatrix}
	a&1\\
	b&1
	\end{pmatrix}
	\neq 0	 	
	\}	
	\\
	&\cong
	\{
	([z:w],[a:b])\in \mathbb{P}^1\times \mathbb{P}^1~|~
	z\neq 0,~
	bz-aw\neq 0,~
	a-b\neq 0	 	
	\}\\
	&\cong
	\{
	(w,[a:b])\in \mathbb{A}^1\times \mathbb{P}^1~|~
	b-aw\neq 0,~
	a-b\neq 0	 	
	\}	
\end{align*}
Change variables $a':=a+b,b':=a-b$ we get
\begin{align*}
	\overline{U_1}
	&\cong
	\{
	(w,[a':b'])\in \mathbb{A}^1\times \mathbb{P}^1~|~
	\frac{a'-b'}{2}-\frac{(a'+b')w}{2}\neq 0,~
	b'\neq 0	 	
	\}\\
	&\cong
	\{
	(w,a'')\in \mathbb{A}^1\times \mathbb{A}^1~|~
	(a'-1)-(a'+1)w\neq 0 	
	\}\\
	&\cong
	\{
	(w,a'')\in \mathbb{A}^1\times \mathbb{A}^1~|~
	(a''+1)(1-w)-2\neq 0 	
	\}\\
	&\cong
	\{
	(w',a''')\in \mathbb{A}^1\times \mathbb{A}^1~|~
	a'''w'\neq 1 	
	\}
\end{align*}
Under this identification,
\[
	\overline{U_1\cap U}_2
	\cong
	\{
	(w',a''')\in \mathbb{A}^1\times \mathbb{A}^1~|~
	a'''w'\neq 1, 2w'\neq 1 	
	\}
\]
Now let's simplify, $ \overline{U}_2 $. Suppose
\[
	\begin{bmatrix}
	x&z&a\\
	y&w&b
	\end{bmatrix}
	\in U_2
\]
Then there exists a 
\[
	\begin{pmatrix}
		\alpha & \beta\\
		0 & \alpha 
	\end{pmatrix}
	\in
	C
\]
such that
\[
	\begin{pmatrix}
		\alpha & \beta\\
		0 & \alpha 	
	\end{pmatrix}
	\cdot
	\begin{bmatrix}
		z\\
		w
	\end{bmatrix}
	=
	\begin{bmatrix}
		0\\
		1
	\end{bmatrix}	
\] 
That is 

	\begin{align*}
		&\alpha=1\\
		&\beta=-\frac{z}{w}
	\end{align*}
This expression makes sense because $w\neq 0$ in $U_2$.
If we take an element of $U_2$ with 
\[
	\begin{bmatrix}
		z\\
		w
	\end{bmatrix}
	=
	\begin{bmatrix}
		0\\
		1
	\end{bmatrix}	
\]
as a representative from each $C$-orbit, we see that 
\begin{align*}
	\overline{U}_2
	&\cong
	\{
	([x:y],[0:1],[a:b])\in \mathbb{P}^1 \times \mathbb{P}^1\times \mathbb{P}^1\mid
	\operatorname{det}
	\begin{pmatrix}
	x&0\\
	y&1
	\end{pmatrix}
	\neq 0,
	\operatorname{det}
	\begin{pmatrix}
	0&a\\
	1&b
	\end{pmatrix}
	\neq 0,
	\operatorname{det}
	\begin{pmatrix}
	a&x+y\\
	b&y
	\end{pmatrix}
	\neq 0	 	
	\}	
	\\
	&\cong
	\{
	([x:y],[a:b])\in \mathbb{P}^1\times \mathbb{P}^1~|~
	x\neq 0,~
	a\neq 0,~
	ay-b(x+y)\neq 0	 	
	\}\\
	&\cong
	\{
	(y,b)\in \mathbb{A}^1\times \mathbb{A}^1~|~
	y-b(1+y)\neq 0	 	
	\}\\
	&\cong
	\{
	(y,b)\in \mathbb{A}^1\times \mathbb{A}^1~|~
	y(1-b)-b\neq 0	 	
	\}	
\end{align*}
Change variables $b':=1-b,y':=y+1$ we get
\[
	\overline{U_2}
	\cong
	\{
	(y',b')\in \mathbb{A}^1\times \mathbb{A}^1~|~
	yb'\neq 1
	\}
\]

Under these identifications, the pushout square above becomes
\begin{displaymath}
\begin{tikzcd}
	\{
	(x,y)\in \mathbb{A}^2~|~
	xy\neq 1 	
	\}
  \arrow[hookrightarrow]{r} & \mathcal{M} \\
	\{
	(x,y)\in \mathbb{A}^2~|~
	xy\neq 1,2x\neq 1 	
	\} 
  \arrow[hookrightarrow]{u} \arrow[hookrightarrow]{r}{f} 
  & \{
	(a,b)\in \mathbb{A}^2~|~
	ab\neq 1 	
	\} 
  \arrow[hookrightarrow]{u} 
\end{tikzcd} 
\end{displaymath}\\
where
\begin{align*}
	f:\{(x,y)\in \mathbb{A}^2~|~xy\neq 1,& 2x \neq 1\} 				\rightarrow 
	\{(a,b)\in \mathbb{A}^2~|~ab\neq 1\} \\
	(x,y)&\mapsto (2x,\frac{4x+y-4}{2xy-2})
\end{align*}
Now define a variety $V$ to be 
\begin{align*}
	&V:=\{(A,B,C)\in \mathbb{A}^3 ~|~ (AC-2)B=2A+C-4\}-\{(1,1,2)\}
\end{align*}

I claim that $V$ is isomorphic to $\mathcal{M}$. More precisely, I claim that 

\begin{displaymath}
\begin{tikzcd}
	\{
	(x,y)\in \mathbb{A}^2~|~
	xy\neq 1 	
	\}
  \arrow[hookrightarrow]{r}{i} & V \\
	\{
	(x,y)\in \mathbb{A}^2~|~
	xy\neq 1,2x\neq 1 	
	\} 
  \arrow[hookrightarrow]{u}{g} \arrow[hookrightarrow]{r}{f} 
  & \{
	(a,b)\in \mathbb{A}^2~|~
	ab\neq 1 	
	\} 
  \arrow[hookrightarrow]{u}{\iota} 
\end{tikzcd} 
\end{displaymath}\\
is a pushout square where
\begin{align*}
	i:\{(x,y)\in \mathbb{A}^2~&|~xy\neq 1\} 							\rightarrow 
	V \\
	(x,y)&\mapsto (2x,\frac{4x+y-4}{2xy-2},y)\\
	\iota:\{(a,b)\in \mathbb{A}^2~&|~ab\neq 1 \} 						\rightarrow 
	V \\
	(a,b)&\mapsto (a,b,\frac{2a+2b-4}{ab-1})
\end{align*}

It is easy to see that the square commutes and $f,g,i,\iota$ are inclusion maps.
\\
Let's check that the images of $i$ and $\iota$ form an open cover of $V$.\\
The image of $i$ and $\iota$ are
\[
	i( \{(x,y)\in \mathbb{A}^2~|~xy\neq 1 \})=\{(A,B,C)\in V| AC\neq 2\}
\]
\[
	\iota ( \{(a,b)\in \mathbb{A}^2~|~ab\neq 1 \})=\{(A,B,C)\in V| AB\neq 1\}
\]
Clearly, they are open subsets of $V$.\\
Let's check that if $(A,B,C)\in V$ and $AC=2$, then $AB\neq1$. If $AC=2$, then the left hand side of the equation $(AC-2)B=2A+C-4$ becomes zero. Therefore, we get $AC=2$ and $2A+C=4$ which implies $2A+\frac{2}{A}=4\Leftrightarrow A^2-2A+1=0$. Solving the quadratic equation, we get $A=1,C=2$. Since $(1,1,2)$ is not contained in $V$, $B$ can take any value except $1$. Therefore, $AB\neq 1$. We conclude that the images of $i$ and $\iota$ cover $V$.\\
Now let's check that 
\[
	i^{-1}(
	\iota (\{(a,b)\in \mathbb{A}^2~|~ab\neq 1 \})
	)
		=
	\{
	(x,y)\in \mathbb{A}^2~|~
	xy\neq 1,2x\neq 1 	
	\}
\]
The image of $\iota$ is
\[
	\iota ( \{(a,b)\in \mathbb{A}^2~|~ab\neq 1 \})=\{(A,B,C)\in V| AB\neq 1\}
\]
we have
\begin{align*}
i(x,y)\in \iota &(\{(a,b)\in \mathbb{A}^2~|~ab\neq 1 \})\\ \Longleftrightarrow & 2x\cdot (\frac{4x+y-4}{2xy-2})\neq 1\\
\Longleftrightarrow & 2x\cdot(4x+y-4)\neq 2xy-2\\
\Longleftrightarrow &~8x^2-8x+2\neq 0 \\
\Longleftrightarrow &~2(2x-1)^2\neq 0\\
\Longleftrightarrow &~2x1\neq 1 
\end{align*}
Therefore,
\[
	i^{-1}(
	\iota (\{(a,b)\in \mathbb{A}^2~|~ab\neq 1 \})
	)
		=
	\{
	(x,y)\in \overline{U}_1~|~
	2x\neq 1 	
	\}
\]
as desired.\\
Therefore, $V$ is isomorphic to $\mathcal{M}$.
\end{example}

\begin{example}
In this section, we prove that the moduli space of a bivalent braid with regular unipotent monodromy is finite type integral scheme over $\mathbb{C}$ not necessarily separated. Suppose we have a bivalent braid word with $n$-strands and a unipotent monodromy
\[
	u=
	\begin{pmatrix}
		1&1&0&\cdots&0\\
		0&1&1&\cdots&0\\
		0&0&1&\ddots&\vdots\\		
		\vdots&\vdots&\ddots&\ddots&1\\
		0&0&\cdots&\cdots&1
	\end{pmatrix}
\]
The framed moduli space is given as
\[
	\mathcal{M}^{fr}=\{X=(X^1,X^2,\cdots,X^k)\in (\mathbb{P}^{n-1})^k~|~	f_1(X)\neq0,\cdots f_m(X)\neq 0\}
\]
where $f_i$ 's are determinants with column vectors of the form $u^s \cdot X_j$. Note that the entries of the last row(i.e. the $n^{th}$ row) is one of $x_{n,i}$ $(i=1,\cdots, k)$. Thus $x_{n,i}$ 's $(i=1,\cdots,k)$ cannot be identically zero otherwise all of the $f_r$'s will vanish. Therefore, we have an open cover of $\mathcal{M}^{fr}$, i.e.  $\{ U_i \}_{i=1,\cdots,k}$ where $U_i:=\{X\in \mathcal{M}^{fr} ~|~ x_{n,i}\neq0\}$. \\
To get the moduli space, we quotient the framed moduli space with the centralizer subgroup of $u$ in $\operatorname{GL}_{n}(\mathbb{C})$ i.e.
\[
	C:=	
	\operatorname{C}_{\operatorname{GL}_n(\mathbb{C})}
	=\{	
	\begin{pmatrix}
		\alpha_1&\alpha_2&\cdots&\alpha_{n-1}&\alpha_n\\
		0&\alpha_1&\alpha_2&\cdots&\alpha_{n-1}\\
		0&0&\alpha_1&\ddots&\vdots\\		
		\vdots&\vdots&\ddots&\ddots&\alpha_2\\
		0&0&\cdots&0&\alpha_1
	\end{pmatrix}
	\in \operatorname{GL}_n(\mathbb{C})~|~\alpha_1\in \mathbb{C^*},~\alpha_i\in \mathbb{C}~for~i=2,\cdots,n  \}
\]
Suppose we have an element 
\[
	X=(X^1,\cdots,X^k)=
	\begin{bmatrix}
		x_{1,1}&\cdots&x_{1,n}\\
		\vdots&\ddots&\vdots\\
		x_{n,1}&\cdots&x_{n,n}
	\end{bmatrix}
	\in U_i
\]
Then there exists a 
\[
	\begin{pmatrix}
		\alpha_1&\alpha_2&\cdots&\alpha_{n-1}&\alpha_n\\
		0&\alpha_1&\alpha_2&\cdots&\alpha_{n-1}\\
		0&0&\alpha_1&\ddots&\vdots\\		
		\vdots&\vdots&\ddots&\ddots&\alpha_2\\
		0&0&\cdots&0&\alpha_1
	\end{pmatrix}
	\in
	C
\]
such that
\[
	\begin{pmatrix}
		\alpha_1&\alpha_2&\cdots&\alpha_{n-1}&\alpha_n\\
		0&\alpha_1&\alpha_2&\cdots&\alpha_{n-1}\\
		0&0&\alpha_1&\ddots&\vdots\\		
		\vdots&\vdots&\ddots&\ddots&\alpha_2\\
		0&0&\cdots&0&\alpha_1
	\end{pmatrix}
	\cdot
	\begin{bmatrix}
		x_{1,i}\\
		\vdots\\
		x_{n,i}
	\end{bmatrix}
	=
	\begin{bmatrix}
		0\\
		\vdots\\
		0\\
		1
	\end{bmatrix}	
\] 
That is, recursively, 

	\begin{align*}
		&\alpha_1=1\\
		&\alpha_s=-\frac{1}{x_{n,i}}(x_{n-1,i}\alpha_{s-1}+x_{n-2,i}\alpha_{s-2}+\cdots x_{n-s+1,i}\alpha_{1})
		=-\frac{1}{x_{n,i}}(\sum_{t=1}^{s-1} x_{n-t,i}\cdot\alpha_{s-t} )
	\end{align*}
This expression makes sense because $x_{n,i}\neq 0$ in $U_i$.
If we take an element of $U_i$ with 
\[
	\begin{bmatrix}
		x\\
		y\\
		z
	\end{bmatrix}
	=
	\begin{bmatrix}
		0\\
		0\\
		1
	\end{bmatrix}	
\]
as a representative from each $C$-orbit, we see that 
\begin{align*}
	\overline{U}_i
	:= C\backslash U_i\cong
	\{
	\underset{\substack{\uparrow \\ \text{i}^{th} column}}
	{\begin{bmatrix}
		x_{1,1}&&0 &&x_{1,n}\\
		\vdots&\cdots&\vdots&\cdots&\vdots\\
		x_{n-1,1}&&0&&x_{n-1,n}\\
		x_{n,1}&&1&&x_{n,n}
	\end{bmatrix}}
	\in (\mathbb{P}^{n-1})^k~|
	&~	
	f_1(\begin{bmatrix}
		x_{1,1}&&0 &&x_{1,n}\\
		\vdots&\cdots&\vdots&\cdots&\vdots\\
		x_{n-1,1}&&0&&x_{n-1,n}\\
		x_{n,1}&&1&&x_{n,n}
	\end{bmatrix})\neq0,
	\cdots,\\
	& 
	f_m(\begin{bmatrix}
		x_{1,1}&&0 &&x_{1,n}\\
		\vdots&\cdots&\vdots&\cdots&\vdots\\
		x_{n-1,1}&&0&&x_{n-1,n}\\
		x_{n,1}&&1&&x_{n,n}
	\end{bmatrix})\neq 0\}	
\end{align*}
which are finite type scheme over $\mathbb{C}$. In summary, we have found an finite open cover of $\mathcal{M}$ i.e. $\{\overline{U}_i\}_{i=1,\cdots,k}$ such that each open is a finite type scheme over $\mathbb{C}$. Thus, $\mathcal{M}$ is also a finite type scheme over $\mathcal{C}$. Now we have a smooth surjective map $\pi : \mathcal{M}^{fr}\rightarrow \mathcal{M}$. $\mathcal{M}$ is irreducible because $\mathcal{M}^{fr}$ is irreducible and $\pi$ is surjective. $\mathcal{M}$ is reduced because $\mathcal{M}^{fr}$ is reduced and $\pi$ is smooth. Therefore, we conclude that the moduli space associated attached to bivalent braid with regular unipotent monodromy is finite type integral scheme over $\mathbb{C}$. But it may not be separated.
\end{example}
\begin{example}
In this section, I will provide an example of the moduli space associated to bivalent braid with regular unipotent monodromy that is non-separated. Consider a $2$-strand braid given by the braid word $s_1^2$ and a regular unipotent monodromy
\[
	u=
	\begin{pmatrix}
		1&1\\
		0&1
	\end{pmatrix}
\] 
The framed moduli space is given as follows
\begin{align*}
	\mathcal{M}^{fr}&=\{([x:y],[a:b])\in\mathbb{P}^1\times\mathbb{P}^1~|~[x:y]\neq [a:b],u\cdot[a:b]\neq [x:y]\}\\
	&=\{([x:y],[a:b])\in\mathbb{P}^1\times\mathbb{P}^1~|~[x:y]\neq [a:b],[a+b:b]\neq [x:y]\}\\
	&=\{([x:y],[a:b])\in\mathbb{P}^1\times\mathbb{P}^1~|~ay\neq bx,(a+b)y\neq bx\}
\end{align*}
Then we have an open cover $\{U_i\}_{i=1,2}$ where
\begin{align*}
	U_1:=\{([x:y],[a:b])\in\mathcal{M}^{fr}|b\neq 0\}\\
	U_2:=\{([x:y],[a:b])\in\mathcal{M}^{fr}|y\neq 0\}
\end{align*}
We get a pushout square
\begin{displaymath}
\begin{tikzcd}
	U_1
  \arrow[hookrightarrow]{r} & \mathcal{M}^{fr} \\
	U_1\cap U_2 
  \arrow[hookrightarrow]{u} \arrow[hookrightarrow]{r} 
  & U_2
  \arrow[hookrightarrow]{u} 
\end{tikzcd} 
\end{displaymath}\\
we take quotients of these opens with respect to the centralizer subgroup of $u$ i.e.
\[
	C:=\operatorname{C}_{\operatorname{GL}_2(\mathbb{C})}
	=\{
		\begin{pmatrix}
			\alpha&\beta\\
			0&\alpha	
		\end{pmatrix}\in \operatorname{GL}_{2}(\mathbb{C})
		~|~\alpha\neq 0		
	\}
\]
we get
\begin{align*}
	\overline{U}_1:=C\backslash U_1&\cong\{([x:y],[0:1])\in\mathbb{P}^1\times\mathbb{P}^1~|~0\neq x,y\neq x\}\\
	&\cong\{([1:y],[0:1])\in\mathbb{P}^1\times\mathbb{P}^1~|~y\neq 1\}\\
	%%
	\overline{U}_2:=C\backslash U_2&\cong\{([0:1],[a:b])\in\mathbb{P}^1\times\mathbb{P}^1~|~a\neq 0,(a+b)\neq 0\}\\
	&\cong\{([0:1],[1:b])\in\mathbb{P}^1\times\mathbb{P}^1~|~b\neq -1\}\\
	%%
	\overline{U}_1\cap\overline{U}_2:=C\backslash U_1&\cong\{([x:y],[0:1])\in\mathbb{P}^1\times\mathbb{P}^1~|~0\neq x,y\neq x,y\neq 0\}\\
	&\cong\{([x:y],[0:1])\in\mathbb{P}^1\times\mathbb{P}^1~|~0\neq x,y\neq x,y\neq 0\}\\
	&\cong\{([1:y],[0:1])\in\mathbb{P}^1\times\mathbb{P}^1~|~y\neq 1,y\neq 0\}
\end{align*}
and a pushout square
\begin{displaymath}
\begin{tikzcd}
	\{([1:y],[0:1])\in\mathbb{P}^1\times\mathbb{P}^1~|~y\neq 1\}
  \arrow[hookrightarrow]{r} & \mathcal{M} \\
	\{([1:y],[0:1])\in\mathbb{P}^1\times\mathbb{P}^1~|~y\neq 1,y\neq 0\} 
  \arrow[hookrightarrow]{u} \arrow[hookrightarrow]{r}{f} 
  & \cong\{([0:1],[1:b])\in\mathbb{P}^1\times\mathbb{P}^1~|~b\neq -1\} 
  \arrow[hookrightarrow]{u} 
\end{tikzcd} 
\end{displaymath}\\
where 
\begin{align*}
	f:\{([1:y],[0:1])\in\mathbb{P}^1\times\mathbb{P}^1~|~y\neq 1,y\neq 0\}&\rightarrow \{([0:1],[1:b])\in\mathbb{P}^1\times\mathbb{P}^1~|~b\neq -1\}\\
	&~~\\
	([1:y],[0:1])&\mapsto ([0:1],[1:-y])
\end{align*}
Now consider the map
\begin{align*}
	g:\mathbb{A}^1&-\{0,1\} \longrightarrow \overline{U}_1 \cap\overline{U}_2\subseteq \mathcal{M}\\
	y&\mapsto ([1:y],[0:1])
\end{align*}
This map extends to $\mathbb{A}^1$ in two different ways i.e. we have two distinct $h_1,h_2$ that fit into the following commutative square 
\begin{displaymath}
\begin{tikzcd}
	\mathbb{A}^1-\{0,1\} \arrow[hookrightarrow]{r}  \arrow{rd}{g} 
  & \mathbb{A}^1 -\{1\}\arrow{d}{h_{i}} \\
    & \mathcal{M} 
\end{tikzcd} 
\end{displaymath}
which are
\begin{align*}
	h_1:\mathbb{A}^1&-\{1\} \longrightarrow \overline{U}_1\subseteq \mathcal{M}\\
	y&\mapsto ([1:y],[0:1])
\end{align*}
\begin{align*}
	h_2:\mathbb{A}^1&-\{1\}\longrightarrow \overline{U}_2\subseteq \mathcal{M}\\
	y&\mapsto ([0:1],[1:-y])
\end{align*}
$h_1$ and $h_2$ are distinct because $h_1(0)\in U_1-U2$. Therefore, by the valuative criterion for separatedness, $\mathcal{M}$ is non-separated.
\end{example}