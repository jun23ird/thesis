\section{2nd Sheaf Cobordism}
In this section, we define $cobord_2$, a compactly supported sheaf cobordism.
\subsection{Notations}
%definition of a Riemann sphere with two punctures
\begin{definition}
$M$ denotes a Riemann sphere with two punctures at $0$ and $\infty$. Topologically, $M$ is homeomorphic to a cylinder.
\end{definition}
%definition of \Phi_t^{symbol}, \Xi_t^{symbol}
\begin{definition}
For $t\in\{0,1\}$ and $symbol\in\{0,\infty, squig \}$
\begin{enumerate}
\item we denote $\Phi_t^{symbol}$ to be smooth maps
\begin{align*}
&\Phi_t^0 : (S^1)^n \rightarrow M \\
&\Phi_t^\infty : (S^1)^m \rightarrow M \\
&\Phi_t^{squig} : [0,1]^{k_t} \rightarrow M
\end{align*}

\item we denote $\Xi_t^{symbol}$ a co-orientation of $\Phi_t^{symbol}$.

\item we denote the pair $(\Phi_t^{symbol},\Xi_t^{symbol})$ as $\Lambda_t^{symbol}$. When $symbol \in \{0,\infty\}$, this could be thought as a front projection of a Legendrian living inside the cocircle bundle of $M$, so we will use $\Lambda_t^{symbol}$ to denote both

\item we denote the triple $(\Lambda_t^{0},\Lambda_t^{\infty},\Lambda_t^{squig})$ as $\Lambda_t$ and call it the squiggly diagram at $t$. Later in the section, $\Lambda_0$ will be used to denote the squiggly diagram at the beginning of the isotopy underlying $cobord_2$ and $\Lambda_1$ will be used to denote the squiggly diagram at the end of the isotopy underlying $cobord_2$. 
\end{enumerate}
\end{definition}

%definition of \Phi_\bullet^{symbol}, \Xi_\bullet^{symbol}
\begin{definition}
For $symbol\in\{0,\infty, squig \}$
\begin{enumerate}
\item we denote $\Phi_\bullet^{symbol}$ to be smooth maps
\begin{align*}
&\Phi_\bullet^0 : (S^1)^n \times [0,1]_t \rightarrow M \times [0,1]_t \\
&\Phi_\bullet^\infty : (S^1)^m \times [0,1]_t \rightarrow M \times [0,1]_t \\
&\Phi_\bullet^{squig} : \coprod_{1\leq i \leq k} ([0,1] \times [a_i,b_i]_t) \rightarrow M \times [0,1]_t
\end{align*}
where the maps are identity maps on the time coordinates. I added auxiliary subscript '$t$' to distinguish the time coordinates from the space coordinates.

\item we denote $\Xi_\bullet^{symbol}$ a co-orientation of $\Phi_\bullet^{symbol}$.

\item we denote the pair $(\Phi_\bullet^{symbol},\Xi_\bullet^{symbol})$ as $\Lambda_\bullet^{symbol}$. Later in the section, $\Lambda_\bullet^{symbol}$ will be used to denote the an isotopy from $\Lambda_0^{symbol}$ to $\Lambda_1^{symbol}$ underlying $cobord_2$.

\item we denote the triple $(\Lambda_\bullet^{0},\Lambda_\bullet^{\infty},\Lambda_\bullet^{squig})$ as $\Lambda_\bullet$ and call it a squiggly isotopy from $\Lambda_0$ to $\Lambda_1$. Later in the section, $\Lambda_\bullet$ will be used to denote the isotopy between squiggly diagrams starting from $\Lambda_0$ ending at $\Lambda_1$ underlying $cobord_2$.
\end{enumerate}
\end{definition}

%definition of a bump function
\begin{definition}
For $t \in [0,1]$, we define $\Psi_t: \R \rightarrow \R$ to be a bump function parametrized by $t$ as follows
\[\Psi_t(x)=\bigg\{
\begin{array}{ll}
    \frac{5}{4}e^{(\frac{4x^2}{4x^2 - 3})}(1-t) - \frac{1}{2} & \text{if } |x| < \frac{\sqrt{3}}{2} \\
    -\frac{1}{2} & \text{if } |x| \geq \frac{\sqrt{3}}{2} 
\end{array}
\bigg.
\]
Note that 
\begin{itemize}
\item $supp(\Psi_t) = [-\frac{\sqrt{3}}{2},\frac{\sqrt{3}}{2}]$

\item $\{(-\frac{\sqrt{3}}{2},-\frac{1}{2})$, $(\frac{\sqrt{3}}{2},-\frac{1}{2}),(0,-\frac{5}{4}t + \frac{3}{4})\} \subset Graph(\Psi_t)$
\end{itemize}
\end{definition}

%definition of standard disks
\begin{definition}
We denote the standard open disk in $\R^2$ of radius $r_0$ centered at the origin as 
\[
D_{r=r_0} := \{(x,z)\rightarrow \R^2 ~|~ x^2+z^2 < r_0^2\}
\]
For $t_0 \in [0,1]$, we canonically identify $D_{r=r_0}\times \{t_0\}$ with $D_{r=r_0}$ using the following diffeomorphism
\begin{align*}
& D_{r=r_0} \xrightarrow{\sim} D_{r=r_0} \times \{t_0\} \\
& (x,z) \mapsto (x,z,t_0)
\end{align*}
and with abuse of expression say that sheaves on $D_{r=r_0}\times \{t_0\}$ as sheaves on $D_{r=r_0}$.
\end{definition}

%definition of subsets of D_{r=2}\times \{0\} and their co-orientations
\begin{definition}
\begin{enumerate}
\item We define the following subsets of $D_{r=2} \cong D_{r=2}\times \{0\}$
\begin{itemize}
\item $\lambda_0^0 := \{(x,z) \in D_{r=2} ~|~ z = \Psi_0(x)\}$

\item $\lambda_0^\infty := \{(x,z) \in D_{r=2} ~|~ z = \frac{1}{2} \}$

\item $\lambda_0^{squig}$ is the union of the following two components
\begin{enumerate}[label=(\roman*)]
\item $\{(x,z) \in D_{r=2} ~|~ z = \frac{1}{2} + \sqrt{\frac{1}{16} - (z+\frac{1}{4})^2}, z > \Psi_0(x) \}$

\item $\{(x,z) \in D_{r=2} ~|~ z = \frac{1}{2} + \sqrt{\frac{1}{16} - (z-\frac{1}{4})^2}, z > \Psi_0(x) \}$
\end{enumerate}
\end{itemize}

\item We define co-orientations $\xi_0^{symbol}$ of $\lambda_0^{symbol}$ as follows
\begin{itemize}
\item $\xi_0^0$: hairs are pointing downward direction i.e. coefficients of $dz$ are negative.

\item $\xi_0^\infty$: hairs are pointing upward direction i.e. coefficients of $dz$ are positive.

\item $\xi_0^{squig}$: hairs are pointing upward direction i.e. coefficients of $dz$ are positive.
\end{itemize}
\end{enumerate}
\end{definition}

%definition of subsets of D_{r=2}\times \{1\} and their co-orientations
\begin{definition}
\begin{enumerate}
\item We define the following subsets of $D_{r=2} \cong D_{r=2}\times \{1\}$
\begin{itemize}
\item $\lambda_1^0 := \{(x,z) \in D_{r=2} ~|~ z = \Psi_1(x)\} = \{(x,z) \in D_{r=2} ~|~ z = -\frac{1}{2}\}$

\item $\lambda_1^\infty := \{(x,z) \in D_{r=2} ~|~ z = \frac{1}{2} \}$

\item $\lambda_0^{squig}$ is the union of the following three components
\begin{enumerate}[label=(\roman*)]

\item $\{(x,z) \in D_{r=2} ~|~ z = \frac{1}{2} + \sqrt{\frac{1}{16} - (z+\frac{1}{4})^2}, z > \Psi_1(x) \}\\ = \{(x,z) \in D_{r=2} ~|~ z = \frac{1}{2} + \sqrt{\frac{1}{16} - (z+\frac{1}{4})^2} \}$

\item $\{(x,z) \in D_{r=2} ~|~ z = \frac{1}{2} + \sqrt{\frac{1}{16} - (z-\frac{1}{4})^2}, z > \Psi_1(x) \}\\ = \{(x,z) \in D_{r=2} ~|~ z = \frac{1}{2} + \sqrt{\frac{1}{16} - (z-\frac{1}{4})^2} \}$

\item $\{(0,z) \in D_{r=2} ~|~ -\frac{1}{2} < z < \frac{1}{2} \}$
\end{enumerate}
\end{itemize}

\item We define co-orientations $\xi_1^{symbol}$ of $\lambda_1^{symbol}$ as follows
\begin{itemize}
\item $\xi_1^0$: hairs are pointing downward direction i.e. coefficients of $dz$ are negative.

\item $\xi_1^\infty$: hairs are pointing upward direction i.e. coefficients of $dz$ are positive.

\item $\xi_0^{squig}$: 
\begin{itemize}
\item for (\rn{1}) and (\rn{2}), hairs are pointing upward direction i.e. coefficients of $dz$ are positive.
\item for (\rn{3}), hairs are pointing towards the left i.e. coefficients of $dx$ are positive.
\end{itemize}
\end{itemize}
\end{enumerate}
\end{definition}

%definition of subsets of D_{r=2}\times [0,1] and their co-orientations
\begin{definition}
\begin{enumerate}
\item We define the following subsets of $D_{r=2}\times [0,1]$
\begin{itemize}
\item $\lambda_\bullet^0 := \{(x,z,t) \in D_{r=2} \times [0,1] ~|~ z = \Psi_t(x)\}$

\item $\lambda_\bullet^\infty := \{(x,z,t) \in D_{r=2}\times [0,1] ~|~ z = \frac{1}{2} \}$

\item $\lambda_\bullet^{squig}$ is the union of the following three components
\begin{enumerate}[label=(\roman*)]

\item $\{(x,z,t) \in D_{r=2}\times [0,1] ~|~ z = \frac{1}{2} + \sqrt{\frac{1}{16} - (z+\frac{1}{4})^2}, z > \Psi_t(x) \}$

\item $\{(x,z,t) \in D_{r=2}\times [0,1] ~|~ z = \frac{1}{2} + \sqrt{\frac{1}{16} - (z-\frac{1}{4})^2}, z > \Psi_t(x) \}$

\item $\{(0,z,t) \in D_{r=2} \times [0,1] ~|~ \Psi_t(0) < z < \frac{1}{2} \}$
\end{enumerate}
\end{itemize}

\item We define co-orientations $\xi_\bullet^{symbol}$ of $\lambda_\bullet^{symbol}$ as follows
\begin{itemize}
\item $\xi_\bullet^0$: hairs are pointing downward direction i.e. coefficients of $dz$ are negative.

\item $\xi_\bullet^\infty$: hairs are pointing upward direction i.e. coefficients of $dz$ are positive.

\item $\xi_\bullet^{squig}$: 
\begin{itemize}
\item for (\rn{1}) and (\rn{2}), hairs are pointing upward direction i.e. coefficients of $dz$ are positive.
\item for (\rn{3}), hairs are pointing towards the left i.e. coefficients of $dx$ are positive.
\end{itemize}
\end{itemize}
\end{enumerate}
\end{definition}

% stratifications \mathcal{S}_??
\begin{definition}
\begin{enumerate}
\item Consider a stratification $\mathcal{S}_0$ on $D_{r=2}$ induced by $\lambda_0$ i.e. stratification where $0$ dimensional strata are either crossings or end points of squiggly lines, $1$ dimensional strata are sub-arcs of co-oriented links and squiggly lines that are separated by $0$ dimensional strata, and $2$ dimensional strata are exactly the connected components of $M-\lambda_0$. Note that $1$ dimensional strata has co-orientations inherited from $\lambda_0$.

\item Consider a stratification $\mathcal{S}_1$ on $D_{r=2}$ induced by $\lambda_1$ i.e. stratification where $0$ dimensional strata are either crossings or end points of squiggly lines, $1$ dimensional strata are sub-arcs of co-oriented links and squiggly lines that are separated by $0$ dimensional strata, and $2$ dimensional strata are exactly the connected components of $M-\lambda_1$. Note that $1$ dimensional strata has co-orientations inherited from $\lambda_1$.
\end{enumerate}

\item Consider a stratification $\mathcal{S}_\bullet$ on $D_{r=2}\times [0,1]$ induced by $\lambda_\bullet$ i.e. strata are non-empty finite intersections of $\lambda_\bullet^0$, $\lambda_\bullet^\infty$, and $\lambda_\bullet^{squig}$. Note that $2$ dimensional strata has co-orientations inherited from $\lambda_\bullet$.
\end{definition}

Now let's list the strata of $\mathcal{S}_0$, $\mathcal{S}_1$, and $\mathcal{S}_\bullet$ using the following notations:
%definition of sgn function
\begin{definition}
$sgn : \R \rightarrow \{-,0,+ \}$ is defined as 
\[sgn(x)=\left\{
\begin{array}{ll}
    + & \text{if } x > 0 \\
    0 & \text{if } x = 0 \\
	- & \text{if } x < 0 \\
\end{array}
\right.
\]
\end{definition}

% +,0,- notations of strata s_{t_0}
\begin{definition}
For $i = 0,1,2,3,4$ , $t_0 = 0,1$, and $sgn_i \in \{-,0,+\}$,
\begin{enumerate}
\item we define
\begin{align*}
s_{t_0}(sgn_1,sgn_2,sgn_3,sgn_4):=~ &\{(x,z) \in D_{r=2}\cong D_{r=2}\times \{t_0\} ~| \\
&sgn(z-\Psi_{t_0}(x))=sgn_1,~ sgn(\frac{1}{2}-z)=sgn_2,\\ 
&sgn((x+\frac{1}{4})+(z-\frac{1}{2})^2 - \frac{1}{16})=sgn_3,\\
&sgn((x-\frac{1}{4})+(z-\frac{1}{2})^2 - \frac{1}{16})=sgn_4 \}
\end{align*}

\item $s_{t_0}(sgn_1,sgn_2) := \bigcup_{sgn_3,sgn_4 \in \{-,0,+\}}  s_{t_0}(sgn_1,sgn_2,sgn_3,sgn_4)$

\item $s_{t_0}^{sgn_0}(sgn_1,sgn_2,sgn_3,sgn_4) := s_{t_0}(sgn_1,sgn_2,sgn_3,sgn_4) \cap \{(x,z)\in \R^2 ~|~ sgn(x)=sgn_0 \}$

\item $s_{t_0}^{sgn_0}(sgn_1,sgn_2) := s_{t_0}(sgn_1,sgn_2) \cap \{(x,z)\in \R^2 ~|~ sgn(x)=sgn_0 \}$
\end{enumerate}
\end{definition}

% +,0,- notations of strata s_\bullet
\begin{definition}
For $i = 0,1,2,3,4$ , $t=0,1$, and $sgn_i \in \{-,0,+\}$,
\begin{enumerate}
\item we define
\begin{align*}
s_\bullet(sgn_1,sgn_2,sgn_3,sgn_4):=~ &\{(x,z,t) \in D_{r=2}\times [0,1] ~|\\
&sgn(z-\Psi_t(x))=sgn_1,~ sgn(\frac{1}{2}-z)=sgn_2,\\ 
&sgn((x+\frac{1}{4})+(z-\frac{1}{2})^2 - \frac{1}{16})=sgn_3,\\
&sgn((x-\frac{1}{4})+(z-\frac{1}{2})^2 - \frac{1}{16})=sgn_4 \}
\end{align*}

\item $s_t(sgn_1,sgn_2) := \bigcup_{sgn_3,sgn_4 \in \{-,0,+\}} s_t(sgn_1,sgn_2,sgn_3,sgn_4)$

\item $s_t^{sgn_0}(sgn_1,sgn_2,sgn_3,sgn_4) := s_t(sgn_1,sgn_2,sgn_3,sgn_4) \cap \{(x,z)\in \R^2 ~|~ sgn(x)=sgn_0 \}$

\item $s_t^{sgn_0}(sgn_1,sgn_2) := s_t(sgn_1,sgn_2) \cap \{(x,z)\in \R^2 ~|~ sgn(x)=sgn_0 \}$
\end{enumerate}
\end{definition}


\begin{definition}
Now I will describe $\mathcal{S}_0$, $\mathcal{S}_1$, and $\mathcal{S}_\bullet$ using the above notations:
\begin{enumerate}
\item $\mathcal{S}_0$:
\begin{itemize}
\item $2$ dimensional strata: $s_0(+,-,+,+)$, $s_0(+,-,-,+)$, $s_0(+,-,+,-)$, $s_0(-,-)$, $s_0^-(+,+)$, $s_0^+(+,+)$, $s_0(-,+)$

\item $1$ dimensional strata: $s_0(0,-,+,+)$, $s_0(+,-,0,+)$, $s_0(+,-,+,0)$, $s_0(0,-,-,+)$, $s_0(0,-,+,-)$, $s_0^-(+,0,+,+)$, $s_0^+(+,0,+,+)$, $s_0(+,0,-,+)$, $s_0(+,0,+,-)$, $s_0(-,0)$, $s_0^-(0,+)$, $s_0^+(0,+)$

\item $0$ dimensional strata: $s_0(0,-,0,+)$, $s_0(0,-,+,0)$, $s_0(+,0,0,+)$, $s_0(+,0,+,0)$, $s_0^-(0,0)$, $s_0^+(0,0)$
\end{itemize}

\item $\mathcal{S}_1$:
\begin{itemize}
\item $2$ dimensional strata: $s_1(+,-,+,+)$, $s_1(+,-,-,+)$, $s_1(+,-,+,-)$, $s_1^-(+,+)$, $s_1^+(+,+)$, $s_1(-,+)$

\item $1$ dimensional strata: $s_1(+,-,0,+)$, $s_1(+,-,+,0)$, $s_1^-(+,0,+,+)$, $s_1^+(+,0,+,+)$, $s_1(+,0,-,+)$, $s_1(+,0,+,-)$, $s_1^0(+,+)$, $s_1^-(0,+)$, $s_1^+(0,+)$

\item $0$ dimensional strata: $s_1^0(+,0)$, $s_1(0,-,+,0)$, $s_1(+,0,0,+)$, $s_1(+,0,+,0)$, $s_1^0(0,+)$
\end{itemize}

\item $\mathcal{S}_\bullet$:
\begin{itemize}
\item $3$ dimensional strata: $s_\bullet(+,-,+,+)$, $s_\bullet(+,-,-,+)$, $s_\bullet(+,-,+,-)$, $s_\bullet(-,-)$, $s_\bullet^-(+,+)$, $s_\bullet^+(+,+)$, $s_\bullet(-,+)$

\item $2$ dimensional strata: $s_\bullet(0,-,+,+)$, $s_\bullet(+,-,0,+)$, $s_\bullet(+,-,+,0)$, $s_\bullet(0,-,-,+)$, $s_\bullet(0,-,+,-)$, $s_\bullet^-(+,0,+,+)$, $s_\bullet^+(+,0,+,+)$, $s_\bullet(+,0,-,+)$, $s_\bullet(+,0,+,-)$, $s_\bullet(-,0)$, $s_\bullet^-(0,+)$, $s_\bullet^+(0,+)$, $s_\bullet^0(+,+)$

\item $1$ dimensional strata: $s_\bullet(0,-,0,+)$, $s_\bullet(0,-,+,0)$, $s_\bullet(+,0,0,+)$, $s_\bullet(+,0,+,0)$, $s_\bullet^-(0,0)$, $s_\bullet^+(0,0)$, $s_\bullet^0(+,0)$, $s_\bullet^0(0,+)$

\item $0$ dimensional strata: $s_\bullet(0,0,0,0)$
\end{itemize}
\end{enumerate}
\end{definition}

% definition of the quiver associated to a squiggly diagram
\begin{definition}
Suppose we have a manifold $R$ with stratification $\mathcal{S}$ such that
\begin{itemize}
\item for each codimension $1$ stratum $s\in \mathcal{S}$, there are exactly two regions(= codimension $0$ strata) contained in $start(s)$.

\item each codimension $1$ stratum is equipped with a co-orientation.
\end{itemize}
then we define the quiver associated to $\mathcal{S}$, say $Q_{\mathcal{S}}$, to be a quiver
\begin{itemize}
\item whose vertices corresponds to codimension $0$ strata of $\mathcal{S}$.

\item whose arrows corresponds to codimension $1$ strata of $\mathcal{S}$.

\item the source of an arrow corresponding to $s\in \mathcal{S}$ is  vertex corresponding to the region where the hairs of $s$ are pointing at and the target is the other region contained in the $star(s)$.
\end{itemize}
\end{definition}

% definition of the subquiver associated to a stratum
\begin{definition}
Suppose we have a manifold $R$ with stratification $\mathcal{S}$ such that
\begin{itemize}
\item for each codimension $1$ stratum $s\in \mathcal{S}$, there are exactly two regions(= codimension $0$ strata) contained in $start(s)$.

\item each codimension $1$ stratum is equipped with a co-orientation.
\end{itemize}
then for each $s\in \mathcal{S}$, we define the subquiver of $Q_{\mathcal{S}}$ associated to $s$, say $Q_{\mathcal{S},s}$, to be the full subquiver whose vertices are the ones that corresponds to the regions contained in the start of $s$.
\end{definition}

% definition of legible stratification, quiver
\begin{definition}
Suppose we have a manifold $R$ with stratification $\mathcal{S}$ such that
\begin{itemize}
\item for each codimension $1$ stratum $s\in \mathcal{S}$, there are exactly two regions(= codimension $0$ strata) contained in $start(s)$.

\item each codimension $1$ stratum is equipped with a co-orientation.
\end{itemize}
then $\mathcal{S}$ is a legible stratification if for all $s\in\mathcal{S}$, $Q_{\mathcal{S},s}$ has the initial vertex. We say the quiver $Q_{\mathcal{S}}$ associated to $\mathcal{S}$ is legible if $\mathcal{S}$ is.
\end{definition}

% definition of legible representation
\begin{definition}
Suppose we have a manifold $R$ with stratification $\mathcal{S}$ such that
\begin{itemize}
\item for each codimension $1$ stratum $s\in \mathcal{S}$, there are exactly two regions(= codimension $0$ strata) contained in $start(s)$.

\item each codimension $1$ stratum is equipped with a co-orientation.
\end{itemize}
then we say the quiver representation $F_{\mathcal{S}}$ of $Q_{\mathcal{S}}$ is a legible representation if
\begin{itemize}
\item $\mathcal{S}$ is legible.

\item for any $v,v' \in Vert(Q_{\mathcal{S}})$ and any paths $(a_1,a_2,\cdots,a_k)$,$(a'_1,a'_2,\cdots,a'_{k'})$ from $v$ to $v'$, $F_{\mathcal{S}}(a_k)\circ \cdots F_{\mathcal{S}}(a_1) = F_{\mathcal{S}}(a'_{k'})\circ \cdots F_{\mathcal{S}}(a'_1) $ i.e. the composition is path independent.
\end{itemize}
\end{definition}

% definition of \rho
\begin{definition}
Suppose we have a manifold $R$ with stratification $\mathcal{S}$ such that
\begin{itemize}
\item for each codimension $1$ stratum $s\in \mathcal{S}$, there are exactly two regions(= codimension $0$ strata) contained in $start(s)$.

\item each codimension $1$ stratum is equipped with a co-orientation.
\end{itemize}
Supoose $\mathcal{S}$ is legible, then we define $\rho:\mathcal{S}\rightarrow \{s\in \mathcal{S} ~|~ codim(s)=0 \}$ as
\[
\rho(s):=\text{the codimension $0$ stratum corresponding to the initial vertex of $Q_{\mathcal{S},s}$}
\]
\end{definition}

% definition of the sheaf associated to a legible diagram
\begin{definition}
Suppose we have a manifold $R$ with stratification $\mathcal{S}$ such that
\begin{itemize}
\item for each codimension $1$ stratum $s\in \mathcal{S}$, there are exactly two regions(= codimension $0$ strata) contained in $start(s)$.

\item each codimension $1$ stratum is equipped with a co-orientation.
\end{itemize}
Suppose the quiver representation $F_\mathcal{S}$ of $Q_\mathcal{S}$ is legible, then we define the associated functor $\overline{F_\mathcal{S}}\in Obj(Fun(\mathcal{S}, \C))$ as follows:
\begin{itemize}
\item for $s\in \mathcal{S}$, $\overline{F_\mathcal{S}} := F_\mathcal{S}(\rho(s))$.

\item for $s_1,s_2 \in \mathcal{S}$ where $s_2 \subset start(s_1)$, then $\overline{F_\mathcal{S}}(s_1 \rightarrow s_2)$ is defined as follows: choose a path from the vertex corresponding to $\rho(s_1)$ to $\rho(s_2)$ in $Q_\mathcal{S}$, say $(a_1,\cdots,a_k)$, then 
\[
\overline{F_\mathcal{S}}(s_1 \rightarrow s_2) := F_\mathcal{S}(a_k)\circ\cdots F_\mathcal{S}(a_1)
\] 
This is well-defined because $F_\mathcal{S}$ is legible.
\end{itemize}
\end{definition}

% definition of a co-chain supported in degree 0 and 1 and maps between them
\begin{definition} 
Let $(C^\bullet,\delta^\bullet)$ and $(C'^\bullet,\delta'^\bullet)$ be the cochain complexes $(C^\bullet,\delta^\bullet)$ supported on degree $0$ and $1$ and $\phi^\bullet$ a morphism between $(C^\bullet,\delta^\bullet)$ and $(C'^\bullet,\delta'^\bullet)$, then
\begin{enumerate}
\item we denote $(C^\bullet,\delta^\bullet)$ as either 
\begin{itemize}
\item $C^0 \xrightarrow{\delta^1} C^1$ \\
or  
\item \begin{tikzcd}
C^1 \\
C^0 \arrow[u, "\delta^1"]
\end{tikzcd}
\end{itemize}

\item we denote $\phi^\bullet$ as 
\begin{itemize}
\item \begin{tikzcd}
C^1 \arrow[r, "\phi^1"]     & C'^1  \\
C^0 \arrow[r, "\phi^0"]\arrow[u, "\delta^1"] & C'^0 \arrow[u,"\delta'^1"]
\end{tikzcd}
\end{itemize}
We omit coboundary maps or cochain maps if they are either zero map or identity map and could be inferred from the context.
\end{enumerate}
\end{definition}

\begin{definition}
For $i=1,\cdots,14$, we define $Map_i$ to be morphisms between chain complexes of vector spaces over $\C$ supported on degree $0$ and $1$ as follows:
\begin{itemize}
\item $Map_1$:
\begin{tikzcd}
\C \arrow[r, "\times 1"]     & \C  \\
0 \arrow[r]\arrow[u] & 0 \arrow[u]
\end{tikzcd}

\item $Map_2$:
\begin{tikzcd}
\C \arrow[r]     & 0  \\
0 \arrow[r]\arrow[u] & 0 \arrow[u]
\end{tikzcd}

\item $Map_3$:
\begin{tikzcd}
0 \arrow[r]     & 0  \\
0 \arrow[r]\arrow[u] & 0 \arrow[u]
\end{tikzcd}

\item $Map_4$:
\begin{tikzcd}
0 \arrow[r]     & \C  \\
0 \arrow[r]\arrow[u,] & \C \arrow[u,"\times a"]
\end{tikzcd}

\item $Map_5$:
\begin{tikzcd}
0 \arrow[r]     & \C  \\
0 \arrow[r]\arrow[u] & \C \arrow[u,"\times b"]
\end{tikzcd}

\item $Map_6$:
\begin{tikzcd}
\C \arrow[r, "\times 1"]     & \C  \\
0 \arrow[r]\arrow[u] & \C \arrow[u,"\times a"]
\end{tikzcd}

\item $Map_7$:
\begin{tikzcd}
\C \arrow[r, "\times 1"]     & \C  \\
0 \arrow[r]\arrow[u] & \C \arrow[u,"\times b"]
\end{tikzcd}

\item $Map_8$:
\begin{tikzcd}
0 \arrow[r]     & 0  \\
0 \arrow[r]\arrow[u] & \C \arrow[u]
\end{tikzcd}

\item $Map_9$:
\begin{tikzcd}
\C \arrow[r, "\times 1"]     & \C  \\
\C \arrow[r, "\times 1"]\arrow[u,"\times a"] & \C \arrow[u, "\times a"]
\end{tikzcd}

\item $Map_{10}$:
\begin{tikzcd}
\C \arrow[r, "\times 1"]     & \C  \\
\C \arrow[r, "\times 1"]\arrow[u,"\times b"] & \C \arrow[u, "\times b"]
\end{tikzcd}

\item $Map_{11}$:
\begin{tikzcd}
\C \arrow[r]     & 0  \\
\C \arrow[r, "\times 1"]\arrow[u,"\times a"] & \C \arrow[u]
\end{tikzcd}

\item $Map_{12}$:
\begin{tikzcd}
\C \arrow[r]     & 0  \\
\C \arrow[r, "\times 1"]\arrow[u,"\times b"] & \C \arrow[u]
\end{tikzcd}

\item $Map_{13}$:
\begin{tikzcd}
0 \arrow[r]     & 0  \\
\C \arrow[r, "\times 1"]\arrow[u] & \C \arrow[u]
\end{tikzcd}

\item $Map_{14}$:
\begin{tikzcd}
0 \arrow[r]     & 0  \\
\C \arrow[r, "\times ab^{-1}"]\arrow[u] & \C \arrow[u]
\end{tikzcd}

\end{itemize}

\end{definition}

%%%%%%%%%%%%%%%%%%%%%%%%%%%%%%%%%%%%%%%%%%%%%%%%%%%%%%%%%%%%%%%%%%%
%%                            Setting                            %%
%%%%%%%%%%%%%%%%%%%%%%%%%%%%%%%%%%%%%%%%%%%%%%%%%%%%%%%%%%%%%%%%%%%
\subsection{Setting}
Suppose on $M$, we have
\begin{itemize}
\item  a squiggly diagram $\Lambda_0$ on $M$

\item nested regions $U' \subset U \subset M$. Note that if we define $V:= M - \overline{U'}$, $\{U,V\}$ form an open cover of $M$.

\item a smooth chart from $D_{r=2}$, say $f: D  \rightarrow U \subset M$
\end{itemize}
such that 
\begin{itemize}
\item $D_{r=1}$ is mapped to $U'$ 

\item $\lambda_0^0$ is mapped to $\Lambda_0^0 |_{U}$

\item $\lambda_0^\infty$ is mapped to $\Lambda_0^\infty |_{U}$

\item $\lambda_0^{squig}$ is mapped to $\Lambda_0^{squig} |_{U}$
\end{itemize}

%%%%%%%%%%%%%%%%%%%%%%%%%%%%%%%%%%%%%%%%%%%%%%%%%%%%%%%%%%%%%%%%%%%
%%                         Initial Sheaf                         %%
%%%%%%%%%%%%%%%%%%%%%%%%%%%%%%%%%%%%%%%%%%%%%%%%%%%%%%%%%%%%%%%%%%%
\subsection{Sheaf at the Beginning}
Suppose we have a sheaf $\mathscr{F}_0$ singular supported on $\Lambda_0$ such that $f^*\mathscr{F}_0$ is isomorphic to the sheaf described by the following squiggly legible diagram $F_0$.

For simplicity, we use the following notations
\begin{itemize}
\item $F_0(sgn_1,sgn_2)$:= $F_0(s_0(sgn_1,sgn_2))$
\item $F_0(sgn_1,sgn_2,sgn_3,sgn_4)$:= $F_0(s_0(sgn_1,sgn_2,sgn_3,sgn_4))$
\item $F_0^{sgn_0}(sgn_1,sgn_2)$:= $F_0(s_0^{sgn_0}(sgn_1,sgn_2))$
\item $F_0^{sgn_0}(sgn_1,sgn_2,sgn_3,sgn_4)$:= $F_0(s_0^{sgn_0}(sgn_1,sgn_2,sgn_3,sgn_4))$
\end{itemize}
\textbf{Stalks:}

\begin{itemize}
\item $F_0(+,-,+,+)$ := $0$
\item $F_0(-,-)$ := $\C[-1]$
\item $F_0(+,-,-,+)$ := $\C \xrightarrow{\times a} \C $
\item $F_0(+,-,+,-)$ := $\C \xrightarrow{\times b} \C $
\item $F_0^-(+,+)$ := $\C$
\item $F_0^+(+,+)$ := $\C$
\item $F_0(-,+)$ := $0$
\end{itemize}
\textbf{Generization maps:}

\begin{itemize}
\item $F_0(-,-)\rightarrow F_0(+,-,+,+)$ := $Map_2$
\item $F_0(+,-,+,+)\rightarrow F_0(+,-,-,+)$ := $Map_4$
\item $F_0(+,-,+,+)\rightarrow F_0(+,-,+,-)$ := $Map_5$
\item $F_0(-,-)\rightarrow F_0(+,-,-,+)$ := $Map_6$
\item $F_0(-,-)\rightarrow F_0(+,-,+,-)$ := $Map_7$
\item $F_0(+,-,+,+)\rightarrow F_0^-(+,+)$ := $Map_8$
\item $F_0(+,-,+,+)\rightarrow F_0^+(+,+)$ := $Map_8$
\item $F_0(+,-,-,+)\rightarrow F_0^-(+,+)$ := $Map_{11}$
\item $F_0(+,-,+,-)\rightarrow F_0^+(+,+)$ := $Map_{12}$
\item $F_0(-,-)\rightarrow F_0(-,+)$ := $Map_2$
\item $F_0(-,+)\rightarrow F_0^-(+,+)$ := $Map_8$
\item $F_0(-,+)\rightarrow F_0^+(+,+)$ := $Map_8$

\end{itemize}

%%%%%%%%%%%%%%%%%%%%%%%%%%%%%%%%%%%%%%%%%%%%%%%%%%%%%%%%%%%%%%%%%%%
%%                   Legendrian Cobordism                        %%
%%%%%%%%%%%%%%%%%%%%%%%%%%%%%%%%%%%%%%%%%%%%%%%%%%%%%%%%%%%%%%%%%%%
\subsection{Legendrian Cobordism}
Then define a Legendrian cobordism $\mathscr{F}_\bullet$ starting from $\mathscr{F}_0$, say $cobord_2$, that is supported on $\overline{U'}$ as follows:\\

By Mayer-Vietoris, this equivalent to the following data
\begin{itemize}
\item a sheaf on $V\times [0,1]$, say $\mathscr{F}_{V\times [0,1]}$

\item a sheaf on $D_{r=2}\times [0,1]$, say $\mathscr{F}_{D_{r=2}\times [0,1]}$

\item a gluing isomorphsim, i.e. $\gamma_\bullet : (f_*\mathscr{F}_{D_{r=2}\times [0,1]})|_{(U\cap V)\times [0,1]} \xrightarrow{\sim} \mathscr{F}_{V\times [0,1]}|_{(U\cap V)\times [0,1]}$.
\end{itemize}
\subsubsection{A. Sheaf on $V\times [0,1]$}
First, I will define $\mathscr{F}_{V\times [0,1]}$ to be $pr_1^*(\mathscr{F}_0|_V)$ where $pr_1 : V \times [0,1] \rightarrow V$ is the projection onto the first argument.
\subsubsection{B. Sheaf on $D_{r=2}\times [0,1]$}
Next, I will describe $\mathscr{F}_{D_{r=2}\times [0,1]}$ as $F_\bullet \in Fun(\mathcal{S}_\bullet,\C)$ i.e. a functor from $\mathcal{S}_\bullet$ to the category of perfect $\C$-modules as follows: 

For simplicity, we use the following notations
\begin{itemize}
\item $F_\bullet(sgn_1,sgn_2)$:= $F_\bullet(s_\bullet(sgn_1,sgn_2))$
\item $F_\bullet(sgn_1,sgn_2,sgn_3,sgn_4)$:= $F_\bullet(s_\bullet(sgn_1,sgn_2,sgn_3,sgn_4))$
\item $F^{sgn_0}_\bullet(sgn_1,sgn_2)$:= $F_\bullet(s^{sgn_0}_\bullet(sgn_1,sgn_2))$
\item $F^{sgn_0}_\bullet(sgn_1,sgn_2,sgn_3,sgn_4)$:= $F_\bullet(s^{sgn_0}_\bullet(sgn_1,sgn_2,sgn_3,sgn_4))$
\end{itemize}
\textbf{Stalks:}
\begin{itemize}
\item $3$ dimensional strata:
\begin{itemize}
\item $F_\bullet(+,-,+,+)$ := $0$
\item $F_\bullet(-,-)$ := $\C[-1]$
\item $F_\bullet(+,-,-,+)$ := $\C \xrightarrow{\times a} \C $
\item $F_\bullet(+,-,+,-)$ := $\C \xrightarrow{\times b} \C $
\item $F_\bullet^-(+,+)$ := $\C$
\item $F_\bullet^+(+,+)$ := $\C$
\item $F_\bullet(-,+)$ := $0$
\end{itemize}

\item $2$ dimensional strata:
\begin{itemize}
\item $F_\bullet(0,-,+,+)$ := $\C[-1]$
\item $F_\bullet(+,-,0,+)$ := $0$
\item $F_\bullet(+,-,+,0)$ := $0$
\item $F_\bullet(0,-,-,+)$ := $\C[-1]$
\item $F_\bullet(0,-,+,-)$ := $\C[-1]$
\item $F_\bullet^-(+,0,+,+)$ := $0$
\item $F_\bullet^+(+,0,+,+)$ := $0$
\item $F_\bullet(+,0,-,+)$ := $\C \xrightarrow{\times a} \C $
\item $F_\bullet(+,0,+,-)$ := $\C \xrightarrow{\times b} \C $
\item $F_\bullet(-,0)$ := $\C[-1]$
\item $F_\bullet^-(0,+)$ := $0$
\item $F_\bullet^+(0,+)$ := $0$
\item $F_\bullet^0(+,+)$ := $\C$
\end{itemize}

\item $1$ dimensional strata:
\begin{itemize}
\item $F_\bullet(0,-,0,+)$ := $\C[-1]$
\item $F_\bullet(0,-,+,0)$ := $\C[-1]$
\item $F_\bullet(+,0,0,+)$ := $0$
\item $F_\bullet(+,0,+,0)$ := $0$
\item $F_\bullet^-(0,0)$ := $\C[-1]$
\item $F_\bullet^+(0,0)$ := $\C[-1]$
\item $F_\bullet^0(+,0)$ := $0$
\item $F_\bullet^0(0,+)$ := $0$
\end{itemize}

\item $0$ dimensional stratum:
\begin{itemize}
\item $F_\bullet(0,0,0,0)$ := $\C[-1]$
\end{itemize}
\end{itemize}
\textbf{Exit path maps:}
\begin{itemize}
\item from $0$ to $1$ dimensional strata:
\begin{itemize}
\item $F_\bullet(0,0,0,0)\rightarrow F_\bullet(0,-,0,+)$ := $Map_1$
\item $F_\bullet(0,0,0,0)\rightarrow F_\bullet(0,-,+,0)$ := $Map_1$
\item $F_\bullet(0,0,0,0)\rightarrow F_\bullet^-(0,0)$ := $Map_1$
\item $F_\bullet(0,0,0,0)\rightarrow F_\bullet^+(0,0)$ := $Map_1$
\item $F_\bullet(0,0,0,0)\rightarrow F_\bullet^0(+,0)$ := $Map_2$
\item $F_\bullet(0,0,0,0)\rightarrow F_\bullet^0(0,+)$ := $Map_2$
\end{itemize}

\item from $1$ to $2$ dimensional strata:
\begin{itemize}
\item $F_\bullet(0,-,0,+)\rightarrow F_\bullet(0,-,+,+)$ := $Map_1$
\item $F_\bullet(0,-,0,+)\rightarrow F_\bullet(+,-,0,+)$ := $Map_2$
\item $F_\bullet(0,-,0,+)\rightarrow F_\bullet(0,-,-,+)$ := $Map_1$
\item $F_\bullet(0,-,+,0)\rightarrow F_\bullet(0,-,+,+)$ := $Map_1$
\item $F_\bullet(0,-,+,0)\rightarrow F_\bullet(+,-,+,0)$ := $Map_2$
\item $F_\bullet(0,-,+,0)\rightarrow F_\bullet(0,-,+,-)$ := $Map_1$
\item $F_\bullet(+,0,0,+)\rightarrow F_\bullet(+,-,0,+)$ := $Map_3$
\item $F_\bullet(+,0,0,+)\rightarrow F_\bullet^-(+,0,+,+)$ := $Map_3$
\item $F_\bullet(+,0,0,+)\rightarrow F_\bullet(+,0,-,+)$ := $Map_4$
\item $F_\bullet(+,0,+,0)\rightarrow F_\bullet(+,-,+,0)$ := $Map_3$
\item $F_\bullet(+,0,+,0)\rightarrow F_\bullet^+(+,0,+,+)$ := $Map_3$
\item $F_\bullet(+,0,+,0)\rightarrow F_\bullet(+,0,+,-)$ := $Map_5$
\item $F_\bullet^-(0,0)\rightarrow F_\bullet(0,-,-,+)$ := $Map_1$
\item $F_\bullet^-(0,0)\rightarrow F_\bullet^+(+,0,-,+)$ := $Map_6$
\item $F_\bullet^-(0,0)\rightarrow F_\bullet(-,0)$ := $Map_1$
\item $F_\bullet^-(0,0)\rightarrow F_\bullet^-(0,+)$ := $Map_2$
\item $F_\bullet^+(0,0)\rightarrow F_\bullet(0,-,+,-)$ := $Map_1$
\item $F_\bullet^+(0,0)\rightarrow F_\bullet(+,0,+,-)$ := $Map_7$
\item $F_\bullet^+(0,0)\rightarrow F_\bullet(-,0)$ := $Map_1$
\item $F_\bullet^+(0,0)\rightarrow F_\bullet^+(0,+)$ := $Map_2$
\item $F_\bullet^0(+,0)\rightarrow F_\bullet(+,-,0,+)$ := $Map_3$
\item $F_\bullet^0(+,0)\rightarrow F_\bullet(+,-,+,0)$ := $Map_3$
\item $F_\bullet^0(+,0)\rightarrow F_\bullet(+,0,-,+)$ := $Map_4$
\item $F_\bullet^0(+,0)\rightarrow F_\bullet(+,0,+,-)$ := $Map_5$
\item $F_\bullet^0(+,0)\rightarrow F_\bullet^0(+,+)$ := $Map_8$
\item $F_\bullet^0(0,+)\rightarrow F_\bullet^0(+,+)$ := $Map_8$
\item $F_\bullet^0(0,+)\rightarrow F_\bullet^-(0,+)$ := $Map_3$
\item $F_\bullet^0(0,+)\rightarrow F_\bullet^+(0,+)$ := $Map_3$
\end{itemize}

\item from $2$ to $3$ dimensional strata:
\begin{itemize}
\item $F_\bullet(0,-,+,+)\rightarrow F_\bullet(+,-,+,+)$ := $Map_2$
\item $F_\bullet(0,-,+,+)\rightarrow F_\bullet(-,-)$ := $Map_1$
\item $F_\bullet(+,-,0,+)\rightarrow F_\bullet(+,-,+,+)$ := $Map_3$
\item $F_\bullet(+,-,0,+)\rightarrow F_\bullet(+,-,-,+)$ := $Map_4$
\item $F_\bullet(+,-,+,0)\rightarrow F_\bullet(+,-,+,+)$ := $Map_3$
\item $F_\bullet(+,-,+,0)\rightarrow F_\bullet(+,-,+,-)$ := $Map_5$
\item $F_\bullet(0,-,-,+)\rightarrow F_\bullet(+,-,-,+)$ := $Map_6$
\item $F_\bullet(0,-,-,+)\rightarrow F_\bullet(-,-)$ := $Map_1$
\item $F_\bullet(0,-,+,-)\rightarrow F_\bullet(+,-,+,-)$ := $Map_7$
\item $F_\bullet(0,-,+,-)\rightarrow F_\bullet(-,-)$ := $Map_1$
\item $F_\bullet^-(+,0,+,+)\rightarrow F_\bullet(+,-,+,+)$ := $Map_3$
\item $F_\bullet^-(+,0,+,+)\rightarrow F_\bullet^-(+,+)$ := $Map_8$
\item $F_\bullet^+(+,0,+,+)\rightarrow F_\bullet(+,-,+,+)$ := $Map_3$
\item $F_\bullet^+(+,0,+,+)\rightarrow F_\bullet^+(+,+)$ := $Map_8$
\item $F_\bullet(+,0,-,+)\rightarrow F_\bullet(+,-,-,+)$ := $Map_9$
\item $F_\bullet(+,0,-,+)\rightarrow F_\bullet^-(+,+)$ := $Map_{11}$
\item $F_\bullet(+,0,+,-)\rightarrow F_\bullet(+,-,+,-)$ := $Map_{10}$
\item $F_\bullet(+,0,+,-)\rightarrow F_\bullet^+(+,+)$ := $Map_{12}$
\item $F_\bullet(-,0)\rightarrow F_\bullet(-,-)$ := $Map_1$
\item $F_\bullet(-,0)\rightarrow F_\bullet(-,+)$ := $Map_2$
\item $F_\bullet^-(0,+)\rightarrow F_\bullet^-(+,+)$ := $Map_8$
\item $F_\bullet^-(0,+)\rightarrow F_\bullet(-,+)$ := $Map_3$
\item $F_\bullet^+(0,+)\rightarrow F_\bullet^+(+,+)$ := $Map_8$
\item $F_\bullet^+(0,+)\rightarrow F_\bullet(-,+)$ := $Map_3$
\item $F_\bullet^0(+,+)\rightarrow F_\bullet^-(+,+)$ := $Map_{13}$
\item $F_\bullet^0(+,+)\rightarrow F_\bullet^+(+,+)$ := $Map_{14}$

\end{itemize}
\end{itemize}

\subsubsection{C. Gluing Isomorphism}
Lastly, I will define a gluing isomorphism $\gamma_\bullet : (f_*\mathscr{F}_{D_{r=2}\times [0,1]})|_{(U\cap V)\times [0,1]} \xrightarrow{\sim} \mathscr{F}_{V\times [0,1]}|_{(U\cap V)\times [0,1]}$ using the following fact.
\begin{proposition}
$(f_*\mathscr{F}_{D_{r=2}\times [0,1]})|_{(U\cap V)\times[0,1]}$ is isomorphic to $pr_1^*(\mathscr{F}_0|_{U\cap V})$ where $pr_1 : (U\cap V) \times [0,1] \rightarrow (U\cap V)$ is the projection onto the first argument.
\end{proposition}
\begin{proof}
pass
\end{proof}
\begin{definition}
we define $\gamma_\bullet$ to be the composition 
\[
(f_*\mathscr{F}_{D_{r=2}\times [0,1]})|_{(U\cap V)\times [0,1]}\xrightarrow{\sim}pr_1^*(\mathscr{F}_0|_{U\cap V})\xrightarrow{\sim}pr_1^*(\mathscr{F}_0|_{V})|_{(U\cap V)\times [0,1]}=\mathscr{F}_{V\times [0,1]}|_{(U\cap V)\times [0,1]}
\]
where
\begin{itemize}
\item the first isomorphism is the one mentioned in the above proposition.

\item the second isomorphism from the fact that the following diagram commutes:
\[
\begin{tikzcd}
(U\cap V)\times [0,1] \arrow[hookrightarrow]{r}\arrow[d, "pr_1"]     & V\times [0,1] \arrow[d, "pr_1"] \\
(U\cap V) \arrow[hookrightarrow]{r} & V 
\end{tikzcd}
\]
\end{itemize}
\end{definition}

Now we have defined a cobordism $\mathscr{F}_\bullet$, we show that this is a Legendrian cobordism.
\begin{proposition}
$\mathscr{F}_\bullet$ is a Legendrian cobordism i.e. $\mathscr{F}_\bullet \in Sh_\Lambda(M \times [0,1],\C)$.
\end{proposition}
\begin{proof}
We can prove this by showing that for all $p\in M\times [0,1]$, the local Morse group $G(p)=0$.\\
Recall that the local Morse group of an $\mathcal{S}$-constructible sheaf $\mathscr{F}$ at $p\in R$ is defined as follows: pick a regular neighborhood of $p$, say $N_p \subset R$ and a local Morse function $h: N_p \rightarrow \R$ whose level sets are transverse to (codimension 1 strata of $\mathcal{S}$), then
\[
G(p):= H^{*}(N_p,h^{-1}(-\infty,c-\epsilon];\mathscr{F})
\]
where $c=h(p)$. We can check this locally on each open sets $U\times[0,1]$ and $V\times[0,1]$ because they form an open cover of $R\times [0,1]$
\begin{enumerate}[label=(Case \arabic*)~]
\item $p = (x_0,z_0,t_0) \in V\times[0,1]$\\
Let $N_{(x_0,z_0)}\subset V$ is a regular neighborhood of $(x_0,z_0)$ in $V$, then we choose $N_p = N_{(x_0,z_0)}$ be the regular neighborhood of $p$ in $V\times [0,1]$. We choose the coordinate function $t$ to be our local Morse function, then since $\mathscr{F}_\bullet |_{N_p}$ on $N_p$ deformation retracts to $\mathscr{F}_\bullet |_{N_p \cap t^{-1}((-\infty, t_0 -\epsilon])}$, $G(p):= H^{*}(N_p,h^{-1}(-\infty,c-\epsilon);\mathscr{F})$ is zero. 

\item  $p = (x_0,z_0,t_0) \in U\times[0,1]$\\
\begin{itemize}
\item If $t_0 \neq \frac{1}{5}$, we define the regular neighborhood $N_p$ as 
\[N_p:=\bigg\{
\begin{array}{ll}
    U\times (0,\frac{1}{5}) & \text{if } t_0 < \frac{1}{5} \\
    U\times (\frac{1}{5},1) & \text{if } t_0 \geq \frac{1}{5} 
\end{array}
\bigg.
\]
and the local Morse function to be $t$, then since $\mathscr{F}_\bullet |_{N_p}$ on $N_p$ deformation retracts to $\mathscr{F}_\bullet |_{N_p \cap t^{-1}((-\infty, t_0 -\epsilon])}$, $G(p):= H^{*}(N_p,h^{-1}(-\infty,c-\epsilon);\mathscr{F})$ is zero. 

\item If $t_0 = \frac{1}{5}$ and $(x_0,z_0)\neq (0,\frac{1}{2})$,

\item if $p = (0,\frac{1}{2},\frac{1}{5})$, we define the regular neighborhood $N_p$ as $U\times (0,1)$ and the local Morse function as $h:=t$. I will show that the mapping cone of $R\Gamma(N_p,\mathscr{F}_\bullet|_{N_p})\rightarrow R\Gamma(N_p\cap t^{-1}(-\infty,\frac{1}{5}-\epsilon],\mathscr{F}_\bullet|_{N_p\cap t^{-1}(-\infty,\frac{1}{5}-\epsilon]})$ is acyclic.
\begin{enumerate}[label = (\roman*)]
\item First, let's calculate
\begin{align*}
&R\Gamma(N_p\cap t^{-1}(-\infty,\frac{1}{5}-\epsilon],\mathscr{F}_\bullet|_{N_p\cap t^{-1}(-\infty,\frac{1}{5}-\epsilon]})\\
\cong & R\Gamma(U\times [0,\frac{1}{5}-\epsilon],\mathscr{F}_\bullet|_{U\times [0,\frac{1}{5}-\epsilon]})\\
\cong & R\Gamma(U,\mathscr{F}_\bullet|_{U\times \{0\}})\\
\cong & R\Gamma(U,\mathscr{F}_\bullet|_{U})\\
\cong & R\Gamma(D_{r=2},f^{*}\mathscr{F}_0)
\end{align*}
We have an open cover 
\begin{align*}
\mathcal{U} = \{&U_1 = star(s_0(0,-,0,+)), U_2=star(s_0(0,-,+,0)), U_3=star(s_0(+,0,0,+)),\\ 
&U_4=star(s_0(+,0,+,0)), U_5=star(s_0^-(0,0)), U_6=star(s_0^+(0,0))\}
\end{align*} 
that gives rise to a \v{C}ech nerve of $\mathcal{U}$ which is a simplicial complex whose $k$-simplices corresponds to nonempty intersections of $k+1$ open sets
\[
U_{i_0 i_1 \cdots i_k} = U_{i_0}\cap \cdots U_{i_k}
\]

\begin{itemize}
\item $0$-simplices: $U_1,U_2,U_3,U_4,U_5,U_6$

\item $1$-simplices: $U_{12},U_{13},U_{14},U_{15},U_{16},U_{23},U_{24},U_{25},U_{26},U_{34},U_{35},U_{46},U_{56}$

\item $2$-simplices: $U_{123},U_{124},U_{125},U_{126},U_{134},U_{135},U_{156},U_{234},U_{246},U_{256},U_{123}$

\item $3$-simplices: $U_{1234},U_{1256}$
\end{itemize}
Now we define the \v{C}ech double complex where
\[
C^{p,q} := \underset{i_0<\cdots<i_p}{\prod} \mathscr{F}^q(U_{i_0 \cdots i_p})
\]
with 
\begin{itemize}
\item horizontal differential comes from \v{C}ech nerve
\item vertical differential comes from the sheaf cochain complex
\end{itemize}
$(\{C^{p,q}\},\{d^{p,q}\})$:=
\[
\begin{tikzcd}[scale cd=0.8]
\C^4_{(1,2,5,6)} \arrow[r, "(1)"]     & \C^8_{(12,15,16,25,26,56,35,46)}  \arrow[r, "(2)"]     & \C^6_{(125,126,156,256,135,246)} \arrow[r, "(3)"]     & \C_{1256}\\
0 \arrow[r]\arrow[u] & C^2_{(35,46)} \arrow[r,"(4)"]\arrow[u,"(5)"]& C^2_{(135,246)} \arrow[r]\arrow[u,"(6)"]& 0 \arrow[u]
\end{tikzcd}
\]
\begin{enumerate}[label= (\arabic*)]
\item

\item

\item

\item

\item

\item
\end{enumerate}
\pagebreak
So we get the total complex
\[
Tot(\begin{tikzcd}[scale cd=0.5]
\C^4_{(1,2,5,6)} \arrow[r, "(1)"]     & \C^8_{(12,15,16,25,26,56,35,46)}  \arrow[r, "(2)"]     & \C^6_{(125,126,156,256,135,246)} \arrow[r, "(3)"]     & \C_{1256}\\
0 \arrow[r]\arrow[u] & C^2_{(35,46)} \arrow[r,"(4)"]\arrow[u,"(5)"]& C^2_{(135,246)} \arrow[r]\arrow[u,"(6)"]& 0 \arrow[u]
\end{tikzcd})
\]
=
\begin{align*}
 0 &\rightarrow \C^6_{(1,2,5,6,35,46)}  \overset{(\rn{1})}{\longrightarrow} \C^10_{(12,15,16,25,26,56,35,46,135,246)}\\ 
&\overset{(\rn{2})}{\longrightarrow} \C^6_{(125,126,156,256,135,246)} \overset{(\rn{3})}{\longrightarrow} \C_{1256} \rightarrow 0
\end{align*}
\begin{enumerate}[label= (\roman*)]
\item

\item

\item
\end{enumerate}
\pagebreak
\item Next, let's calculate $R\Gamma(N_p, \mathscr{F}_\bullet|_{N_p})$. Since $N_p$ is a regular neighborhood of the point stratum $p$,
\[
R\Gamma(N_p, \mathscr{F}_\bullet|_{N_p})\cong (\mathscr{F}_\bullet)_p = \C[-1]
\]
We can see this fact by using a singleton open cover $\{star(p)\}$ of $U\times (0,1)$ so the \v{C}ech double complex is the following double complex concentrated in degree $(0,1)$:
\[
\begin{tikzcd}
\C_{star(p)} \arrow[r]     & 0 \arrow[r]     & 0\arrow[r]     & 0\\
0 \arrow[r]\arrow[u]     & 0 \arrow[r]\arrow[u]     & 0\arrow[r]\arrow[u]     & 0\arrow[u]
\end{tikzcd}
\]
whose total complex is $\C[-1]$.

\item Lastly, let's compute the map 
\[
\phi^{\bullet,\bullet} : R\Gamma(N_p,\mathscr{F}_\bullet|_{N_p})\rightarrow R\Gamma(N_p\cap t^{-1}(-\infty,\frac{1}{5}-\epsilon],\mathscr{F}_\bullet|_{N_p\cap t^{-1}(-\infty,\frac{1}{5}-\epsilon]})
\]
induced by the inclusion $N_p\cap t^{-1}(-\infty,\frac{1}{5}-\epsilon] \subset N_p$. At a double complex level, note that the only possible nonzero map is $\phi^{0,1}$,
\[
\phi^{0,1} =
\begin{blockarray}{ccccc}
 & \underset{\downarrow}{1} & \underset{\downarrow}{2} & \underset{\downarrow}{5} & \underset{\downarrow}{6} \\
\begin{block}{c(cccc)}
\text{star}(p)\rightarrow & 1 & 1 & 1 & 1 \\
\end{block}
\end{blockarray}
\]
and the induced map between total complex is

\[
\begin{tikzcd}[scale cd=0.8]
0 \arrow[r]\arrow[d]     & \C_{star(p)} \arrow[r]\arrow[d, "(*)"]     & 0\arrow[r]\arrow[d]     & 0\arrow[r]\arrow[d] & 0\arrow[d]\\
0 \arrow[r] & \C^6_{(1,2,5,6,35,46)} \arrow[r,"(\rn{1})"] & \C^10_{(12,15,16,25,26,56,35,46,135,246)}\arrow[r,"(\rn{2})"] &\C^6_{(125,126,156,256,135,246)} \arrow[r,"(\rn{3})"] &\C_{1256}
\end{tikzcd}
\]

where
\[
(*) =
\begin{blockarray}{ccccccc}
 & \underset{\downarrow}{1} & \underset{\downarrow}{2} & \underset{\downarrow}{5} & \underset{\downarrow}{6} & \underset{\downarrow}{35} & \underset{\downarrow}{46}\\
\begin{block}{c(cccccc)}
\text{star}(p)\rightarrow & 1 & 1 & 1 & 1 & 1 & 1 \\
\end{block}
\end{blockarray}
\]
Now I will show that this is a quasi-isomorphism. (pass for now)
\end{enumerate}
\end{itemize}
\end{enumerate}
\end{proof}

%%%%%%%%%%%%%%%%%%%%%%%%%%%%%%%%%%%%%%%%%%%%%%%%%%%%%%%%%%%%%%%%%%%
%%                       Terminal Sheaf                          %%
%%%%%%%%%%%%%%%%%%%%%%%%%%%%%%%%%%%%%%%%%%%%%%%%%%%%%%%%%%%%%%%%%%%
\subsection{Sheaf at the End}
In this subsection, I will describe the sheaf $\mathscr{F}_1$ at the end of the $cobord_2$. By Mayer-Vietoris, $\mathscr{F}_1:= \mathscr{F}_\bullet|_{M\times\{1\}}$ on $M \cong M\times\{1\}$ is equivalent to the following data
\begin{itemize}
\item a sheaf on $V$, say $\mathscr{F}_{V}$

\item a sheaf on $D_{r=2}$, say $\mathscr{F}_{D_{r=2}}$

\item a gluing isomorphsim $\gamma_1 : f_*\mathscr{F}_{D_{r=2}}|_{U\cap V} \xrightarrow{\sim} \mathscr{F}_{V}|_{U\cap V}$.
\end{itemize}

\subsubsection{A. Sheaf on $V$}
First, a sheaf on $V\cong V\times\{1\}$ is the restriction of $\mathscr{F}_{V\times [0,1]}$ to $V\times \{1\}$, i.e. $\mathscr{F}_{V\times [0,1]}|_{V\times \{1\}}= pr_1^*(\mathscr{F}_0|_V)|_{V\times \{1\}} = \mathscr{F}_0|_V$.
\subsubsection{B. Sheaf on $D_{r=2}$}
Next, a sheaf on $D_{r=2}\cong D_{r=2}\times \{1\}$ is the restriction of $\mathscr{F}_{D_{r=2}\times [0,1]}$ to $D_{r=2}\times \{1\}$, i.e. $\mathscr{F}_{D_{r=2}\times [0,1]} |_{D_{r=2}\times \{1\}}$. I will describe it as a squiggly legible diagram $F_1$ which is the restriction of $F_\bullet$ defined in the previous section.

For simplicity, we use the following notations
\begin{itemize}
\item $F_1(sgn_1,sgn_2)$:= $F_1(s_1(sgn_1,sgn_2))$
\item $F_1(sgn_1,sgn_2,sgn_3,sgn_4)$:= $F_1(s_1(sgn_1,sgn_2,sgn_3,sgn_4))$
\item $F_1^{sgn_0}(sgn_1,sgn_2)$:= $F_1(s_1^{sgn_0}(sgn_1,sgn_2))$
\item $F_1^{sgn_0}(sgn_1,sgn_2,sgn_3,sgn_4)$:= $F_1(s_1^{sgn_0}(sgn_1,sgn_2,sgn_3,sgn_4))$
\end{itemize}
\textbf{Stalks:}

\begin{itemize}
\item $F_1(+,-,+,+)$ := $0$
\item $F_1(+,-,-,+)$ := $\C \xrightarrow{\times a} \C $
\item $F_1(+,-,+,-)$ := $\C \xrightarrow{\times b} \C $
\item $F_1^-(+,+)$ := $\C$
\item $F_1^+(+,+)$ := $\C$
\item $F_1(-,+)$ := $0$
\end{itemize}
\textbf{Generization maps:}

\begin{itemize}
\item $F_1(+,-,+,+)\rightarrow F_1(+,-,-,+)$ := $Map_4$
\item $F_1(+,-,+,+)\rightarrow F_1(+,-,+,-)$ := $Map_5$
\item $F_1(+,-,+,+)\rightarrow F_1^-(+,+)$ := $Map_8$
\item $F_1(+,-,+,+)\rightarrow F_1^+(+,+)$ := $Map_8$
\item $F_1(+,-,-,+)\rightarrow F_1^-(+,+)$ := $Map_{11}$
\item $F_1(+,-,+,-)\rightarrow F_1^+(+,+)$ := $Map_{12}$
\item $F_1^-(+,+)\rightarrow F_1^+(+,+)$ := $Map_{14}$
\item $F_1(-,+)\rightarrow F_1^-(+,+)$ := $Map_8$
\item $F_1(-,+)\rightarrow F_1^+(+,+)$ := $Map_8$
\end{itemize}

\subsubsection{C. Gluing Isomorphism}
Lastly, the gluing isomorphism $\gamma_1 := \gamma_\bullet|_{(U\cap V)\times \{1\}}:  f_*\mathscr{F}_{D_{r=2}}|_{(U\cap V)\times \{1\}}\xrightarrow{\sim} \mathscr{F}_0|_{U\cap V}$ is described as follows.

\begin{definition}
we define $\gamma_1$ to be the composition
\begin{align*}
&(f_*\mathscr{F}_{D_{r=2}})|_{(U\cap V)\times \{1\}}\xrightarrow{\sim}pr_1^*(\mathscr{F}_0|_{U\cap V})|_{(U\cap V)\times \{1\}}\xrightarrow{\sim}\mathscr{F}_0|_{U\cap V}
\end{align*}
where
\begin{itemize}
\item the first isomorphism follows from the fact that $(f_*\mathscr{F}_{D_{r=2}\times [0,1]})|_{(U\cap V)\times[0,1]}$ is isomorphic to $pr_1^*(\mathscr{F}_0|_{U\cap V})$.

\item the second isomorphism follows from the fact that the following composition is an identity map:
\[
(U\cap V)\xrightarrow{\sim} (U\cap V)\times \{1\} \hookrightarrow (U\cap V)\times [0,1] \twoheadrightarrow (U\cap V)
\]
\end{itemize}
\end{definition}