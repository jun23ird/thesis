\section{1st Sheaf Cobordism}
In this section, we define $cobord_1$, a compactly supported sheaf 	cobordism.
\subsection{Notations}
%definition of a Riemann sphere with two punctures
\begin{definition}
$M$ denotes a Riemann sphere with two punctures at $0$ and $\infty$. Topologically, $M$ is homeomorphic to a cylinder.
\end{definition}
%definition of \Phi_t^{symbol}, \Xi_t^{symbol}
\begin{definition}
For $t_0\in\{0,1\}$ and $symbol\in\{0,\infty, squig \}$
\begin{enumerate}
\item we denote $\Phi_{t_0}^{symbol}$ to be smooth maps
\begin{align*}
&\Phi_{t_0}^0 : (S^1)^n \rightarrow M \\
&\Phi_{t_0}^\infty : (S^1)^m \rightarrow M \\
&\Phi_{t_0}^{squig} : [0,1]^{k_{t_0}} \rightarrow M
\end{align*}

\item we denote $\Xi_{t_0}^{symbol}$ a co-orientation of $\Phi_{t_0}^{symbol}$.

\item we denote the pair $(\Phi_{t_0}^{symbol},\Xi_{t_0}^{symbol})$ as $\Lambda_{t_0}^{symbol}$. When $symbol \in \{0,\infty\}$, this could be thought as a front projection of a Legendrian living inside the cocircle bundle of $M$, so we will use $\Lambda_{t_0}^{symbol}$ to denote both

\item we denote the triple $(\Lambda_{t_0}^{0},\Lambda_{t_0}^{\infty},\Lambda_{t_0}^{squig})$ as $\Lambda_{t_0}$ and call it the squiggly diagram at $t_0$. Later in the section, $\Lambda_0$ will be used to denote the squiggly diagram at the beginning of the isotopy underlying $cobord_1$ and $\Lambda_1$ will be used to denote the squiggly diagram at the end of the isotopy underlying $cobord_1$. 
\end{enumerate}
\end{definition}

%definition of \Phi_\bullet^{symbol}, \Xi_\bullet^{symbol}
\begin{definition}
For $symbol\in\{0,\infty, squig \}$
\begin{enumerate}
\item we denote $\Phi_\bullet^{symbol}$ to be smooth maps
\begin{align*}
&\Phi_\bullet^0 : (S^1)^n \times [0,1]_t \rightarrow M \times [0,1]_t \\
&\Phi_\bullet^\infty : (S^1)^m \times [0,1]_t \rightarrow M \times [0,1]_t \\
&\Phi_\bullet^{squig} : \coprod_{1\leq i \leq k} ([0,1] \times [a_i,b_i]_t) \rightarrow M \times [0,1]_t
\end{align*}
where the maps are identity maps on the time coordinates. I added auxiliary subscript '$t$' to distinguish the time coordinates from the space coordinates.

\item we denote $\Xi_\bullet^{symbol}$ a co-orientation of $\Phi_\bullet^{symbol}$.

\item we denote the pair $(\Phi_\bullet^{symbol},\Xi_\bullet^{symbol})$ as $\Lambda_\bullet^{symbol}$. Later in the section, $\Lambda_\bullet^{symbol}$ will be used to denote the an isotopy from $\Lambda_0^{symbol}$ to $\Lambda_1^{symbol}$ underlying $cobord_1$.

\item we denote the triple $(\Lambda_\bullet^{0},\Lambda_\bullet^{\infty},\Lambda_\bullet^{squig})$ as $\Lambda_\bullet$ and call it a squiggly isotopy from $\Lambda_0$ to $\Lambda_1$. Later in the section, $\Lambda_\bullet$ will be used to denote the isotopy between squiggly diagrams starting from $\Lambda_0$ ending at $\Lambda_1$ underlying $cobord_1$.
\end{enumerate}
\end{definition}

%definition of a bump function
\begin{definition}
For $t \in [0,1]$, we define $\Psi_t: \R \rightarrow \R$ to be a bump function parametrized by $t$ as follows
\[\Psi_t(x)=\bigg\{
\begin{array}{ll}
    -\frac{3}{4}e^{(\frac{x^2}{x^2 - 1})}t & \text{if } |x| < 1 \\
    0 & \text{if } |x| \geq 1
\end{array}
\bigg.
\]
Note that 
\begin{itemize}
\item $supp(\Psi_t) = [-1,1]$ if $t\neq 0$

\item $\{(1,0)$, $(-1,0),(0, -\frac{3}{4}t)\} \subset Graph(\Psi_t)$
\end{itemize}
\end{definition}

%definition of standard disks
\begin{definition}
We denote the standard open disk in $\R^2$ of radius $r_0$ centered at the origin as 
\[
D_{r=r_0} := \{(x,z)\rightarrow \R^2 ~|~ x^2+z^2 < r_0^2\}
\]
For $t_0 \in [0,1]$, we canonically identify $D_{r=r_0}\times \{t_0\}$ with $D_{r=r_0}$ using the following diffeomorphism
\begin{align*}
& D_{r=r_0} \xrightarrow{\sim} D_{r=r_0} \times \{t_0\} \\
& (x,z) \mapsto (x,z,t_0)
\end{align*}
and with abuse of expression say that sheaves on $D_{r=r_0}\times \{t_0\}$ as sheaves on $D_{r=r_0}$.
\end{definition}

%definition of subsets of D_{r=2}\times \{0\} and their co-orientations
\begin{definition}
\begin{enumerate}
\item We define the following subsets of $D_{r=2} \cong D_{r=2}\times \{0\}$
\begin{itemize}
\item $\lambda_0^0 := \{(x,z) \in D_{r=2} ~|~ z = \Psi_0(x)\}=\{(x,z) \in D_{r=2} ~|~ z = 0\}$

\item $\lambda_0^\infty:=\{(x,z) \in D_{r=2} ~|~ z= -\frac{1}{2}\}$

\item $\lambda_0^{squig}:=\{(x,z) \in D_{r=2} ~|~ x= 0\}$
\end{itemize}

\item We define co-orientations $\xi_0^{symbol}$ of $\lambda_0^{symbol}$ as follows
\begin{itemize}
\item $\xi_0^0$: hairs are pointing downward direction i.e. coefficients of $dz$ are negative.

\item $\xi_0^\infty$: hairs are pointing upward direction i.e. coefficients of $dz$ are positive.

\item $\xi_0^{squig}$: hairs are pointing towards the left i.e. coefficients of $dx$ are negative.
\end{itemize}
\end{enumerate}
\end{definition}

%definition of subsets of D_{r=2}\times \{1\} and their co-orientations
\begin{definition}
\begin{enumerate}
\item We define the following subsets of $D_{r=2} \cong D_{r=2}\times \{1\}$
\begin{itemize}
\item $\lambda_1^0 := \{(x,z) \in D_{r=2} ~|~ z = \Psi_1(x)\}$

\item $\lambda_1^\infty:= \{(x,z) \in D_{r=2} ~|~ z = -\frac{1}{2}\}$ 

\item $\lambda_1^{squig}:= \{(x,z) \in D_{r=2} ~|~ x = 0\}$ 
\end{itemize}

\item We define co-orientations $\xi_1^{symbol}$ of $\lambda_1^{symbol}$ as follows
\begin{itemize}
\item $\xi_1^0$: hairs are pointing downward direction i.e. coefficients of $dz$ are negative.

\item $\xi_1^\infty$: hairs are pointing upward direction i.e. coefficients of $dz$ are positive.

\item $\xi_1^{squig}$: hairs are pointing towards the left i.e. coefficients of $dx$ are negative.
\end{itemize}
\end{enumerate}
\end{definition}

%definition of subsets of D_{r=2}\times [0,1] and their co-orientations
\begin{definition}
\begin{enumerate}
\item We define the following subsets of $D_{r=2}\times [0,1]$
\begin{itemize}
\item $\lambda_\bullet^0 := \{(x,z,t) \in D_{r=2} \times [0,1] ~|~ z = \Psi_t(x)\}$

\item $\lambda_\bullet^\infty:=\{(x,z,t) \in D_{r=2}\times [0,1] ~|~ z = -\frac{1}{2}\}$

\item $\lambda_\bullet^{squig} := \{(x,z,t) \in D_{r=2}\times [0,1] ~|~ x=0 \}$
\end{itemize}

\item We define co-orientations $\xi_\bullet^{symbol}$ of $\lambda_\bullet^{symbol}$ as follows
\begin{itemize}
\item $\xi_\bullet^0$: hairs are pointing downward direction i.e. coefficients of $dz$ are negative.

\item $\xi_\bullet^\infty$: hairs are pointing upward direction i.e. coefficients of $dz$ are positive.

\item $\xi_\bullet^{squig}$: hairs are pointing towards the left i.e. coefficients of $dx$ are positive.
\end{itemize}
\end{enumerate}
\end{definition}

% stratifications \mathcal{S}_??
\begin{definition}
\begin{enumerate}
\item Consider a stratification $\mathcal{S}_0$ on $D_{r=2}$ induced by $\lambda_0$ i.e. stratification where $0$ dimensional strata are either crossings or end points of squiggly lines, $1$ dimensional strata are sub-arcs of co-oriented links and squiggly lines that are separated by $0$ dimensional strata, and $2$ dimensional strata are exactly the connected components of $M-\lambda_0$. Note that $1$ dimensional strata has co-orientations inherited from $\lambda_0$.

\item Consider a stratification $\mathcal{S}_1$ on $D_{r=2}$ induced by $\lambda_1$ i.e. stratification where $0$ dimensional strata are either crossings or end points of squiggly lines, $1$ dimensional strata are sub-arcs of co-oriented links and squiggly lines that are separated by $0$ dimensional strata, and $2$ dimensional strata are exactly the connected components of $M-\lambda_1$. Note that $1$ dimensional strata has co-orientations inherited from $\lambda_1$.
\end{enumerate}

\item Consider a stratification $\mathcal{S}_\bullet$ on $D_{r=2}\times [0,1]$ induced by $\lambda_\bullet$ i.e. strata are non-empty finite intersections of $\lambda_\bullet^0$, $\lambda_\bullet^\infty$, and $\lambda_\bullet^{squig}$. Note that $2$ dimensional strata has co-orientations inherited from $\lambda_\bullet$.
\end{definition}

Now let's list the strata of $\mathcal{S}_0$, $\mathcal{S}_1$, and $\mathcal{S}_\bullet$ using the following notations:
%definition of sgn function
\begin{definition}
$sgn : \R \rightarrow \{-,0,+ \}$ is defined as 
\[sgn(x)=\left\{
\begin{array}{ll}
    + & \text{if } x > 0 \\
    0 & \text{if } x = 0 \\
	- & \text{if } x < 0 \\
\end{array}
\right.
\]
\end{definition}

% +,0,- notations of strata s_{t_0}
\begin{definition}
For $i = 1,2,3$ , $t_0 = 0,1$, and $sgn_i \in \{-,0,+\}$, we define
\begin{align*}
s_{t_0}(sgn_1,sgn_2,sgn_3):=~ &\{(x,z) \in D_{r=2}\cong D_{r=2}\times \{t_0\} ~| \\
&sgn(z-\Psi_{t_0}(x))=sgn_1,~ sgn(-\frac{1}{2}-z)=sgn_2,\\ 
&sgn((x)=sgn_3\}
\end{align*}
\end{definition}

% +,0,- notations of strata s_\bullet
\begin{definition}
For $i = 1,2,3$ and $sgn_i \in \{-,0,+\}$, we define
\begin{align*}
s_\bullet(sgn_1,sgn_2,sgn_3):=~ &\{(x,z,t) \in D_{r=2}\times [0,1] ~| \\
&sgn(z-\Psi_{t}(x))=sgn_1,~ sgn(\frac{1}{2}-z)=sgn_2,\\ 
&sgn((x)=sgn_3\}
\end{align*}
\end{definition}


\begin{definition}
Now I will describe $\mathcal{S}_0$, $\mathcal{S}_1$, and $\mathcal{S}_\bullet$ using the above notations:
\begin{enumerate}
\item $\mathcal{S}_0$:
\begin{itemize}
\begin{figure}[H]
    \centering
    \includegraphics[scale = 0.95]{diagrams/lemma1/1.png} 
    \caption{2 dimensional strata of $\mathcal{S}_0$}
    \label{fig:your-label}
\end{figure}
\item $2$ dimensional strata: \\
$s_0(+,-,-)$,$s_0(+,-,+)$,$s_0(-,-,-)$,$s_0(-,-,+)$,$s_0(-,+,-)$,$s_0(-,+,+)$


\begin{figure}[H]
    \centering
    \includegraphics[scale = 0.95]{diagrams/lemma1/2.png} 
    \caption{1 dimensional strata of $\mathcal{S}_0$}
    \label{fig:your-label}
\end{figure}
\item $1$ dimensional strata:\\
$s_0(0,-,-)$,$s_0(0,-,+)$,$s_0(-,0,-)$,$s_0(-,0,+)$,$s_0(-,-,0)$,$s_0(-,+,0)$,$s_0(0,-,+)$

\begin{figure}[H]
    \centering
    \includegraphics[scale = 0.95]{diagrams/lemma1/3.png} 
    \caption{0 dimensional strata of $\mathcal{S}_0$}
    \label{fig:your-label}
\end{figure}
\item $0$ dimensional strata: \\
$s_0(0,-,0)$, $s_0(-,0,0)$
\end{itemize}

\item $\mathcal{S}_1$:
\begin{itemize}
\begin{figure}[H]
    \centering
    \includegraphics[scale = 0.95]{diagrams/lemma1/4.png} 
    \caption{2 dimensional strata of $\mathcal{S}_1$}
    \label{fig:your-label}
\end{figure}
\item $2$ dimensional strata: \\
$\{s_1(sgn_2,sgn_1,sgn_3) ~|~ sgn_i \in \{-,+\}\text{ for i=1,2,3} \}$

\begin{figure}[H]
    \centering
    \includegraphics[scale = 0.95]{diagrams/lemma1/5.png} 
    \caption{1 dimensional strata of $\mathcal{S}_1$}
    \label{fig:your-label}
\end{figure}
\item $1$ dimensional strata:\\
$\{s_1(0,sgn_2,sgn_3) ~|~ sgn_i \in \{-,+\}\text{ for i=2,3}\} \cup \{s_1(sgn_1,0,sgn_3) ~|~ sgn_i \in \{-,+\}\text{ for i=1,3}\} \cup \{s_1(sgn_1,sgn_2,0) ~|~ sgn_i \in \{-,+\}\text{ for i=1,2, except } s_1(-,-,0)\}$ 

\begin{figure}[H]
    \centering
    \includegraphics[scale = 0.95]{diagrams/lemma1/6.png} 
    \caption{0 dimensional strata of $\mathcal{S}_1$}
    \label{fig:your-label}
\end{figure}
\item $0$ dimensional strata: \\
$s_1(0,0,-)$, $s_1(0,+,0)$, $s_1(0,0,+)$, $s_1(+,0,0)$
\end{itemize}

\item $\mathcal{S}_\bullet$:
\begin{itemize}
\begin{figure}[H]
    \centering
    \includegraphics[scale = 0.95]{diagrams/lemma1/7.png} 
    \caption{3 dimensional strata of $\mathcal{S}_\bullet$ at $t=0$}
    \label{fig:your-label}
\end{figure}
\begin{figure}[H]
    \centering
    \includegraphics[scale = 0.95]{diagrams/lemma1/8.png} 
    \caption{3 dimensional strata of $\mathcal{S}_\bullet$ at $t=\frac{1}{2}$}
    \label{fig:your-label}
\end{figure}
\begin{figure}[H]
    \centering
    \includegraphics[scale = 0.95]{diagrams/lemma1/9.png} 
    \caption{3 dimensional strata of $\mathcal{S}_\bullet$ at $t=1$}
    \label{fig:your-label}
\end{figure}
\item $3$ dimensional strata: \\
$\{s_\bullet(sgn_2,sgn_1,sgn_3) ~|~ sgn_i \in \{-,+\}\text{ for i=1,2,3}\}$

\begin{figure}[H]
    \centering
    \includegraphics[scale = 0.95]{diagrams/lemma1/10.png} 
    \caption{2 dimensional strata of $\mathcal{S}_\bullet$ at $t=0$}
    \label{fig:your-label}
\end{figure}
\begin{figure}[H]
    \centering
    \includegraphics[scale = 0.95]{diagrams/lemma1/11.png} 
    \caption{2 dimensional strata of $\mathcal{S}_\bullet$ at $t=\frac{1}{2}$}
    \label{fig:your-label}
\end{figure}
\begin{figure}[H]
    \centering
    \includegraphics[scale = 0.95]{diagrams/lemma1/12.png} 
    \caption{2 dimensional strata of $\mathcal{S}_\bullet$ at $t=1$}
    \label{fig:your-label}
\end{figure}
\item $2$ dimensional strata: \\
$\{s_\bullet(0,sgn_2,sgn_3) ~|~ sgn_i \in \{-,+\}\text{ for i=2,3}\}\cup\{s_\bullet(sgn_1,0,sgn_3) ~|~ sgn_i \in \{-,+\}\text{ for i=1,3}\}\cup\{s_\bullet(sgn_1,sgn_2,0) ~|~ sgn_i \in \{-,+\}\text{ for i=1,2}\}$

\begin{figure}[H]
    \centering
    \includegraphics[scale = 0.95]{diagrams/lemma1/13.png} 
    \caption{1 dimensional strata of $\mathcal{S}_\bullet$ at $t=0$}
    \label{fig:your-label}
\end{figure}
\begin{figure}[H]
    \centering
    \includegraphics[scale = 0.95]{diagrams/lemma1/14.png} 
    \caption{1 dimensional strata of $\mathcal{S}_\bullet$ at $t=\frac{1}{2}$}
    \label{fig:your-label}
\end{figure}
\begin{figure}[H]
    \centering
    \includegraphics[scale = 0.95]{diagrams/lemma1/15.png} 
    \caption{1 dimensional strata of $\mathcal{S}_\bullet$ at $t=1$}
    \label{fig:your-label}
\end{figure}
\item $1$ dimensional strata: \\
$\{s_\bullet(sgn_1,0,0) ~|~ sgn_1 \in \{-,+\}\}\cup\{s_\bullet(0,sgn_2,0) ~|~ sgn_2 \in \{-,+\}\}\cup\{s_\bullet(0,0,sgn_3) ~|~ sgn_3 \in \{-,+\}\}$ 

\begin{figure}[H]
    \centering
    \includegraphics[scale = 0.95]{diagrams/lemma1/16.png} 
    \caption{0 dimensional strata of $\mathcal{S}_\bullet$ at $t=0$}
    \label{fig:your-label}
\end{figure}
\begin{figure}[H]
    \centering
    \includegraphics[scale = 0.95]{diagrams/lemma1/17.png} 
    \caption{0 dimensional strata of $\mathcal{S}_\bullet$ at $t=\frac{1}{2}$}
    \label{fig:your-label}
\end{figure}
\begin{figure}[H]
    \centering
    \includegraphics[scale = 0.95]{diagrams/lemma1/18.png} 
    \caption{0 dimensional strata of $\mathcal{S}_\bullet$ at $t=1$}
    \label{fig:your-label}
\end{figure}
\item $0$ dimensional strata: \\
$s_\bullet(0,0,0)$
\end{itemize}
\end{enumerate}
\end{definition}

% definition of the quiver associated to a squiggly diagram
\begin{definition}
Suppose we have a manifold $R$ with stratification $\mathcal{S}$ such that
\begin{itemize}
\item for each codimension $1$ stratum $s\in \mathcal{S}$, there are exactly two regions(= codimension $0$ strata) contained in $star(s)$.

\item each codimension $1$ stratum is equipped with a co-orientation.
\end{itemize}
then we define the quiver associated to $\mathcal{S}$, say $Q_{\mathcal{S}}$, to be a quiver
\begin{itemize}
\item whose vertices corresponds to codimension $0$ strata of $\mathcal{S}$.

\item whose arrows corresponds to codimension $1$ strata of $\mathcal{S}$.

\item the source of an arrow corresponding to $s\in \mathcal{S}$ is  vertex corresponding to the region where the hairs of $s$ are pointing at and the target is the other region contained in the $star(s)$.
\end{itemize}
\end{definition}

% definition of the subquiver associated to a stratum
\begin{definition}
Suppose we have a manifold $R$ with stratification $\mathcal{S}$ such that
\begin{itemize}
\item for each codimension $1$ stratum $s\in \mathcal{S}$, there are exactly two regions(= codimension $0$ strata) contained in $star(s)$.

\item each codimension $1$ stratum is equipped with a co-orientation.
\end{itemize}
then for each $s\in \mathcal{S}$, we define the subquiver of $Q_{\mathcal{S}}$ associated to $s$, say $Q_{\mathcal{S},s}$, to be the full subquiver whose vertices are the ones that corresponds to the regions contained in the start of $s$.
\end{definition}

% definition of legible stratification, quiver
\begin{definition}
Suppose we have a manifold $R$ with stratification $\mathcal{S}$ such that
\begin{itemize}
\item for each codimension $1$ stratum $s\in \mathcal{S}$, there are exactly two regions(= codimension $0$ strata) contained in $star(s)$.

\item each codimension $1$ stratum is equipped with a co-orientation.
\end{itemize}
then $\mathcal{S}$ is a legible stratification if for all $s\in\mathcal{S}$, $Q_{\mathcal{S},s}$ has the initial vertex. We say the quiver $Q_{\mathcal{S}}$ associated to $\mathcal{S}$ is legible if $\mathcal{S}$ is.
\end{definition}

% definition of legible representation
\begin{definition}
Suppose we have a manifold $R$ with stratification $\mathcal{S}$ such that
\begin{itemize}
\item for each codimension $1$ stratum $s\in \mathcal{S}$, there are exactly two regions(= codimension $0$ strata) contained in $star(s)$.

\item each codimension $1$ stratum is equipped with a co-orientation.
\end{itemize}
then we say the quiver representation $F_{\mathcal{S}}$ of $Q_{\mathcal{S}}$ is a legible representation if
\begin{itemize}
\item $\mathcal{S}$ is legible.

\item for any $v,v' \in Vert(Q_{\mathcal{S}})$ and any paths $(a_1,a_2,\cdots,a_k)$,$(a'_1,a'_2,\cdots,a'_{k'})$ from $v$ to $v'$, $F_{\mathcal{S}}(a_k)\circ \cdots F_{\mathcal{S}}(a_1) = F_{\mathcal{S}}(a'_{k'})\circ \cdots F_{\mathcal{S}}(a'_1) $ i.e. the composition is path independent.
\end{itemize}
\end{definition}

% definition of \rho
\begin{definition}
Suppose we have a manifold $R$ with stratification $\mathcal{S}$ such that
\begin{itemize}
\item for each codimension $1$ stratum $s\in \mathcal{S}$, there are exactly two regions(= codimension $0$ strata) contained in $star(s)$.

\item each codimension $1$ stratum is equipped with a co-orientation.
\end{itemize}
Supoose $\mathcal{S}$ is legible, then we define $\rho:\mathcal{S}\rightarrow \{s\in \mathcal{S} ~|~ codim(s)=0 \}$ as
\[
\rho(s):=\text{the codimension $0$ stratum corresponding to the initial vertex of $Q_{\mathcal{S},s}$}
\]
\end{definition}

% definition of the sheaf associated to a legible diagram
\begin{definition}
Suppose we have a manifold $R$ with stratification $\mathcal{S}$ such that
\begin{itemize}
\item for each codimension $1$ stratum $s\in \mathcal{S}$, there are exactly two regions(= codimension $0$ strata) contained in $star(s)$.

\item each codimension $1$ stratum is equipped with a co-orientation.
\end{itemize}
Suppose the quiver representation $F_\mathcal{S}$ of $Q_\mathcal{S}$ is legible, then we define the associated functor $\overline{F_\mathcal{S}}\in Obj(Fun(\mathcal{S}, \C))$ as follows:
\begin{itemize}
\item for $s\in \mathcal{S}$, $\overline{F_\mathcal{S}} := F_\mathcal{S}(\rho(s))$.

\item for $s_1,s_2 \in \mathcal{S}$ where $s_2 \subset start(s_1)$, then $\overline{F_\mathcal{S}}(s_1 \rightarrow s_2)$ is defined as follows: choose a path from the vertex corresponding to $\rho(s_1)$ to $\rho(s_2)$ in $Q_\mathcal{S}$, say $(a_1,\cdots,a_k)$, then 
\[
\overline{F_\mathcal{S}}(s_1 \rightarrow s_2) := F_\mathcal{S}(a_k)\circ\cdots F_\mathcal{S}(a_1)
\] 
This is well-defined because $F_\mathcal{S}$ is legible.
\end{itemize}
\end{definition}

% definition of a co-chain supported in degree 0 and 1 and maps between them
\begin{definition} 
Let $(C^\bullet,\delta^\bullet)$ and $(C'^\bullet,\delta'^\bullet)$ be the cochain complexes $(C^\bullet,\delta^\bullet)$ supported on degree $0$ and $1$ and $\phi^\bullet$ a morphism between $(C^\bullet,\delta^\bullet)$ and $(C'^\bullet,\delta'^\bullet)$, then
\begin{enumerate}
\item we denote $(C^\bullet,\delta^\bullet)$ as either 
\begin{itemize}
\item $C^0 \xrightarrow{\delta^1} C^1$ \\
or  
\item \begin{tikzcd}
C^1 \\
C^0 \arrow[u, "\delta^1"]
\end{tikzcd}
\end{itemize}

\item we denote $\phi^\bullet$ as 
\begin{itemize}
\item \begin{tikzcd}
C^1 \arrow[r, "\phi^1"]     & C'^1  \\
C^0 \arrow[r, "\phi^0"]\arrow[u, "\delta^1"] & C'^0 \arrow[u,"\delta'^1"]
\end{tikzcd}
\end{itemize}
We omit coboundary maps or cochain maps if they are either zero map or identity map and could be inferred from the context.
\end{enumerate}
\end{definition}

%%%%%%%%%%%%%%%%%%%%%%%%%%%%%%%%%%%%%%%%%%%%%%%%%%%%%%%%%%%%%%%%%%%
%%                            Setting                            %%
%%%%%%%%%%%%%%%%%%%%%%%%%%%%%%%%%%%%%%%%%%%%%%%%%%%%%%%%%%%%%%%%%%%
\subsection{Setting}
Suppose on $M$, we have
\begin{itemize}
\item  a squiggly diagram $\Lambda_0$ on $M$

\item nested regions $U' \subset U \subset M$. Note that if we define $V:= M - \overline{U'}$, $\{U,V\}$ form an open cover of $M$.

\item a smooth chart from $D_{r=2}$, say $f: D  \rightarrow U \subset M$
\end{itemize}
such that 
\begin{itemize}
\item $D_{r=1}$ is mapped to $U'$ 

\item $\lambda_0^0$ is mapped to $\Lambda_0^0 |_{U}$

\item $\lambda_0^\infty$ is mapped to $\Lambda_0^\infty |_{U}$

\item $\lambda_0^{squig}$ is mapped to $\Lambda_0^{squig} |_{U}$
\end{itemize}

%%%%%%%%%%%%%%%%%%%%%%%%%%%%%%%%%%%%%%%%%%%%%%%%%%%%%%%%%%%%%%%%%%%
%%                         Initial Sheaf                         %%
%%%%%%%%%%%%%%%%%%%%%%%%%%%%%%%%%%%%%%%%%%%%%%%%%%%%%%%%%%%%%%%%%%%
\subsection{Sheaf at the Beginning}
Suppose we have a sheaf $\mathscr{F}_0$ singular supported on $\Lambda_0$ such that $f^*\mathscr{F}_0$ is isomorphic to the sheaf described by the following squiggly legible diagram $F_0$.

For simplicity, we use the following notations
\[
F_0(sgn_1,sgn_2,sgn_3):= F_0(s_0(sgn_1,sgn_2,sgn_3))
\]

\textbf{Stalks:}
\begin{figure}[H]
    \centering
    \includegraphics[scale = 0.95]{diagrams/lemma1/19.png} 
    \caption{}
    \label{fig:your-label}
\end{figure}
\begin{figure}[H]
    \centering
    \includegraphics[scale = 0.95]{diagrams/lemma1/20.png} 
    \caption{}
    \label{fig:your-label}
\end{figure}
\begin{itemize}
\item $F_0(-,-,-)$ := $\C^m$
\item $F_0(-,-,+)$ := $\C^m$
\item $F_0(+,-,-)$ := $\C^{m+1}$
\item $F_0(+,-,+)$ := $\C^{m+1}$
\item $F_0(-,+,-)$ := $\C^{m+1}$
\item $F_0(-,+,+)$ := $\C^{m+1}$
\end{itemize}

\textbf{Generization maps:}
\begin{figure}[H]
    \centering
    \includegraphics[scale = 0.95]{diagrams/lemma1/21.png} 
    \caption{}
    \label{fig:your-label}
\end{figure}

\begin{enumerate}[label = (\arabic*)]
\item $\C^m \xrightarrow{T(2,2,m+1,m+1)} \C^m$

\item $\C^{m+1} \xrightarrow{T(1,1,m+1,m+1)} \C^{m+1}$

\item $\C^{m+1} \xrightarrow{T(2,2,m+2,m+2)} \C^{m+1}$

\item $\C^{m} \overset{\iota_1}{\rightarrow} \C^{m+1}$

\item $\C^{m} \overset{\iota_1}{\rightarrow} \C^{m+1}$

\item $\C^{m} \overset{\iota_0}{\rightarrow} \C^{m+1}$

\item $\C^{m} \overset{\iota_0}{\rightarrow} \C^{m+1}$
\end{enumerate}

%%%%%%%%%%%%%%%%%%%%%%%%%%%%%%%%%%%%%%%%%%%%%%%%%%%%%%%%%%%%%%%%%%%
%%                   Legendrian Cobordism                        %%
%%%%%%%%%%%%%%%%%%%%%%%%%%%%%%%%%%%%%%%%%%%%%%%%%%%%%%%%%%%%%%%%%%%
\subsection{Legendrian Cobordism}
Then define a Legendrian cobordism $\mathscr{F}_\bullet$ starting from $\mathscr{F}_0$, say $cobord_2$, that is supported on $\overline{U'}$ as follows:\\

By Mayer-Vietoris, this equivalent to the following data
\begin{itemize}
\item a sheaf on $V\times [0,1]$, say $\mathscr{F}_{V\times [0,1]}$

\item a sheaf on $D_{r=2}\times [0,1]$, say $\mathscr{F}_{D_{r=2}\times [0,1]}$

\item a gluing isomorphsim, i.e. $\gamma_\bullet : (f_*\mathscr{F}_{D_{r=2}\times [0,1]})|_{(U\cap V)\times [0,1]} \xrightarrow{\sim} \mathscr{F}_{V\times [0,1]}|_{(U\cap V)\times [0,1]}$.
\end{itemize}
\subsubsection{A. Sheaf on $V\times [0,1]$}
First, I will define $\mathscr{F}_{V\times [0,1]}$ to be $pr_1^*(\mathscr{F}_0|_V)$ where $pr_1 : V \times [0,1] \rightarrow V$ is the projection onto the first argument.
\subsubsection{B. Sheaf on $D_{r=2}\times [0,1]$}
Next, I will describe $\mathscr{F}_{D_{r=2}\times [0,1]}$ as $F_\bullet \in Fun(\mathcal{S}_\bullet,\C)$ i.e. a functor from $\mathcal{S}_\bullet$ to the category of perfect $\C$-modules as follows: 

For simplicity, we use the following notations
\[
F_\bullet(sgn_1,sgn_2,sgn_3):= F_\bullet(s_\bullet(sgn_1,sgn_2,sgn_3))
\]

\textbf{Stalks:}
\begin{figure}[H]
    \centering
    \includegraphics[scale = 0.95]{diagrams/lemma1/22.png} 
    \caption{}
    \label{fig:your-label}
\end{figure}
\begin{figure}[H]
    \centering
    \includegraphics[scale = 0.95]{diagrams/lemma1/23.png} 
    \caption{}
    \label{fig:your-label}
\end{figure}
\begin{figure}[H]
    \centering
    \includegraphics[scale = 0.95]{diagrams/lemma1/24.png} 
    \caption{}
    \label{fig:your-label}
\end{figure}
\begin{figure}[H]
    \centering
    \includegraphics[scale = 0.95]{diagrams/lemma1/25.png} 
    \caption{}
    \label{fig:your-label}
\end{figure}
\begin{figure}[H]
    \centering
    \includegraphics[scale = 0.95]{diagrams/lemma1/26.png} 
    \caption{}
    \label{fig:your-label}
\end{figure}
\begin{figure}[H]
    \centering
    \includegraphics[scale = 0.95]{diagrams/lemma1/27.png} 
    \caption{}
    \label{fig:your-label}
\end{figure}
\begin{itemize}
\item $F_\bullet(-,-,-)$ := $\C^m$
\item $F_\bullet(-,-,+)$ := $\C^m$
\item $F_\bullet(+,-,-)$ := $\C^{m+1}$
\item $F_\bullet(+,-,+)$ := $\C^{m+1}$
\item $F_\bullet(-,+,-)$ := $\C^{m+1}$
\item $F_\bullet(-,+,+)$ := $\C^{m+1}$
\item $F_\bullet(+,+,-)$ := $\C^{m+2}$
\item $F_\bullet(+,+,+)$ := $\C^{m+2}$
\end{itemize}

\textbf{Generization maps:}
\begin{figure}[H]
    \centering
    \includegraphics[scale = 0.95]{diagrams/lemma1/28.png} 
    \caption{}
    \label{fig:your-label}
\end{figure}
\begin{figure}[H]
    \centering
    \includegraphics[scale = 0.95]{diagrams/lemma1/29.png} 
    \caption{}
    \label{fig:your-label}
\end{figure}
\begin{figure}[H]
    \centering
    \includegraphics[scale = 0.95]{diagrams/lemma1/30.png} 
    \caption{}
    \label{fig:your-label}
\end{figure}
\begin{enumerate}[label = (\arabic*)]
\item $\C^m \xrightarrow{T(2,2,m+1,m+1)} \C^m$

\item $\C^{m+1} \xrightarrow{T(1,1,m+1,m+1)} \C^{m+1}$

\item $\C^{m+1} \xrightarrow{T(2,2,m+2,m+2)} \C^{m+1}$

\item $\C^{m} \overset{\iota_1}{\rightarrow} \C^{m+1}$

\item $\C^{m} \overset{\iota_1}{\rightarrow} \C^{m+1}$

\item $\C^{m} \overset{\iota_0}{\rightarrow} \C^{m+1}$

\item $\C^{m} \overset{\iota_0}{\rightarrow} \C^{m+1}$

\item $\C^{m+1} \overset{\iota_0}{\rightarrow} \C^{m+2}$

\item $\C^{m+1} \overset{\iota_1}{\rightarrow} \C^{m+2}$

\item $\C^{m+1} \overset{\iota_0}{\rightarrow} \C^{m+2}$

\item $\C^{m+1} \overset{\iota_1}{\rightarrow} \C^{m+2}$

\item $\C^{m+2} \xrightarrow{T} \C^{m+2}$
\end{enumerate}

\subsubsection{C. Gluing Isomorphism}
Lastly, I will define a gluing isomorphism $\gamma_\bullet : (f_*\mathscr{F}_{D_{r=2}\times [0,1]})|_{(U\cap V)\times [0,1]} \xrightarrow{\sim} \mathscr{F}_{V\times [0,1]}|_{(U\cap V)\times [0,1]}$ using the following fact.
\begin{proposition}
$(f_*\mathscr{F}_{D_{r=2}\times [0,1]})|_{(U\cap V)\times[0,1]}$ is isomorphic to $pr_1^*(\mathscr{F}_0|_{U\cap V})$ where $pr_1 : (U\cap V) \times [0,1] \rightarrow (U\cap V)$ is the projection onto the first argument.
\end{proposition}
\begin{proof}
pass
\end{proof}
\begin{definition}
we define $\gamma_\bullet$ to be the composition 
\[
(f_*\mathscr{F}_{D_{r=2}\times [0,1]})|_{(U\cap V)\times [0,1]}\xrightarrow{\sim}pr_1^*(\mathscr{F}_0|_{U\cap V})\xrightarrow{\sim}pr_1^*(\mathscr{F}_0|_{V})|_{(U\cap V)\times [0,1]}=\mathscr{F}_{V\times [0,1]}|_{(U\cap V)\times [0,1]}
\]
where
\begin{itemize}
\item the first isomorphism is the one mentioned in the above proposition.

\item the second isomorphism from the fact that the following diagram commutes:
\[
\begin{tikzcd}
(U\cap V)\times [0,1] \arrow[hookrightarrow]{r}\arrow[d, "pr_1"]     & V\times [0,1] \arrow[d, "pr_1"] \\
(U\cap V) \arrow[hookrightarrow]{r} & V 
\end{tikzcd}
\]
\end{itemize}
\end{definition}

Now we have defined a cobordism $\mathscr{F}_\bullet$, we show that this is a Legendrian cobordism.
\begin{proposition}
$\mathscr{F}_\bullet$ is a Legendrian cobordism i.e. $\mathscr{F}_\bullet \in Sh_\Lambda(M,\C)$.
\end{proposition}
\begin{proof}
pass
\end{proof}

%%%%%%%%%%%%%%%%%%%%%%%%%%%%%%%%%%%%%%%%%%%%%%%%%%%%%%%%%%%%%%%%%%%
%%                       Terminal Sheaf                          %%
%%%%%%%%%%%%%%%%%%%%%%%%%%%%%%%%%%%%%%%%%%%%%%%%%%%%%%%%%%%%%%%%%%%
\subsection{Sheaf at the End}
In this subsection, I will describe the sheaf $\mathscr{F}_1$ at the end of the $cobord_2$. By Mayer-Vietoris, $\mathscr{F}_1:= \mathscr{F}_\bullet|_{M\times\{1\}}$ on $M \cong M\times\{1\}$ is equivalent to the following data
\begin{itemize}
\item a sheaf on $V$, say $\mathscr{F}_{V}$

\item a sheaf on $D_{r=2}$, say $\mathscr{F}_{D_{r=2}}$

\item a gluing isomorphsim $\gamma_1 : f_*\mathscr{F}_{D_{r=2}}|_{U\cap V} \xrightarrow{\sim} \mathscr{F}_{V}|_{U\cap V}$.
\end{itemize}

\subsubsection{A. Sheaf on $V$}
First, a sheaf on $V\cong V\times\{1\}$ is the restriction of $\mathscr{F}_{V\times [0,1]}$ to $V\times \{1\}$, i.e. $\mathscr{F}_{V\times [0,1]}|_{V\times \{1\}}= pr_1^*(\mathscr{F}_0|_V)|_{V\times \{1\}} = \mathscr{F}_0|_V$.
\subsubsection{B. Sheaf on $D_{r=2}$}
Next, a sheaf on $D_{r=2}\cong D_{r=2}\times \{1\}$ is the restriction of $\mathscr{F}_{D_{r=2}\times [0,1]}$ to $D_{r=2}\times \{1\}$, i.e. $\mathscr{F}_{D_{r=2}\times [0,1]} |_{D_{r=2}\times \{1\}}$. I will describe it as a squiggly legible diagram $F_1$ which is the restriction of $F_\bullet$ defined in the previous section.

For simplicity, we use the following notations
\[
F_1(sgn_1,sgn_2,sgn_3):= F_1(s_1(sgn_1,sgn_2,sgn_3))
\]
\textbf{Stalks:}
\begin{figure}[H]
    \centering
    \includegraphics[scale = 0.95]{diagrams/lemma1/31.png} 
    \caption{}
    \label{fig:your-label}
\end{figure}
\begin{figure}[H]
    \centering
    \includegraphics[scale = 0.95]{diagrams/lemma1/32.png} 
    \caption{}
    \label{fig:your-label}
\end{figure}
\begin{itemize}
\item $F_1(-,-,-)$ := $\C^m$
\item $F_1(-,-,+)$ := $\C^m$
\item $F_1(+,-,-)$ := $\C^{m+1}$
\item $F_1(+,-,+)$ := $\C^{m+1}$
\item $F_1(-,+,-)$ := $\C^{m+1}$
\item $F_1(-,+,+)$ := $\C^{m+1}$
\item $F_1(+,+,-)$ := $\C^{m+2}$
\item $F_1(+,+,+)$ := $\C^{m+2}$
\end{itemize}

\textbf{Generization maps:}
\begin{figure}[H]
    \centering
    \includegraphics[scale = 0.95]{diagrams/lemma1/33.png} 
    \caption{}
    \label{fig:your-label}
\end{figure}
\begin{enumerate}[label = (\arabic*)]
\item $\C^{m+1} \xrightarrow{T(1,1,m+1,m+1)} \C^{m+1}$

\item $\C^{m+1} \xrightarrow{T(2,2,m+2,m+2)} \C^{m+1}$

\item $\C^{m+2} \xrightarrow{T} \C^{m+2}$

\item $\C^{m} \overset{\iota_1}{\rightarrow} \C^{m+1}$

\item $\C^{m} \overset{\iota_0}{\rightarrow} \C^{m+1}$

\item $\C^{m+1} \overset{\iota_0}{\rightarrow} \C^{m+2}$

\item $\C^{m+1} \overset{\iota_1}{\rightarrow} \C^{m+2}$

\item $\C^{m} \overset{\iota_1}{\rightarrow} \C^{m+1}$

\item $\C^{m} \overset{\iota_0}{\rightarrow} \C^{m+1}$

\item $\C^{m+1} \overset{\iota_0}{\rightarrow} \C^{m+2}$

\item $\C^{m+1} \overset{\iota_1}{\rightarrow} \C^{m+2}$
\end{enumerate}

\subsubsection{C. Gluing Isomorphism}
Lastly, the gluing isomorphism $\gamma_1 := \gamma_\bullet|_{(U\cap V)\times \{1\}}:  f_*\mathscr{F}_{D_{r=2}}|_{(U\cap V)\times \{1\}}\xrightarrow{\sim} \mathscr{F}_0|_{U\cap V}$ is described as follows.

\begin{definition}
we define $\gamma_1$ to be the composition
\begin{align*}
&(f_*\mathscr{F}_{D_{r=2}})|_{(U\cap V)\times \{1\}}\xrightarrow{\sim}pr_1^*(\mathscr{F}_0|_{U\cap V})|_{(U\cap V)\times \{1\}}\xrightarrow{\sim}\mathscr{F}_0|_{U\cap V}
\end{align*}
where
\begin{itemize}
\item the first isomorphism follows from the fact that $(f_*\mathscr{F}_{D_{r=2}\times [0,1]})|_{(U\cap V)\times[0,1]}$ is isomorphic to $pr_1^*(\mathscr{F}_0|_{U\cap V})$.

\item the second isomorphism follows from the fact that the following composition is an identity map:
\[
(U\cap V)\xrightarrow{\sim} (U\cap V)\times \{1\} \hookrightarrow (U\cap V)\times [0,1] \twoheadrightarrow (U\cap V)
\]
\end{itemize}
\end{definition}