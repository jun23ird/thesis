\section{2nd sheaf cobordism'}
In this section, we define $cobord'_2$, a compactly supported sheaf cobordism between the following squiggly legible diagrams on the support of the cobordism from
\begin{figure}[H]
    \centering
    \includegraphics[scale = 0.45]{diagrams/cobord'2/29.png} 
    \caption{}
    \label{fig:your-label}
\end{figure}
\textbf{Generization maps:}
\begin{enumerate}[label = (\arabic*)]
\item \begin{tikzcd}
\C \arrow[r]     & 0  \\
0 \arrow[r]\arrow[u] & 0 \arrow[u]
\end{tikzcd}

\item \begin{tikzcd}
\C \arrow[r, "\times 1"]     & \C  \\
0 \arrow[r]\arrow[u] & \C \arrow[u,"\times c"]
\end{tikzcd}

\item \begin{tikzcd}
\C \arrow[r, "\times 1"]     & \C  \\
0 \arrow[r]\arrow[u] & \C \arrow[u,"\times b"]
\end{tikzcd}

\item \begin{tikzcd}
\C \arrow[r]     & 0  \\
0 \arrow[r]\arrow[u] & 0 \arrow[u]
\end{tikzcd}

\item \begin{tikzcd}
0 \arrow[r]     & 0  \\
0 \arrow[r]\arrow[u] & \C \arrow[u]
\end{tikzcd}

\item \begin{tikzcd}
\C \arrow[r,"\times 1"]     & \C  \\
\C \arrow[r,"\iota_1"]\arrow[u,"\times a"] & \C^2 \arrow[u,"(b~a)"]
\end{tikzcd}

\item \begin{tikzcd}
\C \arrow[r]     & 0  \\
\C \arrow[r]\arrow[u,"\times c"] & 0 \arrow[u]
\end{tikzcd}


\item \begin{tikzcd}
\C \arrow[r]     & 0  \\
\C \arrow[r, "\times 1"]\arrow[u,"\times c"] & \C \arrow[u]
\end{tikzcd}

\item \begin{tikzcd}
\C \arrow[r]     & 0  \\
\C \arrow[r]\arrow[u,"\times b"] & 0 \arrow[u]
\end{tikzcd}


\item \begin{tikzcd}
\C \arrow[r,"\times 1"]     & \C  \\
\C \arrow[r,"\iota_0"]\arrow[u,"\times b"] & \C^2 \arrow[u,"(b~a)"]
\end{tikzcd}

\item \begin{tikzcd}
\C \arrow[r,"\times 1"]     & \C  \\
0 \arrow[r]\arrow[u] & \C \arrow[u,"\times a"]
\end{tikzcd}

\item \begin{tikzcd}
\C \arrow[r]     & 0 \\
0 \arrow[r]\arrow[u] & 0 \arrow[u]
\end{tikzcd}

\item \begin{tikzcd}
0 \arrow[r]     & 0 \\
0 \arrow[r]\arrow[u] & 0 \arrow[u]
\end{tikzcd}

\item \begin{tikzcd}
\C \arrow[r]     & 0 \\
\C \arrow[r]\arrow[u,"\times a"] & 0 \arrow[u]
\end{tikzcd}

\item \begin{tikzcd}
\C \arrow[r,"\times 1"]     & \C \\
0 \arrow[r]\arrow[u] & 0 \arrow[u]
\end{tikzcd}
\end{enumerate}
to
\begin{figure}[H]
    \centering
    \includegraphics[scale = 0.45]{diagrams/cobord'2/47.png} 
    \caption{}
    \label{fig:your-label}
\end{figure}
\textbf{Generization maps:}
\begin{enumerate}[label = (\arabic*)]
\item \begin{tikzcd}
0 \arrow[r]     & 0  \\
0 \arrow[r]\arrow[u] & \C \arrow[u]
\end{tikzcd}

\item \begin{tikzcd}
\C \arrow[r,"\times 1"]     & \C  \\
\C \arrow[r,"\iota_1"]\arrow[u,"\times a"] & \C^2 \arrow[u,"(b~a)"]
\end{tikzcd}

\item \begin{tikzcd}
\C \arrow[r, "\times 1"]     & \C  \\
\C \arrow[r, "\times bc^{-1}"]\arrow[u,"\times b"] & \C \arrow[u, "\times c"]
\end{tikzcd}

\item \begin{tikzcd}
\C \arrow[r]     & 0  \\
\C \arrow[r]\arrow[u,"\times c"] & 0 \arrow[u]
\end{tikzcd}

\item \begin{tikzcd}
\C \arrow[r]     & 0  \\
\C \arrow[r]\arrow[u,"\times b"] & 0 \arrow[u]
\end{tikzcd}


\item \begin{tikzcd}
\C \arrow[r]     & 0  \\
\C \arrow[r, "\times 1"]\arrow[u,"\times c"] & \C \arrow[u]
\end{tikzcd}

\item \begin{tikzcd}
\C \arrow[r,"\times 1"]     & \C  \\
\C \arrow[r,"\iota_0"]\arrow[u,"\times b"] & \C^2 \arrow[u,"(b~a)"]
\end{tikzcd}

\item \begin{tikzcd}
\C \arrow[r]     & 0  \\
\C^2 \arrow[r, "(bc^{-1}~0)"]\arrow[u, "(b~a)"] & \C \arrow[u]
\end{tikzcd}

\item \begin{tikzcd}
\C \arrow[r]     & 0  \\
\C \arrow[r]\arrow[u, "\times a"] & 0 \arrow[u]
\end{tikzcd}
\end{enumerate}

\subsection*{Notations}
%definition of a Riemann sphere with two punctures
\begin{definition}
$M$ denotes a Riemann sphere with two punctures at $0$ and $\infty$. $M$ is diffeomorphic to a cylinder.
\end{definition}
%definition of \Phi_t^{symbol}, \Xi_t^{symbol}
\begin{definition}
For $t_0\in\{0,1\}$ and $symbol\in\{0,\infty, squig \}$
\begin{enumerate}
\item we denote $\Phi_{t_0}^{symbol}$ to be smooth maps
\begin{align*}
&\Phi_{t_0}^0 : (S^1)^n \rightarrow M \\
&\Phi_{t_0}^\infty : (S^1)^m \rightarrow M \\
&\Phi_{t_0}^{squig} : [0,1]^{k_{t_0}} \rightarrow M
\end{align*}

\item we denote $\Xi_{t_0}^{symbol}$ a co-orientation of $\Phi_{t_0}^{symbol}$.

\item we denote the pair $(\Phi_{t_0}^{symbol},\Xi_{t_0}^{symbol})$ as $\Lambda_{t_0}^{symbol}$. When $symbol \in \{0,\infty\}$, this could be thought as a front projection of a Legendrian living inside the cocircle bundle of $M$, so we will use $\Lambda_{t_0}^{symbol}$ to denote both

\item we denote the triple $(\Lambda_{t_0}^{0},\Lambda_{t_0}^{\infty},\Lambda_{t_0}^{squig})$ as $\Lambda_{t_0}$ and call it the squiggly diagram at $t_0$. Later in the section, $\Lambda_0$ will be used to denote the squiggly diagram at the beginning of the isotopy underlying $cobord'_2$ and $\Lambda_1$ will be used to denote the squiggly diagram at the end of the isotopy underlying $cobord'_2$. 
\end{enumerate}
\end{definition}

%definition of \Phi_\bullet^{symbol}, \Xi_\bullet^{symbol}
\begin{definition}
For $symbol\in\{0,\infty, squig \}$
\begin{enumerate}
\item we denote $\Phi_\bullet^{symbol}$ to be smooth maps
\begin{align*}
&\Phi_\bullet^0 : (S^1)^n \times [0,1]_t \rightarrow M \times [0,1]_t \\
&\Phi_\bullet^\infty : (S^1)^m \times [0,1]_t \rightarrow M \times [0,1]_t \\
&\Phi_\bullet^{squig} : \coprod_{1\leq i \leq k} ([0,1] \times [a_i,b_i]_t) \rightarrow M \times [0,1]_t
\end{align*}
where the maps are identity maps on the time coordinates. I added auxiliary subscript `$t$' to distinguish the time coordinates from the space coordinates.

\item we denote $\Xi_\bullet^{symbol}$ a co-orientation of $\Phi_\bullet^{symbol}$.

\item we denote the pair $(\Phi_\bullet^{symbol},\Xi_\bullet^{symbol})$ as $\Lambda_\bullet^{symbol}$. Later in the section, $\Lambda_\bullet^{symbol}$ will be used to denote the an isotopy from $\Lambda_0^{symbol}$ to $\Lambda_1^{symbol}$ underlying $cobord'_2$.

\item we denote the triple $(\Lambda_\bullet^{0},\Lambda_\bullet^{\infty},\Lambda_\bullet^{squig})$ as $\Lambda_\bullet$ and call it a squiggly isotopy from $\Lambda_0$ to $\Lambda_1$. Later in the section, $\Lambda_\bullet$ will be used to denote the isotopy between squiggly diagrams starting from $\Lambda_0$ ending at $\Lambda_1$ underlying $cobord'_2$.
\end{enumerate}
\end{definition}

%definition of a bump function
\begin{definition}
For $t \in [0,1]$, we define $\Psi_t: \R \rightarrow \R$ to be a bump function parametrized by $t$ as follows
\[\Psi_t(x)=\bigg\{
\begin{array}{ll}
    \frac{5}{4}e^{(\frac{4x^2}{4x^2 - 3})}(1-t) - \frac{1}{2} & \text{if } |x| < \frac{\sqrt{3}}{2} \\
    -\frac{1}{2} & \text{if } |x| \geq \frac{\sqrt{3}}{2} 
\end{array}
\bigg.
\]
Note that 
\begin{itemize}
\item $supp(\Psi_t) = [-\frac{\sqrt{3}}{2},\frac{\sqrt{3}}{2}]$

\item $\{(-\frac{\sqrt{3}}{2},-\frac{1}{2})$, $(\frac{\sqrt{3}}{2},-\frac{1}{2}),(0,-\frac{5}{4}t + \frac{3}{4})\} \subset Graph(\Psi_t)$
\end{itemize}
\end{definition}

%definition of standard disks
\begin{definition}
We denote the standard open disk in $\R^2$ of radius $r_0$ centered at the origin as 
\[
D_{r=r_0} := \{(x,z)\rightarrow \R^2 ~|~ x^2+z^2 < r_0^2\}
\]
For $t_0 \in [0,1]$, we canonically identify $D_{r=r_0}\times \{t_0\}$ with $D_{r=r_0}$ using the following diffeomorphism
\begin{align*}
& D_{r=r_0} \xrightarrow{\sim} D_{r=r_0} \times \{t_0\} \\
& (x,z) \mapsto (x,z,t_0)
\end{align*}
and with abuse of expression say that sheaves on $D_{r=r_0}\times \{t_0\}$ as sheaves on $D_{r=r_0}$.
\end{definition}

%definition of subsets of D_{r=2}\times \{0\} and their co-orientations
\begin{definition}
\begin{enumerate}
\item We define the following subsets of $D_{r=2} \cong D_{r=2}\times \{0\}$
\begin{itemize}
\item $\lambda_0^0 := \{(x,z) \in D_{r=2} ~|~ z = \Psi_0(x)\}$

\item $\lambda_0^\infty := \{(x,z) \in D_{r=2} ~|~ z = 0 \}$

\item $\lambda_0^{squig}$ is the union of the following three components
\begin{enumerate}[label=(\roman*)]
\item $\{(x,\frac{1}{2}) \in D_{r=2} ~|~ x \leq 0, \frac{1}{2} \geq \Psi_0(x) \}$

\item $\{(x,\frac{1}{2}) \in D_{r=2} ~|~ x \geq 0,\frac{1}{2} \geq \Psi_0(x) \}$

\item $\{(x,z)\in D_{r=2} \mid x=0\}$
\end{enumerate}
\end{itemize}

\item We define co-orientations $\xi_0^{symbol}$ of $\lambda_0^{symbol}$ as follows
\begin{itemize}
\item $\xi_0^0$: hairs are pointing downward direction i.e. coefficients of $dz$ are negative.

\item $\xi_0^\infty$: hairs are pointing upward direction i.e. coefficients of $dz$ are positive.

\item $\xi_0^{squig}$: the hairs of the components $\Rn{1},\Rn{2}$ are pointing downward direction i.e. coefficients of $dz$ are negative and the hairs of the component $\Rn{3}$ are pointing towards the right.
\end{itemize}
\end{enumerate}
\end{definition}

%definition of subsets of D_{r=2}\times \{1\} and their co-orientations
\begin{definition}
\begin{enumerate}
\item We define the following subsets of $D_{r=2} \cong D_{r=2}\times \{1\}$
\begin{itemize}
\item $\lambda_1^0 := \{(x,z) \in D_{r=2} ~|~ z = \Psi_1(x)\} = \{(x,z) \in D_{r=2} ~|~ z = -\frac{1}{2}\}$

\item $\lambda_1^\infty := \{(x,z) \in D_{r=2} ~|~ z = 0 \}$

\item $\lambda_1^{squig}$ is the union of the following three components
\begin{enumerate}[label=(\roman*)]

\item $\{(x,\frac{1}{2}) \in D_{r=2} \}$

\item $\{(x,z) \in D_{r=2} ~|~ x=0\}$
\end{enumerate}
\end{itemize}

\item We define co-orientations $\xi_1^{symbol}$ of $\lambda_1^{symbol}$ as follows
\begin{itemize}
\item $\xi_1^0$: hairs are pointing downward direction i.e. coefficients of $dz$ are negative.

\item $\xi_1^\infty$: hairs are pointing upward direction i.e. coefficients of $dz$ are positive.

\item $\xi_1^{squig}$: 
\begin{itemize}
\item for (\rn{1}), hairs are pointing downward direction i.e. coefficients of $dz$ are negative.
\item for (\rn{2}), hairs are pointing towards the right i.e. coefficients of $dx$ are positive.
\end{itemize}
\end{itemize}
\end{enumerate}
\end{definition}

%definition of subsets of D_{r=2}\times [0,1] and their co-orientations
\begin{definition}
\begin{enumerate}
\item We define the following subsets of $D_{r=2}\times [0,1]$
\begin{itemize}
\item $\lambda_\bullet^0 := \{(x,z,t) \in D_{r=2} \times [0,1] ~|~ z = \Psi_t(x)\}$

\item $\lambda_\bullet^\infty := \{(x,z,t) \in D_{r=2}\times [0,1] ~|~ z = 0 \}$

\item $\lambda_\bullet^{squig}$ is the union of the following two components
\begin{enumerate}[label=(\roman*)]

\item $\{(x,\frac{1}{2},t) \in D_{r=2}\times [0,1] ~|~ \frac{1}{2} > \Psi_t(x) \}$

\item $\{(0,z,t) \in D_{r=2} \times [0,1] ~|~ x=0 \}$
\end{enumerate}
\end{itemize}

\item We define co-orientations $\xi_\bullet^{symbol}$ of $\lambda_\bullet^{symbol}$ as follows
\begin{itemize}
\item $\xi_\bullet^0$: hairs are pointing downward direction i.e. coefficients of $dz$ are negative.

\item $\xi_\bullet^\infty$: hairs are pointing upward direction i.e. coefficients of $dz$ are positive.

\item $\xi_\bullet^{squig}$: 
\begin{itemize}
\item for (\rn{1}), hairs are pointing downward direction i.e. coefficients of $dz$ are negative.

\item for (\rn{2}), hairs are pointing towards the right i.e. coefficients of $dx$ are positive.
\end{itemize}
\end{itemize}
\end{enumerate}
\end{definition}

% stratifications \mathcal{S}_??
\begin{definition}
\begin{enumerate}
\item Consider a stratification $\mathcal{S}_0$ on $D_{r=2}$ induced by $\lambda_0$ i.e. stratification where $0$ dimensional strata are either crossings or end points of squiggly lines, $1$ dimensional strata are sub-arcs of co-oriented links and squiggly lines that are separated by $0$ dimensional strata, and $2$ dimensional strata are exactly the connected components of $M-\lambda_0$. Note that $1$ dimensional strata has co-orientations inherited from $\lambda_0$.

\item Consider a stratification $\mathcal{S}_1$ on $D_{r=2}$ induced by $\lambda_1$ i.e. stratification where $0$ dimensional strata are either crossings or end points of squiggly lines, $1$ dimensional strata are sub-arcs of co-oriented links and squiggly lines that are separated by $0$ dimensional strata, and $2$ dimensional strata are exactly the connected components of $M-\lambda_1$. Note that $1$ dimensional strata has co-orientations inherited from $\lambda_1$.
\end{enumerate}

\item Consider a stratification $\mathcal{S}_\bullet$ on $D_{r=2}\times [0,1]$ induced by $\lambda_\bullet$ i.e. strata are non-empty finite intersections of $\lambda_\bullet^0$, $\lambda_\bullet^\infty$, and $\lambda_\bullet^{squig}$. Note that $2$ dimensional strata has co-orientations inherited from $\lambda_\bullet$.
\end{definition}

Now let's list the strata of $\mathcal{S}_0$, $\mathcal{S}_1$, and $\mathcal{S}_\bullet$ using the following notations:
%definition of sgn function
\begin{definition}
$\operatorname{sgn} : \R \rightarrow \{-,0,+ \}$ is defined as 
\[\operatorname{sgn}(x)=\left\{
\begin{array}{ll}
    + & \text{if } x > 0 \\
    0 & \text{if } x = 0 \\
	- & \text{if } x < 0 \\
\end{array}
\right.
\]
\end{definition}

% +,0,- notations of strata s_{t_0}
\begin{definition}
For $i = 1,2,3,4$ , $t_0 = 0,1$, and $sgn_i \in \{-,0,+\}$,
\begin{enumerate}
\item we define
\begin{align*}
s_{t_0}(sgn_1,sgn_2,sgn_3,sgn_4):=~ &\{(x,z) \in D_{r=2}\cong D_{r=2}\times \{t_0\} ~| \\
&\operatorname{sgn}(z-\Psi_{t_0}(x))=sgn_1,~ \operatorname{sgn}(-z)=sgn_2,\\ 
&\operatorname{sgn}(z-\frac{1}{2})=sgn_3,\\
&\operatorname{sgn}(-x)=sgn_4 \}
\end{align*}

\item we use $*$ as a wild card sign i.e. 
\[
s_{t_0}(\cdots,\underset{\underset{i^{th}}{\uparrow}}{*},\cdots):=~ s_{t_0}(\cdots,\underset{\underset{i^{th}}{\uparrow}}{-},\cdots)\cup s_{t_0}(\cdots,\underset{\underset{i^{th}}{\uparrow}}{0},\cdots)\cup s_{t_0}(\cdots,\underset{\underset{i^{th}}{\uparrow}}{+},\cdots)
\]

\item we omit trailing $*$'s e.g. $s_0(+,-) = s_0(+,-,*,*)$

\item note that we do not oit $*$'s located in between $-,0,+$ e.g. $s_0(+,-,*,-) \neq s_0(+,-,-)$

\end{enumerate}
\end{definition}

% +,0,- notations of strata s_\bullet
\begin{definition}
For $i = 0,1,2,3,4$ and $sgn_i \in \{-,0,+\}$,
\begin{enumerate}
\item we define
\begin{align*}
s_{\bullet}(sgn_1,sgn_2,sgn_3,sgn_4):=~ &\{(x,z,t) \in D_{r=2}\times [0,1] ~| \\
&\operatorname{sgn}(z-\Psi_{t}(x))=sgn_1,~ \operatorname{sgn}(-z)=sgn_2,\\ 
&\operatorname{sgn}(z-\frac{1}{2})=sgn_3,\\
&\operatorname{sgn}(-x)=sgn_4 \}
\end{align*}

\item we use $*$ as a wild card sign i.e. 
\[
s_{\bullet}(\cdots,\underset{\underset{i^{th}}{\uparrow}}{*},\cdots):=~ s_{\bullet}(\cdots,\underset{\underset{i^{th}}{\uparrow}}{-},\cdots)\cup s_{\bullet}(\cdots,\underset{\underset{i^{th}}{\uparrow}}{0},\cdots)\cup s_{\bullet}(\cdots,\underset{\underset{i^{th}}{\uparrow}}{+},\cdots)
\]


\item we omit trailing $*$'s e.g. $s_\bullet(+,-) = s_\bullet(+,-,*,*)$

\item note that we do not oit $*$'s located in between $-,0,+$ e.g. $s_\bullet(+,-,*,-) \neq s_\bullet(+,-,-)$
\end{enumerate}
\end{definition}


\begin{definition}
Now I will describe $\mathcal{S}_0$, $\mathcal{S}_1$, and $\mathcal{S}_\bullet$ using the above notations:
\begin{enumerate}
\item $\mathcal{S}_0$:
\begin{itemize}
\begin{figure}[H]
    \centering
    \includegraphics[scale = 0.45]{diagrams/cobord'2/1.png} 
    \caption{}
    \label{fig:your-label}
\end{figure}
\item $2$ dimensional strata: \\
$s_0(-,-,*,-)$,$s_0(-,-,*,+)$, $s_0(+,-,-,-)$, $s_0(+,-,-,+)$, $s_0(+,-,+,-)$,$s_0(+,-,+,+)$, $s_0(+,+,*,-)$, $s_0(+,+,*,-)$,$s_0(-,+,*,-)$,$s_0(-,+,*,+)$

\begin{figure}[H]
    \centering
    \includegraphics[scale = 0.45]{diagrams/cobord'2/2.png} 
    \caption{}
    \label{fig:your-label}
\end{figure}
\item $1$ dimensional strata: \\
$s_0(0,+,*,-)$, $s_0(0,-,-,-)$, $s_0(0,-,+,-)$, $s_0(0,-,+,+)$, $s_0(0,-,-,+)$, $s_0(0,+,*,+)$, $s_0(+,0,*,-)$, $s_0(-,0,*,-)$, $s_0(-,0,*,+)$, $s_0(+,0,*,+)$, $s_0(+,*,0,-)$, $s_0(+,*,0,+)$, $s_0(+,-,+,0)$,
$s_0(-,-,*,0)$, $s_0(-,+,-,0)$
\begin{figure}[H]
    \centering
    \includegraphics[scale = 0.45]{diagrams/cobord'2/3.png} 
    \caption{}
    \label{fig:your-label}
\end{figure}
\item $0$ dimensional strata: \\
$s_0(0,0,-,-)$, $s_0(0,-,0,-)$, $s_0(0,-,0,+)$, $s_0(0,0,-,+)$, $s_0(0,-,+,-)$, $s_0(-,0,-,0)$
\end{itemize}

\item $\mathcal{S}_1$:
\begin{itemize}
\begin{figure}[H]
    \centering
    \includegraphics[scale = 0.45]{diagrams/cobord'2/4.png} 
    \caption{}
    \label{fig:your-label}
\end{figure}
\item $2$ dimensional strata: \\
$s_1(-,+,*,-)$, $s_1(-,+,*,+)$, $s_1(+,-,-,-)$, $s_1(+,-,-,+)$, $s_1(+,-,+,-)$, $s_1(+,-,+,+)$, $s_1(+,+,*,-)$, $s_1(+,+,*,+)$

\begin{figure}[H]
    \centering
    \includegraphics[scale = 0.45]{diagrams/cobord'2/5.png} 
    \caption{}
    \label{fig:your-label}
\end{figure}
\item $1$ dimensional strata: \\
$s_1(0,+,*,-)$, $s_1(0,+,*,+)$, $s_1(+,0,*,-)$, $s_1(+,0,*,+)$, $s_1(+,*,0,-)$, $s_1(+,*,0,+)$, $s_1(+,-,-,0)$, $s_1(+,+,*,0)$, $s_1(+,-,+,0)$, $s_1(-,+,*,0)$

\begin{figure}[H]
    \centering
    \includegraphics[scale = 0.45]{diagrams/cobord'2/6.png} 
    \caption{}
    \label{fig:your-label}
\end{figure}
\item $0$ dimensional strata: \\
$s_1(+,-,0,0)$, $s_1(-,0,-,0)$, $s_1(0,+,-,0)$
\end{itemize}

\item $\mathcal{S}_\bullet$:
\begin{itemize}
\begin{figure}[H]
    \centering
    \includegraphics[scale = 0.45]{diagrams/cobord'2/7.png} 
    \caption{}
    \label{fig:your-label}
\end{figure}
\begin{figure}[H]
    \centering
    \includegraphics[scale = 0.45]{diagrams/cobord'2/8.png} 
    \caption{}
    \label{fig:your-label}
\end{figure}
\begin{figure}[H]
    \centering
    \includegraphics[scale = 0.45]{diagrams/cobord'2/9.png} 
    \caption{}
    \label{fig:your-label}
\end{figure}
\begin{figure}[H]
    \centering
    \includegraphics[scale = 0.45]{diagrams/cobord'2/10.png} 
    \caption{}
    \label{fig:your-label}
\end{figure}
\begin{figure}[H]
    \centering
    \includegraphics[scale = 0.45]{diagrams/cobord'2/11.png} 
    \caption{}
    \label{fig:your-label}
\end{figure}
\item $3$ dimensional strata: \\
$s_\bullet(-,+,*,-)$, $s_\bullet(-,+,*,+)$, $s_\bullet(-,-,*,-)$, $s_\bullet(-,-,*,+)$, $s_\bullet(-,+)$, $s_\bullet(+,-,-,-)$, $s_\bullet(+,-,-,+)$, $s_\bullet(+,-,+,-)$, $s_\bullet(+,-,+,+)$, $s_\bullet(+,+,*,-)$, $s_\bullet(+,+,*,+)$

\begin{figure}[H]
    \centering
    \includegraphics[scale = 0.45]{diagrams/cobord'2/12.png} 
    \caption{}
    \label{fig:your-label}
\end{figure}
\begin{figure}[H]
    \centering
    \includegraphics[scale = 0.45]{diagrams/cobord'2/13.png} 
    \caption{}
    \label{fig:your-label}
\end{figure}
\begin{figure}[H]
    \centering
    \includegraphics[scale = 0.45]{diagrams/cobord'2/14.png} 
    \caption{}
    \label{fig:your-label}
\end{figure}
\begin{figure}[H]
    \centering
    \includegraphics[scale = 0.45]{diagrams/cobord'2/15.png} 
    \caption{}
    \label{fig:your-label}
\end{figure}
\begin{figure}[H]
    \centering
    \includegraphics[scale = 0.45]{diagrams/cobord'2/16.png} 
    \caption{}
    \label{fig:your-label}
\end{figure}
\item $2$ dimensional strata: \\
$s_\bullet(0,+,*,-)$, $s_\bullet(0,-,-,-)$, $s_\bullet(0,-,+)$, $s_\bullet(0,-,+,-)$, $s_\bullet(0,-,+,+)$, $s_\bullet(0,-,-,+)$, $s_\bullet(0,+,*,+)$, $s_\bullet(+,0,*,-)$, $s_\bullet(-,0,*,-)$, $s_\bullet(-,0,*,+)$, $s_\bullet(+,0,*,+)$, $s_\bullet(+,*,0,-)$, $s_\bullet(+,*,0,+)$, $s_\bullet(+,-,-,0)$, $s_\bullet(+,+,*,0)$, $s_\bullet(+,-,+,0)$, $s_\bullet(-,+,*,0)$, $s_\bullet(-,-,*,0)$

\begin{figure}[H]
    \centering
    \includegraphics[scale = 0.45]{diagrams/cobord'2/17.png} 
    \caption{}
    \label{fig:your-label}
\end{figure}
\begin{figure}[H]
    \centering
    \includegraphics[scale = 0.45]{diagrams/cobord'2/18.png} 
    \caption{}
    \label{fig:your-label}
\end{figure}
\begin{figure}[H]
    \centering
    \includegraphics[scale = 0.45]{diagrams/cobord'2/19.png} 
    \caption{}
    \label{fig:your-label}
\end{figure}
\begin{figure}[H]
    \centering
    \includegraphics[scale = 0.45]{diagrams/cobord'2/20.png} 
    \caption{}
    \label{fig:your-label}
\end{figure}
\begin{figure}[H]
    \centering
    \includegraphics[scale = 0.45]{diagrams/cobord'2/21.png} 
    \caption{}
    \label{fig:your-label}
\end{figure}
\item $1$ dimensional strata: \\
$s_\bullet(0,0,-,-)$, $s_\bullet(0,-,0,-)$, $s_\bullet(0,-,0,+)$, $s_\bullet(0,0,-,+)$, $s_\bullet(+,-,0,0)$, $s_\bullet(0,-,-,0)$, $s_\bullet(-,0,-,0)$, $s_\bullet(0,+,-,0)$, $s_\bullet(0,-,+,0)$, $s_\bullet(-,0,*,0)$

\begin{figure}[H]
    \centering
    \includegraphics[scale = 0.45]{diagrams/cobord'2/22.png} 
    \caption{}
    \label{fig:your-label}
\end{figure}
\begin{figure}[H]
    \centering
    \includegraphics[scale = 0.45]{diagrams/cobord'2/23.png} 
    \caption{}
    \label{fig:your-label}
\end{figure}
\begin{figure}[H]
    \centering
    \includegraphics[scale = 0.45]{diagrams/cobord'2/24.png} 
    \caption{}
    \label{fig:your-label}
\end{figure}
\begin{figure}[H]
    \centering
    \includegraphics[scale = 0.45]{diagrams/cobord'2/25.png} 
    \caption{}
    \label{fig:your-label}
\end{figure}
\begin{figure}[H]
    \centering
    \includegraphics[scale = 0.45]{diagrams/cobord'2/26.png} 
    \caption{}
    \label{fig:your-label}
\end{figure}
\item $0$ dimensional strata: \\
$s_\bullet(0,-,0,0)$, $s_\bullet(0,0,-,0)$
\end{itemize}
\end{enumerate}
\end{definition}

% definition of the quiver associated to a squiggly diagram
\begin{definition}
Suppose we have a manifold $R$ with stratification $\mathcal{S}$ such that
\begin{itemize}
\item for each codimension $1$ stratum $s\in \mathcal{S}$, there are exactly two regions(= codimension $0$ strata) contained in $\operatorname{star}(s)$.

\item each codimension $1$ stratum is equipped with a co-orientation.
\end{itemize}
then we define the quiver associated to $\mathcal{S}$, say $Q_{\mathcal{S}}$, to be a quiver
\begin{itemize}
\item whose vertices corresponds to codimension $0$ strata of $\mathcal{S}$.

\item whose arrows corresponds to codimension $1$ strata of $\mathcal{S}$.

\item the source of an arrow corresponding to $s\in \mathcal{S}$ is  vertex corresponding to the region where the hairs of $s$ are pointing at and the target is the other region contained in the $\operatorname{star}(s)$.
\end{itemize}
\end{definition}

% definition of the subquiver associated to a stratum
\begin{definition}
Suppose we have a manifold $R$ with stratification $\mathcal{S}$ such that
\begin{itemize}
\item for each codimension $1$ stratum $s\in \mathcal{S}$, there are exactly two regions(= codimension $0$ strata) contained in $\operatorname{star}(s)$.

\item each codimension $1$ stratum is equipped with a co-orientation.
\end{itemize}
then for each $s\in \mathcal{S}$, we define the subquiver of $Q_{\mathcal{S}}$ associated to $s$, say $Q_{\mathcal{S},s}$, to be the full subquiver whose vertices are the ones that corresponds to the regions contained in the star of $s$.
\end{definition}

% definition of legible stratification, quiver
\begin{definition}
Suppose we have a manifold $R$ with stratification $\mathcal{S}$ such that
\begin{itemize}
\item for each codimension $1$ stratum $s\in \mathcal{S}$, there are exactly two regions(= codimension $0$ strata) contained in $\operatorname{star}(s)$.

\item each codimension $1$ stratum is equipped with a co-orientation.
\end{itemize}
then $\mathcal{S}$ is a legible stratification if for all $s\in\mathcal{S}$, $Q_{\mathcal{S},s}$ has the initial vertex. We say the quiver $Q_{\mathcal{S}}$ associated to $\mathcal{S}$ is legible if $\mathcal{S}$ is.
\end{definition}

% definition of legible representation
\begin{definition}
Suppose we have a manifold $R$ with stratification $\mathcal{S}$ such that
\begin{itemize}
\item for each codimension $1$ stratum $s\in \mathcal{S}$, there are exactly two regions(= codimension $0$ strata) contained in $\operatorname{star}(s)$.

\item each codimension $1$ stratum is equipped with a co-orientation.
\end{itemize}
then we say the quiver representation $F_{\mathcal{S}}$ of $Q_{\mathcal{S}}$ is a legible representation if
\begin{itemize}
\item $\mathcal{S}$ is legible.

\item for any $v,v' \in Vert(Q_{\mathcal{S}})$ and any paths $(a_1,a_2,\cdots,a_k)$,$(a'_1,a'_2,\cdots,a'_{k'})$ from $v$ to $v'$, $F_{\mathcal{S}}(a_k)\circ \cdots F_{\mathcal{S}}(a_1) = F_{\mathcal{S}}(a'_{k'})\circ \cdots F_{\mathcal{S}}(a'_1) $ i.e. the composition is path independent.
\end{itemize}
\end{definition}

% definition of \rho
\begin{definition}
Suppose we have a manifold $R$ with stratification $\mathcal{S}$ such that
\begin{itemize}
\item for each codimension $1$ stratum $s\in \mathcal{S}$, there are exactly two regions(= codimension $0$ strata) contained in $\operatorname{star}(s)$.

\item each codimension $1$ stratum is equipped with a co-orientation.
\end{itemize}
Supoose $\mathcal{S}$ is legible, then we define $\rho:\mathcal{S}\rightarrow \{s\in \mathcal{S} ~|~ \operatorname{codim}(s)=0 \}$ as
\[
\rho(s):=\text{the codimension $0$ stratum corresponding to the initial vertex of $Q_{\mathcal{S},s}$}
\]
\end{definition}

% definition of the sheaf associated to a legible diagram
\begin{definition}
Suppose we have a manifold $R$ with stratification $\mathcal{S}$ such that
\begin{itemize}
\item for each codimension $1$ stratum $s\in \mathcal{S}$, there are exactly two regions(= codimension $0$ strata) contained in $\operatorname{star}(s)$.

\item each codimension $1$ stratum is equipped with a co-orientation.
\end{itemize}
Suppose the quiver representation $F_\mathcal{S}$ of $Q_\mathcal{S}$ is legible, then we define the associated functor $\overline{F_\mathcal{S}}\in Obj(Fun(\mathcal{S}, \C))$ as follows:
\begin{itemize}
\item for $s\in \mathcal{S}$, $\overline{F_\mathcal{S}} := F_\mathcal{S}(\rho(s))$.

\item for $s_1,s_2 \in \mathcal{S}$ where $s_2 \subset \operatorname{star}(s_1)$, then $\overline{F_\mathcal{S}}(s_1 \rightarrow s_2)$ is defined as follows: choose a path from the vertex corresponding to $\rho(s_1)$ to $\rho(s_2)$ in $Q_\mathcal{S}$, say $(a_1,\cdots,a_k)$, then 
\[
\overline{F_\mathcal{S}}(s_1 \rightarrow s_2) := F_\mathcal{S}(a_k)\circ\cdots F_\mathcal{S}(a_1)
\] 
This is well-defined because $F_\mathcal{S}$ is legible.
\end{itemize}
\end{definition}

%%%%%%%%%%%%%%%%%%%%%%%%%%%%%%%%%%%%%%%%%%%%%%%%%%%%%%%%%%%%%%%%%%%
%%                            Setting                            %%
%%%%%%%%%%%%%%%%%%%%%%%%%%%%%%%%%%%%%%%%%%%%%%%%%%%%%%%%%%%%%%%%%%%
\subsection*{Setting}
Suppose on $M$, we have
\begin{itemize}
\item  a squiggly diagram $\Lambda_0$ on $M$

\item nested regions $U' \subset U \subset M$. Note that if we define $V:= M - \overline{U'}$, $\{U,V\}$ form an open cover of $M$.

\item a smooth chart from $D_{r=2}$, say $f: D  \rightarrow U \subset M$
\end{itemize}
such that 
\begin{itemize}
\item $D_{r=1}$ is mapped to $U'$ 

\item $\lambda_0^0$ is mapped to $\Lambda_0^0 |_{U}$

\item $\lambda_0^\infty$ is mapped to $\Lambda_0^\infty |_{U}$

\item $\lambda_0^{squig}$ is mapped to $\Lambda_0^{squig} |_{U}$
\end{itemize}

%%%%%%%%%%%%%%%%%%%%%%%%%%%%%%%%%%%%%%%%%%%%%%%%%%%%%%%%%%%%%%%%%%%
%%                         Initial Sheaf                         %%
%%%%%%%%%%%%%%%%%%%%%%%%%%%%%%%%%%%%%%%%%%%%%%%%%%%%%%%%%%%%%%%%%%%
\subsection*{Sheaf at the Beginning}
Suppose we have a sheaf $\mathscr{F}_0$ singular supported on $\Lambda_0$ such that $f^*\mathscr{F}_0$ is isomorphic to the sheaf described by the following squiggly legible diagram $F_0$.

For simplicity, we use the following notations
\[
F_0(sgn_1,sgn_2,sgn_3,sgn_4):=F_0(s_0(sgn_1,sgn_2,sgn_3,sgn_4))
\]
\textbf{Stalks:}
\begin{figure}[H]
    \centering
    \includegraphics[scale = 0.45]{diagrams/cobord'2/27.png} 
    \caption{}
    \label{fig:your-label}
\end{figure}
\begin{figure}[H]
    \centering
    \includegraphics[scale = 0.45]{diagrams/cobord'2/28.png} 
    \caption{}
    \label{fig:your-label}
\end{figure}
\begin{itemize}
\item $F_0(-,-,*,-)$ := $\C[-1]$
\item $F_0(-,-,*,+)$ := $\C[-1]$
\item $F_0(-,+,*,-)$ := $\C \xrightarrow{\times a}\C$
\item $F_0(-,+,*,+)$ := $0$
\item $F_0(+,-,-,-)$ := $\C \xrightarrow{\times b} \C $
\item $F_0(+,-,-,+)$ := $\C \xrightarrow{\times c} \C $
\item $F_0(+,-,+,-)$ := $0$
\item $F_0(+,-,+,+)$ := $0$
\item $F_0(+,+,*,-)$ := $\C^2 \xrightarrow{(b~a)}\C$
\item $F_0(+,+,*,+)$ := $\C$
\end{itemize}

\textbf{Generization maps:}
\begin{figure}[H]
    \centering
    \includegraphics[scale = 0.45]{diagrams/cobord'2/29.png} 
    \caption{}
    \label{fig:your-label}
\end{figure}
\begin{enumerate}[label = (\arabic*)]
\item \begin{tikzcd}
\C \arrow[r]     & 0  \\
0 \arrow[r]\arrow[u] & 0 \arrow[u]
\end{tikzcd}

\item \begin{tikzcd}
\C \arrow[r, "\times 1"]     & \C  \\
0 \arrow[r]\arrow[u] & \C \arrow[u,"\times c"]
\end{tikzcd}

\item \begin{tikzcd}
\C \arrow[r, "\times 1"]     & \C  \\
0 \arrow[r]\arrow[u] & \C \arrow[u,"\times b"]
\end{tikzcd}

\item \begin{tikzcd}
\C \arrow[r]     & 0  \\
0 \arrow[r]\arrow[u] & 0 \arrow[u]
\end{tikzcd}

\item \begin{tikzcd}
0 \arrow[r]     & 0  \\
0 \arrow[r]\arrow[u] & \C \arrow[u]
\end{tikzcd}

\item \begin{tikzcd}
\C \arrow[r,"\times 1"]     & \C  \\
\C \arrow[r,"\iota_1"]\arrow[u,"\times a"] & \C^2 \arrow[u,"(b~a)"]
\end{tikzcd}

\item \begin{tikzcd}
\C \arrow[r]     & 0  \\
\C \arrow[r]\arrow[u,"\times c"] & 0 \arrow[u]
\end{tikzcd}


\item \begin{tikzcd}
\C \arrow[r]     & 0  \\
\C \arrow[r, "\times 1"]\arrow[u,"\times c"] & \C \arrow[u]
\end{tikzcd}

\item \begin{tikzcd}
\C \arrow[r]     & 0  \\
\C \arrow[r]\arrow[u,"\times b"] & 0 \arrow[u]
\end{tikzcd}


\item \begin{tikzcd}
\C \arrow[r,"\times 1"]     & \C  \\
\C \arrow[r,"\iota_0"]\arrow[u,"\times b"] & \C^2 \arrow[u,"(b~a)"]
\end{tikzcd}

\item \begin{tikzcd}
\C \arrow[r,"\times 1"]     & \C  \\
0 \arrow[r]\arrow[u] & \C \arrow[u,"\times a"]
\end{tikzcd}

\item \begin{tikzcd}
\C \arrow[r]     & 0 \\
0 \arrow[r]\arrow[u] & 0 \arrow[u]
\end{tikzcd}

\item \begin{tikzcd}
0 \arrow[r]     & 0 \\
0 \arrow[r]\arrow[u] & 0 \arrow[u]
\end{tikzcd}

\item \begin{tikzcd}
\C \arrow[r]     & 0 \\
\C \arrow[r]\arrow[u,"\times a"] & 0 \arrow[u]
\end{tikzcd}

\item \begin{tikzcd}
\C \arrow[r,"\times 1"]     & \C \\
0 \arrow[r]\arrow[u] & 0 \arrow[u]
\end{tikzcd}
\end{enumerate}

%%%%%%%%%%%%%%%%%%%%%%%%%%%%%%%%%%%%%%%%%%%%%%%%%%%%%%%%%%%%%%%%%%%
%%                   Legendrian Cobordism                        %%
%%%%%%%%%%%%%%%%%%%%%%%%%%%%%%%%%%%%%%%%%%%%%%%%%%%%%%%%%%%%%%%%%%%
\subsection*{Legendrian Cobordism}
Then define a Legendrian cobordism $\mathscr{F}_\bullet$ starting from $\mathscr{F}_0$, say $cobord'_2$, that is supported on $\overline{U'}$ as follows:\\

By Mayer-Vietoris, this equivalent to the following data
\begin{itemize}
\item a sheaf on $V\times [0,1]$, say $\mathscr{F}_{V\times [0,1]}$

\item a sheaf on $D_{r=2}\times [0,1]$, say $\mathscr{F}_{D_{r=2}\times [0,1]}$

\item a gluing isomorphsim, i.e. $\gamma_\bullet : (f_*\mathscr{F}_{D_{r=2}\times [0,1]})|_{(U\cap V)\times [0,1]} \xrightarrow{\sim} \mathscr{F}_{V\times [0,1]}|_{(U\cap V)\times [0,1]}$.
\end{itemize}
\subsubsection{A. Sheaf on $V\times [0,1]$}
First, I will define $\mathscr{F}_{V\times [0,1]}$ to be $pr_1^*(\mathscr{F}_0|_V)$ where $pr_1 : V \times [0,1] \rightarrow V$ is the projection onto the first argument.
\subsubsection{B. Sheaf on $D_{r=2}\times [0,1]$}
Next, I will describe $\mathscr{F}_{D_{r=2}\times [0,1]}$ as $F_\bullet \in Fun(\mathcal{S}_\bullet,\C)$ i.e. a functor from $\mathcal{S}_\bullet$ to the category of perfect $\C$-modules as follows: 

For simplicity, we use the following notations
\[
F_\bullet(sgn_1,sgn_2,sgn_3,sgn_4):=F_\bullet(s_\bullet(sgn_1,sgn_2,sgn_3,sgn_4))
\]
\textbf{Stalks:}
\begin{figure}[H]
    \centering
    \includegraphics[scale = 0.45]{diagrams/cobord'2/30.png} 
    \caption{}
    \label{fig:your-label}
\end{figure}
\begin{figure}[H]
    \centering
    \includegraphics[scale = 0.45]{diagrams/cobord'2/31.png} 
    \caption{}
    \label{fig:your-label}
\end{figure}
\begin{figure}[H]
    \centering
    \includegraphics[scale = 0.45]{diagrams/cobord'2/32.png} 
    \caption{}
    \label{fig:your-label}
\end{figure}
\begin{figure}[H]
    \centering
    \includegraphics[scale = 0.45]{diagrams/cobord'2/33.png} 
    \caption{}
    \label{fig:your-label}
\end{figure}
\begin{figure}[H]
    \centering
    \includegraphics[scale = 0.45]{diagrams/cobord'2/34.png} 
    \caption{}
    \label{fig:your-label}
\end{figure}
\begin{figure}[H]
    \centering
    \includegraphics[scale = 0.45]{diagrams/cobord'2/35.png} 
    \caption{}
    \label{fig:your-label}
\end{figure}
\begin{figure}[H]
    \centering
    \includegraphics[scale = 0.45]{diagrams/cobord'2/36.png} 
    \caption{}
    \label{fig:your-label}
\end{figure}
\begin{figure}[H]
    \centering
    \includegraphics[scale = 0.45]{diagrams/cobord'2/37.png} 
    \caption{}
    \label{fig:your-label}
\end{figure}
\begin{figure}[H]
    \centering
    \includegraphics[scale = 0.45]{diagrams/cobord'2/38.png} 
    \caption{}
    \label{fig:your-label}
\end{figure}
\begin{figure}[H]
    \centering
    \includegraphics[scale = 0.45]{diagrams/cobord'2/39.png} 
    \caption{}
    \label{fig:your-label}
\end{figure}
\begin{itemize}
\item $F_\bullet(-,-,*,-)$ := $\C[-1]$
\item $F_\bullet(-,-,*,+)$ := $\C[-1]$
\item $F_\bullet(-,+,*,-)$ := $\C\xrightarrow{\times a}\C$
\item $F_\bullet(-,+,*,+)$ := $0$
\item $F_\bullet(+,-,-,-)$ := $\C \xrightarrow{\times b} \C $
\item $F_\bullet(+,-,-,+)$ := $\C \xrightarrow{\times c} \C $
\item $F_\bullet(+,-,+,-)$ := $0$
\item $F_\bullet(+,-,+,+)$ := $0$
\item $F_\bullet(+,+,*,-)$ := $\C^2\xrightarrow{(b~a)}\C$
\item $F_\bullet(+,+,*,+)$ := $\C$
\end{itemize}

\textbf{Generization maps:}
\begin{figure}[H]
    \centering
    \includegraphics[scale = 0.45]{diagrams/cobord'2/40.png} 
    \caption{}
    \label{fig:your-label}
\end{figure}
\begin{figure}[H]
    \centering
    \includegraphics[scale = 0.45]{diagrams/cobord'2/41.png} 
    \caption{}
    \label{fig:your-label}
\end{figure}
\begin{figure}[H]
    \centering
    \includegraphics[scale = 0.45]{diagrams/cobord'2/42.png} 
    \caption{}
    \label{fig:your-label}
\end{figure}
\begin{figure}[H]
    \centering
    \includegraphics[scale = 0.45]{diagrams/cobord'2/43.png} 
    \caption{}
    \label{fig:your-label}
\end{figure}
\begin{figure}[H]
    \centering
    \includegraphics[scale = 0.45]{diagrams/cobord'2/44.png} 
    \caption{}
    \label{fig:your-label}
\end{figure}
\begin{enumerate}[label = (\arabic*)]
\item \begin{tikzcd}
\C \arrow[r]     & 0  \\
0 \arrow[r]\arrow[u] & 0 \arrow[u]
\end{tikzcd}

\item \begin{tikzcd}
\C \arrow[r, "\times 1"]     & \C  \\
0 \arrow[r]\arrow[u] & \C \arrow[u,"\times c"]
\end{tikzcd}

\item \begin{tikzcd}
\C \arrow[r, "\times 1"]     & \C  \\
0 \arrow[r]\arrow[u] & \C \arrow[u,"\times b"]
\end{tikzcd}

\item \begin{tikzcd}
\C \arrow[r]     & 0  \\
0 \arrow[r]\arrow[u] & 0 \arrow[u]
\end{tikzcd}

\item \begin{tikzcd}
0 \arrow[r]     & 0  \\
0 \arrow[r]\arrow[u] & \C \arrow[u]
\end{tikzcd}

\item \begin{tikzcd}
\C \arrow[r,"\times 1"]     & \C  \\
\C \arrow[r,"\iota_1"]\arrow[u,"\times a"] & \C^2 \arrow[u,"(b~a)"]
\end{tikzcd}

\item \begin{tikzcd}
\C \arrow[r]     & 0  \\
\C \arrow[r]\arrow[u,"\times c"] & 0 \arrow[u]
\end{tikzcd}


\item \begin{tikzcd}
\C \arrow[r]     & 0  \\
\C \arrow[r, "\times 1"]\arrow[u,"\times c"] & \C \arrow[u]
\end{tikzcd}

\item \begin{tikzcd}
\C \arrow[r]     & 0  \\
\C \arrow[r]\arrow[u,"\times b"] & 0 \arrow[u]
\end{tikzcd}


\item \begin{tikzcd}
\C \arrow[r,"\times 1"]     & \C  \\
\C \arrow[r,"\iota_0"]\arrow[u,"\times b"] & \C^2 \arrow[u,"(b~a)"]
\end{tikzcd}

\item \begin{tikzcd}
\C \arrow[r, "\times 1"]     & \C  \\
0 \arrow[r]\arrow[u] & \C \arrow[u, "\times a"]
\end{tikzcd}

\item \begin{tikzcd}
\C \arrow[r]     & 0  \\
0 \arrow[r]\arrow[u] & 0 \arrow[u]
\end{tikzcd}

\item \begin{tikzcd}
0 \arrow[r]     & 0  \\
0 \arrow[r]\arrow[u] & 0 \arrow[u]
\end{tikzcd}

\item \begin{tikzcd}
\C \arrow[r]     & 0  \\
\C \arrow[r,"\times a"]\arrow[u] & 0 \arrow[u]
\end{tikzcd}

\item \begin{tikzcd}
\C \arrow[r,"\times 1"]     & \C  \\
0 \arrow[r]\arrow[u] & 0 \arrow[u]
\end{tikzcd}

\item \begin{tikzcd}
\C \arrow[r,"\times 1"]     & \C  \\
\C \arrow[r,"\times bc^{-1}"]\arrow[u,"\times b"] & 0 \arrow[u,"\times c"]
\end{tikzcd}

\item \begin{tikzcd}
\C \arrow[r]     & 0  \\
\C \arrow[r,"(bc^{-1}~0)"]\arrow[u,"(b~a)"] & \C \arrow[u]
\end{tikzcd}
\end{enumerate}

\subsubsection{C. Gluing Isomorphism}
Lastly, I will define a gluing isomorphism $\gamma_\bullet : (f_*\mathscr{F}_{D_{r=2}\times [0,1]})|_{(U\cap V)\times [0,1]} \xrightarrow{\sim} \mathscr{F}_{V\times [0,1]}|_{(U\cap V)\times [0,1]}$ using the fact that $(f_*\mathscr{F}_{D_{r=2}\times [0,1]})|_{(U\cap V)\times[0,1]}$ is isomorphic to $pr_1^*(\mathscr{F}_0|_{U\cap V})$ where $pr_1 : (U\cap V) \times [0,1] \rightarrow (U\cap V)$ is the projection onto the first argument.
\begin{definition}
we define $\gamma_\bullet$ to be the composition 
\[
(f_*\mathscr{F}_{D_{r=2}\times [0,1]})|_{(U\cap V)\times [0,1]}\xrightarrow{\sim}pr_1^*(\mathscr{F}_0|_{U\cap V})\xrightarrow{\sim}pr_1^*(\mathscr{F}_0|_{V})|_{(U\cap V)\times [0,1]}=\mathscr{F}_{V\times [0,1]}|_{(U\cap V)\times [0,1]}
\]
where
\begin{itemize}
\item the first isomorphism is the one mentioned in the above proposition.

\item the second isomorphism from the fact that the following diagram commutes:
\[
\begin{tikzcd}
(U\cap V)\times [0,1] \arrow[hookrightarrow]{r}\arrow[d, "pr_1"]     & V\times [0,1] \arrow[d, "pr_1"] \\
(U\cap V) \arrow[hookrightarrow]{r} & V 
\end{tikzcd}
\]
\end{itemize}
\end{definition}

Now we have defined a cobordism $\mathscr{F}_\bullet$, we show that this is a Legendrian cobordism.
\begin{proposition}
$\mathscr{F}_\bullet$ is a Legendrian cobordism i.e. $\mathscr{F}_\bullet \in Sh_\Lambda(M \times [0,1],\C)$.
\end{proposition}
\begin{proof}
To prove the claim, I will show that the microlocal stalks of $\mathscr{F}_\bullet$ vanishes at every points on a contangent bundle of $M\times [0,1]$.\\
Since $\mathscr{F}^\bullet$ is constant along the time coordinate on $U'^c \times [0,1]$, it is enough to check for the points of $U\times [0,1] \cong D_{r=2}\times [0,1]$. Now consider the following open cover of $D_{r=2}\times [0,1]$
\[
\{\operatorname{star}(s_\bullet(0,-,0,0)),\operatorname{star}(s_\bullet(0,0,-,0))\}
\] 
(1) First, let's show that the microlocal stalks of $\mathscr{F}^\bullet|_{\operatorname{star}(s_\bullet(0,-,0,0))}$ vanishes. Note that there is a diffeomorphism beteween $\operatorname{star}(s_\bullet(0,-,0,0))$ and $\R^3$ that preserves there stratification i.e.
\[
s^3(sgn_1,sgn_2,sgn_3) \mapsto s_\bullet(sgn_1,-,sgn_2,sgn_3)
\]
Then it is enough to prove that the microlocal stalk of the pullback of $\mathscr{F}^\bullet$ along the above diffeomorphism vanishes at every points of $T^*\R^n$. The pullback of $\mathscr{F}^\bullet$ along the diffeomorphism could be described using the following legible diagram, say $F^3$. To simplify the notation, we denote
\[
F^3(sgn_1,sgn_2,sgn_3):= F^3(s^3(sgn_1,sgn_2,sgn_3))
\]
\textbf{Stalks:}
\begin{itemize}
\item $F^3(-,-,-)$ := $\C[-1]$
\item $F^3(-,-,+)$ := $\C[-1]$
\item $F^3(+,-,-)$ := $\C\xrightarrow{\times b} \C$
\item $F^3(+,-,+)$ := $\C\xrightarrow{\times c} \C$
\item $F^3(-,+,-)$ := $\C[-1]$
\item $F^3(-,+,+)$ := $\C[-1]$
\item $F^3(+,+,-)$ := $0$
\item $F^3(+,+,+)$ := $0$
\end{itemize}

\textbf{Generization maps:}\\
\begin{tikzcd}[row sep=1.5cm, column sep=2cm]
& s(-,+,-)\arrow[dd,"{(7)}"] \arrow[rr,"{(6)}"] & & s(-,+,+) \arrow[dd,"{(12)}"]\\
s(-,-,-) \arrow[dd,"{(3)}"]\arrow[rr,"{(1)}"] \arrow[ur,"{(2)}"] & & s(-,-,+) \arrow[dd,"{(5)}"]\arrow[ur,"{(4)}"]& \\
& s(+,+,-) \arrow[rr,"{(10)}"] & & s(+,+,+)  \\
s(+,-,-) \arrow[rr,"{(8)}"] \arrow[ur,"{(9)}"] & & s(+,-,+) \arrow[ur,"{(11)}"] &
\end{tikzcd}
\begin{enumerate}[label = (\arabic*)]
\item 
\begin{tikzcd}
\C \arrow[r,"\times 1"] & \C \\
0 \arrow[u]\arrow[r] & 0 \arrow[u,]
\end{tikzcd}

\item 
\begin{tikzcd}
\C \arrow[r,"\times 1"] & \C \\
0 \arrow[u]\arrow[r] & 0 \arrow[u,]
\end{tikzcd}

\item 
\begin{tikzcd}
\C \arrow[r,"\times 1"] & \C \\
0 \arrow[u]\arrow[r] & \C\arrow[u,"\times b"]
\end{tikzcd}

\item 
\begin{tikzcd}
\C \arrow[r, "\times 1"] & \C \\
0 \arrow[u,]\arrow[r,] & 0\arrow[u,]
\end{tikzcd}

\item 
\begin{tikzcd}
\C \arrow[r,"\times 1"] & \C \\
0 \arrow[u,]\arrow[r,] & \C\arrow[u,"\times c"]
\end{tikzcd}

\item 
\begin{tikzcd}
\C \arrow[r, "\times 1"] & \C \\
0 \arrow[u,]\arrow[r,] & 0\arrow[u,]
\end{tikzcd}

\item 
\begin{tikzcd}
\C \arrow[r] & 0 \\
0 \arrow[u,]\arrow[r] & 0 \arrow[u,]
\end{tikzcd}

\item 
\begin{tikzcd}
\C \arrow[r,"\times 1"] & \C \\
\C \arrow[u,"\times b"]\arrow[r,"\times bc^{-1}"] & \C\arrow[u,"\times c"]
\end{tikzcd}

\item 
\begin{tikzcd}
\C \arrow[r] & 0 \\
\C \arrow[u, "\times b"]\arrow[r,] & 0\arrow[u,]
\end{tikzcd}

\item 
\begin{tikzcd}
0 \arrow[r,] & 0 \\
0 \arrow[u,]\arrow[r,] & 0 \arrow[u,]
\end{tikzcd}

\item 
\begin{tikzcd}
\C \arrow[r,] & 0 \\
\C \arrow[u,"\times c"]\arrow[r] & 0\arrow[u,]
\end{tikzcd}

\item 
\begin{tikzcd}
\C \arrow[r,] & 0 \\
0 \arrow[u,]\arrow[r,] & 0\arrow[u,]
\end{tikzcd}
\end{enumerate}
To prove that microlocal stalk vanishes everywhere, by Lemma \ref{morse}, it is enough to show that the total complexes of $F^3$ restricted to the following squares and cubes are acyclic

\begin{enumerate}[label = (\roman*)]
\item 
\begin{tikzcd}
F^3(-,-,-) \arrow[r]\arrow[d] & F^3(-,+,-) \arrow[d] \\
F^3(+,-,-) \arrow[r] & F^3(+,+,-)
\end{tikzcd}
=
\begin{tikzcd}
\C[-1] \arrow[r,]\arrow[d,] & \C[-1] \arrow[d, ] \\
\C\xrightarrow{\times b}\C \arrow[r,] & 0
\end{tikzcd}


\item 
\begin{tikzcd}
F^3(-,-,+) \arrow[r]\arrow[d] & F^3(-,+,+) \arrow[d] \\
F^3(+,-,+) \arrow[r] & F^3(+,+,+)
\end{tikzcd}
=
\begin{tikzcd}
\C[-1] \arrow[r,]\arrow[d,] & \C[-1] \arrow[d,] \\
\C\xrightarrow{\times c}\C \arrow[r,] & 0
\end{tikzcd}

\item 
\begin{tikzcd}
F^3(-,-,-) \arrow[r]\arrow[d] & F^3(-,-,+) \arrow[d] \\
F^3(+,-,-) \arrow[r] & F^3(+,-,+)
\end{tikzcd}
=
\begin{tikzcd}
\C[-1] \arrow[r,] \arrow[d,] & \C[-1] \arrow[d,] \\
\C\xrightarrow{\times b} \C \arrow[r, ] & \C\xrightarrow{\times c}\C
\end{tikzcd}

\item 
\begin{tikzcd}
F^3(-,+,-) \arrow[r]\arrow[d] & F^3(-,+,+) \arrow[d] \\
F^3(+,+,-) \arrow[r] & F^3(+,+,+)
\end{tikzcd}
=
\begin{tikzcd}
\C[-1] \arrow[r,] \arrow[d,] & \C[-1] \arrow[d,] \\
0 \arrow[r, ] & 0
\end{tikzcd}

\item 
\begin{tikzcd}
F^3(-,-,-) \arrow[r]\arrow[d] & F^3(-,-,+) \arrow[d] \\
F^3(-,+,-) \arrow[r] & F^3(-,+,+)
\end{tikzcd}
=
\begin{tikzcd}
\C[-1] \arrow[r,] \arrow[d,] & \C[-1] \arrow[d,] \\
\C[-1] \arrow[r,] & \C[-1] \\
\end{tikzcd}

\item 
\begin{tikzcd}
F^3(+,-,-) \arrow[r]\arrow[d] & F^3(+,-,+) \arrow[d] \\
F^3(+,+,-) \arrow[r] & F^3(+,+,+)
\end{tikzcd}
=
\begin{tikzcd}
\C\xrightarrow{\times b}\C \arrow[r,]\arrow[d,] & \C\xrightarrow{\times c}\C\arrow[d] \\
0 \arrow[r] & 0
\end{tikzcd}

\item the cubic diagram:\\
\begin{tikzcd}[row sep=1.5cm, column sep=2cm]
& s(-,+,-)\arrow[dd,"{s(0,+,-)}"] \arrow[rr,"{s(-,+,0)}"] & & s(-,+,+) \arrow[dd,"{s(0,+,+)}"]\\
s(-,-,-) \arrow[dd,"{s(0,-,-)}"]\arrow[rr,"{s(-,-,0)}"] \arrow[ur,"{s(-,0,-)}"] & & s(-,-,+) \arrow[dd,"{s(0,-,+)}"]\arrow[ur,"{s(-,0,+)}"]& \\
& s(+,+,-) \arrow[rr,"{s(+,+,0)}"] & & s(+,+,+)  \\
s(+,-,-) \arrow[rr,"{s(+,-,0)}"] \arrow[ur,"{s(+,0,-)}"] & & s(+,-,+) \arrow[ur,"{s(+,0,+)}"] &
\end{tikzcd}\\
=\\
\begin{tikzcd}[row sep=2cm, column sep=2.5cm]
& \C[-1] \arrow[dd,]\arrow[rr,] & & \C[-1] \arrow[dd,] \\
\C[-1] \arrow[dd,]\arrow[rr,] \arrow[ur,] & & \C[-1] \arrow[dd,]\arrow[ur,]& \\
& 0 \arrow[rr,] & & 0  \\
\C\xrightarrow{\times b}\C \arrow[rr,] \arrow[ur,] & & \C\xrightarrow{\times c}\C \arrow[ur,] &
\end{tikzcd}
For \Rn{1}-\Rn{6}, horizontal cochain map in each degree are quasi-isomorphism. Therefore, the total complex is acyclic.\\
For \Rn{7}, $\begin{tikzcd} \C \arrow[r,"\times 1"] & \C \\ 0 \arrow[u]\arrow[r] & 0\arrow[u] \end{tikzcd}$, $\begin{tikzcd} \C \arrow[r,"\times 1"] & \C \\ 0 \arrow[u]\arrow[r] & 0\arrow[u] \end{tikzcd}$, $\begin{tikzcd} \C \arrow[r,"\times 1"] & \C \\ \C \arrow[u, "\times b"]\arrow[r, "\times bc^{-1}"] & \C\arrow[u, "\times c"] \end{tikzcd}$, $\begin{tikzcd} 0 \arrow[r,] & 0 \\ 0 \arrow[u]\arrow[r] & 0\arrow[u] \end{tikzcd}$ are isomorphisms. Therefore, we can think of the cube diagram as isomorphism of two double complexes. Therefore, the total complex is acyclic.
\end{enumerate}

(2) Next, let's show that the microlocal stalks of $\mathscr{F}^\bullet|_{\operatorname{star}(s_\bullet(0,0,-,0))}$ vanishes. Note that there is a diffeomorphism beteween $\operatorname{star}(s_\bullet(0,0,-,0))$ and $\R^3$ that preserves there stratification i.e.
\[
s^3(sgn_1,sgn_2,sgn_3) \mapsto s_\bullet(sgn_1,sgn_2,-,sgn_3)
\]
Then it is enough to prove that the microlocal stalk of the pullback of $\mathscr{F}^\bullet$ along the above diffeomorphism vanishes at every points of $T^*\R^n$. The pullback of $\mathscr{F}^\bullet$ along the diffeomorphism could be described using the following legible diagram, say $F^3$. To simplify the notation, we denote
\[
F^3(sgn_1,sgn_2,sgn_3):= F^3(s^3(sgn_1,sgn_2,sgn_3))
\]
\textbf{Stalks:}
\begin{itemize}
\item $F^3(-,-,-)$ := $\C[-1]$
\item $F^3(-,-,+)$ := $\C[-1]$
\item $F^3(+,-,-)$ := $\C\xrightarrow{\times b} \C$
\item $F^3(+,-,+)$ := $\C\xrightarrow{\times c} \C$
\item $F^3(-,+,-)$ := $\C\xrightarrow{\times a}\C$
\item $F^3(-,+,+)$ := $0$
\item $F^3(+,+,-)$ := $\C^2\xrightarrow{(b~a)}\C$
\item $F^3(+,+,+)$ := $\C$
\end{itemize}

\textbf{Generization maps:}\\
\begin{tikzcd}[row sep=1.5cm, column sep=2cm]
& s(-,+,-)\arrow[dd,"{(7)}"] \arrow[rr,"{(6)}"] & & s(-,+,+) \arrow[dd,"{(12)}"]\\
s(-,-,-) \arrow[dd,"{(3)}"]\arrow[rr,"{(1)}"] \arrow[ur,"{(2)}"] & & s(-,-,+) \arrow[dd,"{(5)}"]\arrow[ur,"{(4)}"]& \\
& s(+,+,-) \arrow[rr,"{(10)}"] & & s(+,+,+)  \\
s(+,-,-) \arrow[rr,"{(8)}"] \arrow[ur,"{(9)}"] & & s(+,-,+) \arrow[ur,"{(11)}"] &
\end{tikzcd}
\begin{enumerate}[label = (\arabic*)]
\item 
\begin{tikzcd}
\C \arrow[r,"\times 1"] & \C \\
0 \arrow[u]\arrow[r] & 0 \arrow[u,]
\end{tikzcd}

\item 
\begin{tikzcd}
\C \arrow[r,"\times 1"] & \C \\
0 \arrow[u]\arrow[r] & \C \arrow[u,"\times a"]
\end{tikzcd}

\item 
\begin{tikzcd}
\C \arrow[r,"\times 1"] & \C \\
0 \arrow[u]\arrow[r] & \C\arrow[u,"\times b"]
\end{tikzcd}

\item 
\begin{tikzcd}
\C \arrow[r, ] & 0 \\
0 \arrow[u,]\arrow[r,] & 0\arrow[u,]
\end{tikzcd}

\item 
\begin{tikzcd}
\C \arrow[r,"\times 1"] & \C \\
0 \arrow[u,]\arrow[r,] & \C\arrow[u,"\times c"]
\end{tikzcd}

\item 
\begin{tikzcd}
\C \arrow[r, ] & 0 \\
\C \arrow[u,"\times a"]\arrow[r,] & 0\arrow[u,]
\end{tikzcd}

\item 
\begin{tikzcd}
\C \arrow[r] & \C \\
\C \arrow[u,"\times a"]\arrow[r,"\iota_1"] & \C^2 \arrow[u,"(b~a)"]
\end{tikzcd}

\item 
\begin{tikzcd}
\C \arrow[r,"\times 1"] & \C \\
\C \arrow[u,"\times b"]\arrow[r,"\times bc^{-1}"] & \C\arrow[u,"\times c"]
\end{tikzcd}

\item 
\begin{tikzcd}
\C \arrow[r,"\times 1"] & \C \\
\C \arrow[u, "\times b"]\arrow[r,"\iota_0"] & \C^2\arrow[u,"(b~a)"]
\end{tikzcd}

\item 
\begin{tikzcd}
\C \arrow[r,] & 0 \\
\C^2 \arrow[u,"(b~a)"]\arrow[r,"(bc^{-1}~0)"] & \C \arrow[u,]
\end{tikzcd}

\item 
\begin{tikzcd}
\C \arrow[r,] & 0 \\
\C \arrow[u,"\times c"]\arrow[r, "\times 1"] & \C\arrow[u,]
\end{tikzcd}

\item 
\begin{tikzcd}
0 \arrow[r,] & 0 \\
0 \arrow[u,]\arrow[r,] & \C\arrow[u,]
\end{tikzcd}
\end{enumerate}
To prove that microlocal stalk vanishes everywhere, by Lemma \ref{morse}, it is enough to show that the total complexes of $F^3$ restricted to the following squares and cubes are acyclic

\begin{enumerate}[label = (\roman*)]
\item 
\begin{tikzcd}
F^3(-,-,-) \arrow[r]\arrow[d] & F^3(-,+,-) \arrow[d] \\
F^3(+,-,-) \arrow[r] & F^3(+,+,-)
\end{tikzcd}
=
\begin{tikzcd}
\C[-1] \arrow[r,]\arrow[d,] & \C\xrightarrow{\times a}\C \arrow[d, ] \\
\C\xrightarrow{\times b}\C \arrow[r,] & \C^2\xrightarrow{(b~a)}\C
\end{tikzcd}


\item 
\begin{tikzcd}
F^3(-,-,+) \arrow[r]\arrow[d] & F^3(-,+,+) \arrow[d] \\
F^3(+,-,+) \arrow[r] & F^3(+,+,+)
\end{tikzcd}
=
\begin{tikzcd}
\C[-1] \arrow[r,]\arrow[d,] & 0 \arrow[d,] \\
\C\xrightarrow{\times c}\C \arrow[r,] & \C
\end{tikzcd}

\item 
\begin{tikzcd}
F^3(-,-,-) \arrow[r]\arrow[d] & F^3(-,-,+) \arrow[d] \\
F^3(+,-,-) \arrow[r] & F^3(+,-,+)
\end{tikzcd}
=
\begin{tikzcd}
\C[-1] \arrow[r,] \arrow[d,] & \C[-1] \arrow[d,] \\
\C\xrightarrow{\times b} \C \arrow[r, ] & \C\xrightarrow{\times c}\C
\end{tikzcd}

\item 
\begin{tikzcd}
F^3(-,+,-) \arrow[r]\arrow[d] & F^3(-,+,+) \arrow[d] \\
F^3(+,+,-) \arrow[r] & F^3(+,+,+)
\end{tikzcd}
=
\begin{tikzcd}
\C\xrightarrow{\times a}\C \arrow[r,] \arrow[d,] & 0 \arrow[d,] \\
\C^2\xrightarrow{(b~a)}\C \arrow[r, "(bc^{-1}~0)"] & \C
\end{tikzcd}

\item 
\begin{tikzcd}
F^3(-,-,-) \arrow[r]\arrow[d] & F^3(-,-,+) \arrow[d] \\
F^3(-,+,-) \arrow[r] & F^3(-,+,+)
\end{tikzcd}
=
\begin{tikzcd}
\C[-1] \arrow[r,] \arrow[d,] & \C[-1] \arrow[d,] \\
\C\xrightarrow{\times a}\C \arrow[r,] & 0 \\
\end{tikzcd}

\item 
\begin{tikzcd}
F^3(+,-,-) \arrow[r]\arrow[d] & F^3(+,-,+) \arrow[d] \\
F^3(+,+,-) \arrow[r] & F^3(+,+,+)
\end{tikzcd}
=
\begin{tikzcd}
\C\xrightarrow{\times b}\C \arrow[r]\arrow[d,] & \C\xrightarrow{\times c}\C\arrow[d] \\
\C^2\xrightarrow{(b~a)}\C \arrow[r] & \C
\end{tikzcd}

\item the cubic diagram:\\
\begin{tikzcd}[row sep=1.5cm, column sep=2cm]
& s(-,+,-)\arrow[dd,"{s(0,+,-)}"] \arrow[rr,"{s(-,+,0)}"] & & s(-,+,+) \arrow[dd,"{s(0,+,+)}"]\\
s(-,-,-) \arrow[dd,"{s(0,-,-)}"]\arrow[rr,"{s(-,-,0)}"] \arrow[ur,"{s(-,0,-)}"] & & s(-,-,+) \arrow[dd,"{s(0,-,+)}"]\arrow[ur,"{s(-,0,+)}"]& \\
& s(+,+,-) \arrow[rr,"{s(+,+,0)}"] & & s(+,+,+)  \\
s(+,-,-) \arrow[rr,"{s(+,-,0)}"] \arrow[ur,"{s(+,0,-)}"] & & s(+,-,+) \arrow[ur,"{s(+,0,+)}"] &
\end{tikzcd}\\
=\\
\begin{tikzcd}[row sep=2cm, column sep=2.5cm]
& \C\xrightarrow{\times a}\C \arrow[dd,]\arrow[rr,] & & 0 \arrow[dd,] \\
\C[-1] \arrow[dd,]\arrow[rr,] \arrow[ur,] & & \C[-1] \arrow[dd,]\arrow[ur,]& \\
& \C^2\xrightarrow{(b~a)}\C \arrow[rr,] & & \C  \\
\C\xrightarrow{\times b}\C \arrow[rr,] \arrow[ur,] & & \C\xrightarrow{\times c}\C \arrow[ur,] &
\end{tikzcd}
For \Rn{3}-\Rn{6}, the horizontal cochain map in each degree are quasi-isomorphism. Therefore, the total complex is acyclic.\\
For \Rn{1} and \Rn{2}, straightforward calculation shows that the total complexes are acyclic.\\
For \Rn{7}, $\begin{tikzcd} \C \arrow[r,"\times 1"] & \C \\ 0 \arrow[u]\arrow[r] & 0\arrow[u] \end{tikzcd}$, $\begin{tikzcd} \C \arrow[r,] & 0 \\ \C \arrow[u,"\times a"]\arrow[r] & 0\arrow[u] \end{tikzcd}$, $\begin{tikzcd} \C \arrow[r,"\times 1"] & \C \\ \C \arrow[u, "\times b"]\arrow[r, "\times bc^{-1}"] & \C\arrow[u, "\times c"] \end{tikzcd}$, $\begin{tikzcd} \C \arrow[r,] & 0 \\ \C^2 \arrow[u,"(b~a)"]\arrow[r, "(bc^{-1}~0)"] & \C\arrow[u,] \end{tikzcd}$ are quasi-isomorphisms. Therefore, we can think of the cube diagram as quasi-isomorphism of two double complexes. Therefore, the total complex is acyclic.
\end{enumerate}
Therefore, the proof is complete.
\end{proof}

%%%%%%%%%%%%%%%%%%%%%%%%%%%%%%%%%%%%%%%%%%%%%%%%%%%%%%%%%%%%%%%%%%%
%%                       Terminal Sheaf                          %%
%%%%%%%%%%%%%%%%%%%%%%%%%%%%%%%%%%%%%%%%%%%%%%%%%%%%%%%%%%%%%%%%%%%
\subsection*{Sheaf at the End}
In this subsection, I will describe the sheaf $\mathscr{F}_1$ at the end of the $cobord'_2$. By Mayer-Vietoris, $\mathscr{F}_1:= \mathscr{F}_\bullet|_{M\times\{1\}}$ on $M \cong M\times\{1\}$ is equivalent to the following data
\begin{itemize}
\item a sheaf on $V$, say $\mathscr{F}_{V}$

\item a sheaf on $D_{r=2}$, say $\mathscr{F}_{D_{r=2}}$

\item a gluing isomorphsim $\gamma_1 : f_*\mathscr{F}_{D_{r=2}}|_{U\cap V} \xrightarrow{\sim} \mathscr{F}_{V}|_{U\cap V}$.
\end{itemize}

\subsubsection{A. Sheaf on $V$}
First, a sheaf on $V\cong V\times\{1\}$ is the restriction of $\mathscr{F}_{V\times [0,1]}$ to $V\times \{1\}$, i.e. $\mathscr{F}_{V\times [0,1]}|_{V\times \{1\}}= pr_1^*(\mathscr{F}_0|_V)|_{V\times \{1\}} = \mathscr{F}_0|_V$.
\subsubsection{B. Sheaf on $D_{r=2}$}
Next, a sheaf on $D_{r=2}\cong D_{r=2}\times \{1\}$ is the restriction of $\mathscr{F}_{D_{r=2}\times [0,1]}$ to $D_{r=2}\times \{1\}$, i.e. $\mathscr{F}_{D_{r=2}\times [0,1]} |_{D_{r=2}\times \{1\}}$. I will describe it as a squiggly legible diagram $F_1$ which is the restriction of $F_\bullet$ defined in the previous section.

For simplicity, we use the following notations
\[
F_1(sgn_1,sgn_2,sgn_3,sgn_4):= F_1(s_1(sgn_1,sgn_2,sgn_3,sgn_4))
\]
\textbf{Stalks:}
\begin{figure}[H]
    \centering
    \includegraphics[scale = 0.45]{diagrams/cobord'2/45.png} 
    \caption{}
    \label{fig:your-label}
\end{figure}
\begin{figure}[H]
    \centering
    \includegraphics[scale = 0.45]{diagrams/cobord'2/46.png} 
    \caption{}
    \label{fig:your-label}
\end{figure}
\begin{itemize}
\item $F_1(-,+,*,-)$ := $\C\xrightarrow{\times a}\C$
\item $F_1(-,+,*,+)$ := $0$
\item $F_1(+,-,-,-)$ := $\C \xrightarrow{\times b} \C $
\item $F_1(+,-,-,+)$ := $\C \xrightarrow{\times c} \C $
\item $F_1(+,-,+,-)$ := $0$
\item $F_1(+,-,+,+)$ := $0$
\item $F_1(+,+,*,-)$ := $\C^2\xrightarrow{(b~a)}\C$
\item $F_1(+,+,*,+)$ := $\C$
\end{itemize}
\textbf{Generization maps:}
\begin{figure}[H]
    \centering
    \includegraphics[scale = 0.45]{diagrams/cobord'2/47.png} 
    \caption{}
    \label{fig:your-label}
\end{figure}
\begin{enumerate}[label = (\arabic*)]
\item \begin{tikzcd}
0 \arrow[r]     & 0  \\
0 \arrow[r]\arrow[u] & \C \arrow[u]
\end{tikzcd}

\item \begin{tikzcd}
\C \arrow[r,"\times 1"]     & \C  \\
\C \arrow[r,"\iota_1"]\arrow[u,"\times a"] & \C^2 \arrow[u,"(b~a)"]
\end{tikzcd}

\item \begin{tikzcd}
\C \arrow[r, "\times 1"]     & \C  \\
\C \arrow[r, "\times bc^{-1}"]\arrow[u,"\times b"] & \C \arrow[u, "\times c"]
\end{tikzcd}

\item \begin{tikzcd}
\C \arrow[r]     & 0  \\
\C \arrow[r]\arrow[u,"\times c"] & 0 \arrow[u]
\end{tikzcd}

\item \begin{tikzcd}
\C \arrow[r]     & 0  \\
\C \arrow[r]\arrow[u,"\times b"] & 0 \arrow[u]
\end{tikzcd}


\item \begin{tikzcd}
\C \arrow[r]     & 0  \\
\C \arrow[r, "\times 1"]\arrow[u,"\times c"] & \C \arrow[u]
\end{tikzcd}

\item \begin{tikzcd}
\C \arrow[r,"\times 1"]     & \C  \\
\C \arrow[r,"\iota_0"]\arrow[u,"\times b"] & \C^2 \arrow[u,"(b~a)"]
\end{tikzcd}

\item \begin{tikzcd}
\C \arrow[r]     & 0  \\
\C^2 \arrow[r, "(bc^{-1}~0)"]\arrow[u, "(b~a)"] & \C \arrow[u]
\end{tikzcd}

\item \begin{tikzcd}
\C \arrow[r]     & 0  \\
\C \arrow[r]\arrow[u, "\times a"] & 0 \arrow[u]
\end{tikzcd}
\end{enumerate}

\subsubsection{C. Gluing Isomorphism}
Lastly, the gluing isomorphism $\gamma_1 := \gamma_\bullet|_{(U\cap V)\times \{1\}}:  f_*\mathscr{F}_{D_{r=2}}|_{(U\cap V)\times \{1\}}\xrightarrow{\sim} \mathscr{F}_0|_{U\cap V}$ is described as follows.

\begin{definition}
we define $\gamma_1$ to be the composition
\begin{align*}
&(f_*\mathscr{F}_{D_{r=2}})|_{(U\cap V)\times \{1\}}\xrightarrow{\sim}pr_1^*(\mathscr{F}_0|_{U\cap V})|_{(U\cap V)\times \{1\}}\xrightarrow{\sim}\mathscr{F}_0|_{U\cap V}
\end{align*}
where
\begin{itemize}
\item the first isomorphism follows from the fact that $(f_*\mathscr{F}_{D_{r=2}\times [0,1]})|_{(U\cap V)\times[0,1]}$ is isomorphic to $pr_1^*(\mathscr{F}_0|_{U\cap V})$.

\item the second isomorphism follows from the fact that the following composition is an identity map:
\[
(U\cap V)\xrightarrow{\sim} (U\cap V)\times \{1\} \hookrightarrow (U\cap V)\times [0,1] \twoheadrightarrow (U\cap V)
\]
\end{itemize}
\end{definition}
\pagebreak